\documentclass[../book.tex]{subfiles}

\begin{document}

%\section{Conventions}
% Von Neumann Ordinals (specifially, Naturals) 
% \subseteq as subset
% 
% Add to this list as you go on. 


% Objects, Maps, Subobjects (3rd Iso), Quotient Objects (1st Iso)
% for Groups, Vector spaces, Rings, Fields

% Groups
\subsection{Groups}
%   - Def - Group
\begin{dfn} Group. 

A \textbf{group} is a pair $(G, \cdot)$, where $G$ is a set and $\cdot : G\times G \rightarrow G$ is an binary operation, satisfying the four group axioms:
\begin{enumerate}
\item Closure.\footnote{Although it is included in the definition of binary operation, we will include it to emphasis its importance.}  $ \forall a,b\in G$, $a\cdot b \in G$.
\item Associativity. $ \forall a,b,c\in G$, $a \cdot (b \cdot c) = (a \cdot b) \cdot c$.
\item Identity. $ \exists e_G \in G$ such that for all $a \in G$, 
$e_G \cdot a = a \cdot e_G=a$.
\item Inverse. $ \forall a\in G$,  $ \exists a^{-1} \in G$ such that $a \cdot a^{-1} = a^{-1} \cdot a = e$.
\end{enumerate}
For convenience, we write $ab$ for $a \cdot b$, 
as well as write $G$ for $(G,\cdot)$ 
unless the group multiplication is of significance. 
A group $G$ is called \textbf{abelian} 
if the group operation is \textbf{commutative}, that is
\[\forall a, b \in G, ab = ba.\]
In this case, the group ``multiplication" is often denoted with a plus ``$+$"
and the group identity with ``$0$".

\end{dfn}

%   - Ex - Unique identity and inverse

\begin{ex} Uniqueness of Identity and Inverse.

    Prove that the identity element in a group is unique, denoted by $e$ and every element $g \in G$ has a unique inverse, denoted by $g^{-1}$. And thus $(g^{-1})^{-1}=g$.
\end{ex}

%\begin{ex} 
%    Verify that the following are groups: 
%    \begin{enumerate}
%        \item $(\Z, +), (\Q, +), (\R,+), (\C,+)$ all abelian
%        \item $(\Q\backslash\{0\}, \cdot), (\R\backslash\{0\}, \cdot),
%        (\C\backslash\{0\}, \cdot)$ all abelian
%        \item Let $n \in \N$. 
%        $C(\R^n) := \{f : \R^n \to \R \mid f \text{ continuous}\}$.
%        Then $(C(\R^n),+)$ where $+$ is pointwise addition. 
%        \item Let $X$ be a set, $(G, \cdot)$ a group. 
%        Define $G^X := \{f : X \to G\}$. 
%        Then $(G^X, \cdot)$ where $\cdot$ is pointwise multiplication. 
%    \end{enumerate}
%    Show that the following are \emph{not} groups: 
%    \begin{enumerate}
%       \item $(\R, \cdot)$
%        \item $(C(\R^n),\cdot)$ where $\cdot$ is pointwise multiplication.
%        \item Calle Non-Group. 
%    \end{enumerate}
%\end{ex}

%   - Def - Group Homomorphism, Isomorphism, Kernel
\begin{dfn} Morphism of Groups, Isomorphisms, Image, Kernel. 

Let $(G,\cdot)$ and $(H,\circ)$ be two groups. 
Then a \textbf{morphism of groups} (AKA group homomorphism) from $G$ to $H$ is a mapping $\phi : G \to H$ 
satisfying the following condition for all $g_1, g_2 \in G$:
\[\phi(g_1 \cdot g_2)=\phi(g_1) \circ \phi(g_2).\]
In ``the world of groups", 
morphisms of groups are really the only functions we will consider between groups,
since they preserve group multiplication. 
So we use ``$\phi : G \to H$ in $\Grp$" to say $\phi$ is a morphism of groups. 
\footnote{$\Grp$ is short-hand for "the world of groups".
    This idea is made precise by the concept of a \emph{category},
    but we will not delve into that here. 
}
If $\phi$ is bijective, we say $G$ and $H$ are \textbf{isomorphic as groups}, 
denoted by $G \iso_{\Grp} H$ and $\phi$ is called an \textbf{isomorphism}. 
Notice that a morphism of groups preserve identity (i.e. $\phi(e_G)=e_H$), 
inverse and powers.

If $f:G \to H, g: H \to K$ both in $\Grp$, then $g \circ f: G \to K$ in $\Grp$. 
Also the identity mapping is an isomorphism, i.e. $G \iso_{\Grp} G$.

Given $\phi: G \to H$ in $\Grp$, we define
\begin{enumerate}
    \item The \textbf{image} of $\phi$ as 
        $\im \phi=\{h \in H \mid \exists g \in G, \phi(g)=h\}$;
    \item The \textbf{kernel} of $\phi$ as 
        $\ker \phi=\{g \in G \mid \phi(g)=e_H\}$.
\end{enumerate}
\end{dfn}
\ex{Prove that both $\im  \phi$ and $\text{ker } \phi$ are groups 
    under the multiplication from the groups they are respectively subsets of. 
}

%   - Ex - Homomorphism Inject iff Kernel trivial
\ex{Prove that $\phi$ is injective if and only if $\text{ker }\phi=\{e_G\}$.}

%   - Def - Subgroups
\begin{dfn} Subgroups. 

    Let $G$ be a group, $H \subseteq G$.
    Then the following are equivalent: 
    \begin{enumerate}
        \item $H$ forms a group under the multiplication from $G$. 
        \item $H$ is the image of an injective morphism of groups into $G$. 
    \end{enumerate}
    If either of the above is true, $H$ is called a \emph{subgroup of $G$}.
    This is denoted with $H \leq_{\Grp} G$. 
    If it is unambiguous that these are groups,
    we omit the subscript $\Grp$. 
    
\end{dfn}

\begin{eg} Subgroups. 
    Images and kernels of morphisms of groups.
\end{eg}
%   - Ex - Arbitrary intersection of subgroups is subgroup
\ex{Let $(H_i)_{i \in I}$ be subgroups of $G$. 
    Prove that $\bigcap_{i \in I} H_i \leq G$. 
    Hence deduce that 
    \[
        \forall S \subseteq G, \exists ! \<S\> \leq G, 
        S \subseteq \< S \> \land 
        \forall H \leq G, S \subseteq H \imp \< S \> \subseteq H. 
    \]
    i.e. $\< S \>$ is the \emph{smallest} subgroup containing $S$.
    $\< S \>$ is called the \textbf{subgroup generated by $S$}. 
    If $S = \{x\}$ is a singleton set, 
    we write $\< x \>$ instead of $\< \{x\} \>$. 
}

\begin{ex} Alternative construction for Subgroup Generated by a Subset.

    Let $S \subseteq G$. 
    Define \[
        \tilde{S} := \{g \in G \mid \exists s \in S, g = s \lor g = s^{-1}\}
    \]
    For $n \in \N$, 
    \[
        \tilde{S}^0 = \{e\} \,\,\,\,\,\,\,
        \tilde{S}^n := \{g_1\cdots g_n \in G \mid g_i \in \tilde{S}\}
    \]
    Show that $\< S \> = \bigcup_{n \in \N} \tilde{S}^n$. 
\end{ex}

\begin{ex} [Skippable] Construction of Free Groups. 

    Let $S$ be a set. We proceed to turn $S$ into a group.
    The idea is essentially the same as the alternative 
    construction for the subgroup generated by a subset.
    The only difference is that for subgroups, 
    we had an ambient group structure to use,
    but for sets "sitting in a vacuum", 
    we will have to make a group structure from scratch.
    
    Define 
    \[\tilde{S} := (\{1\}\times S) \cup (\{-1\}\times S)\]
    Let $s$ denote $(1,s) \in \{1\} \times S$ and
    $s^{-1}$ denote $(-1,s) \in \{-1\} \times S$. 
    For $n \in \N$, define the \textbf{$\tilde{S}$-strings of length $n$} as
    \[
        \tilde{S}^n := \{ s : n \to \tilde{S} \}
    \]
    For $s \in \tilde{S}^n, i \in n$, denote $s_i$ as $s(i)$.
    So $s$ can be seen as an $n$-tuple $(s_i)_{i \in n}$\footnote{If you like,
    this is a way of formally
    defining a tuple.},
    and we write $s_0\cdots s_{n-1}$ for the function $s$ instead.
    This should be thought of as finite products of elements in $\tilde{S}$.
    So our to-be group is the set of all such products, 
    $\bigsqcup_{n \in \N} \tilde{S}^n$.
    For 
    $s : n \to \tilde{S}, t : m \to \tilde{S} \in \bigsqcup_{n \in \N} \tilde{S}^n$,
    we define the multiplication as concatenation of strings: 
    \[
        st : n+m \to \tilde{S}, i \mapsto 
        \begin{cases}
            s(i), & i < n \\
            t(i-n), & i \geq n
        \end{cases}
    \]
    Then clearly this multiplication is associative and 
    we have the empty string $e : 0 \to \tilde{S}$ as the identity. 
    
    We are not quite finished yet.
    As of now, for $s \in \tilde{S}$, 
    $ss^{-1}$, $s^{-1}s$ and $e$ are \emph{different} elements in our to-be group,
    i.e. we have not defined cancellation by inverses. 
    This is done as follows. 
    For $n \leq m \in \N$, $a \in S$,  
    define the \textbf{$n$-th Chirs-Frod insertion of $a$}
    as 
    \[
        CF_n : \tilde{S}^m \to \tilde{S}^{m+2}, 
        s_0\cdots s_{m-1} \mapsto 
        \begin{cases}
        aa^{-1} & \text{, if } 0 = m\\
        s_0 \cdots s_{n-1} a a^{-1} s_n \cdots s_{m-1} & \text{, if } 0 < m
        \end{cases}
    \]
    i.e. insert $aa^{-1}$ after the $n$-th symbol. 
    Similarly, we define the \textbf{$n$-th Frod-Chirs insertion of $a$}, $FC_n$, 
    as inserting $a^{-1}a$ after the $n$-th symbol. 
    Then for 
    $s_0\cdots s_{n-1}, t_0\cdots t_{m-1} \in \bigsqcup_{n \in \N} \tilde{S}^n$, 
    we define $s_0\cdots s_{n-1} \sim t_0\cdots t_{m-1}$ when
    they are equal or there exists a Chirs-Frod or Frod-Chirs insertion
    mapping one to the other. 
    
    Prove that $\sim$ is an equivalence relation on 
    $\bigsqcup_{n \in \N} \tilde{S}^n$
    and the multiplication of strings respects the equivalence classes of $\sim$,
    that is
    \[
        s \sim f \land t \sim g \imp st \sim fg
    \]
    Hence show that $\< S \> := 
    \bigsqcup_{n \in \N} \tilde{S}^n / \sim$
    forms a group with the multiplication carried over to equivalence classes. 
    
    $\< S \>$ is called the \textbf{free group generated by $S$}. 
    A free group is one that is isomorphic to a free group generated by a set. 
\end{ex}

\begin{ex} [Skippable] The ``Characterising" Property of Free Groups. 

    Let $\< S \>$ be the free group generated by a set $S$. 
    We have a natural inclusion $\iota : S \to \< S \>, s \mapsto s$.
    Let $G$ be a group, $f : S \to G$ any function. 
    Show that 
    $\exists ! \< f \> : \< S \> : \<G\> \text{ in } \Grp, 
    \< f \> \circ \iota = f$. 
    Diagrammatically, 
    \begin{figure} [ht]
        \centering
        \begin{tikzcd}
        & \< S \> \arrow[dd,"\exists ! \< f \>",dashed]\\
        S \arrow{ru}{\iota} \arrow{rd}{f} & \\
        & G \\
        \end{tikzcd}
    \end{figure}
    
    This can be interpreted as saying $\< S \>$ is the 
    ``smallest" group containing $S$. 
    
    Show that any other group with the property above is automatically
    isomorphic to $\< S \>$ as groups. 
    Thus, this property characterises $\< S \>$ uniquely
    up to isomorphisms of groups. 
    Hence, we never have to think about 
    the construction of $\< S \>$ ever again.
    \footnote{
        This idea of determining an object
        as the "best" with a certain property will appear again and again. 
        It is formalised by the concept of \emph{universal properties},
        which we shall not explore explicitly here. 
    }
\end{ex}

\begin{ex} Automorphism Group of a Set. 

    Let $X$ a set. Prove that the \emph{set-theoretic automorphisms of $X$}
    \[Aut_\Set(X) := \{f : X \to X \mid f \text{ bijects }\}\] forms a group
    with function composition as multiplication, i.e.
    \[
        Aut_\Set(X) \times Aut_\Set(X) \to Aut_\Set(X), 
        (f, g) \mapsto fg := f \circ g
    \]
\end{ex}

\begin{rmk}

    One should see $Aut_\Set(X)$ as all the symmetries of $X$. 
    Since a set $X$ does not have any ``internal structure", 
    $Aut_\Set(X)$ simply consists of all permutations of elements of $X$. 
    Thus, when $X$ has more ``structure", 
    we will often be considering a subgroup of $Aut_\Set(X)$ instead, 
    as the following simple example illustrates. 
    
\end{rmk}

\begin{eg}
    Let $X = \{v_0, v_1, v_2, v_3 \in \R^2\}$ be 
    vertices of a square centred at the origin, labelled counterclockwise. 
    Let $(a, b, c, d)$ denote the permutation of $X$ sending
    $v_0 \mapsto a, v_1 \mapsto b, v_2 \mapsto c, v_3 \mapsto d$. 
    We have the following set of ``symmetries" of $X$ as a square, 
    \begin{align*}
        \{
            &(v_0, v_1, v_2, v_3), (v_1, v_2, v_3, v_0), 
            (v_2, v_3, v_0, v_1), (v_3, v_0, v_1, v_2), \\
            &(v_0, v_3, v_2, v_1), (v_1, v_0, v_3, v_2), 
            (v_2, v_1, v_0, v_3), (v_3, v_2, v_1, v_0)
        \} \subseteq Aut_\Set(X)
    \end{align*}
    
    \ex{Prove that the symmetries above form a subgroup of $Aut_\Set(X)$.}
\end{eg}

\begin{rmk}
    In general, given a group $G$ and a set $X$, 
    we may want to interpret $G$ as ``symmetries" of $X$. 
    This is made precise via a \emph{group action}. 
\end{rmk}

%   - Def - Group Action
\begin{dfn} (Left) Group Actions. 

    Let $G$ be a group. 
    Then a \emph{G-Set} is a pair $(X, \rho)$ where 
    $X$ is a set and $\rho : G \to Aut_\Set(X)$ in $\Grp$.
    We say \emph{$G$ acts on $X$ via $\rho$} or 
    \emph{$\rho$ is a $G$-action on $X$}. 
    For $g \in G$, we denote $g_\rho := \rho(g) : X \to X$. 
    If the action $\rho$ is unambiguous, we use $X$ instead of $(X,\rho)$
    and $g$ instead of $g_\rho$.
    
\end{dfn}

Just as we had morphisms of groups to compare groups, 
given a group $G$, we also have \emph{morphisms of $G$-sets}. 

% Should sub-G-sets be included formally?
\begin{dfn} Morphism of $G$-Sets. 

    Let $(X,\rho), (Y,\sigma)$ be $G$-sets, $f : X \to Y$ a function. 
    Then $f$ is a \textbf{morphism of $G$-sets} when
    \[
        \forall g \in G, f \circ g_\rho = g_\sigma \circ f
    \]
    Morphisms of $G$-sets are essentially 
    the functions between $G$-sets that preserve the actions from $G$. 
    Thus, they are really the only maps to consider in ``the world of $G$-sets".
    Hence we write ``$f : X \to Y$ in $G\mhyph\Set$". 
    If $f$ bijects, it is called an \emph{isomorphism of G-sets}. 
    We then denote $(X,\rho) \iso_{G\mhyph\Set} (Y,\sigma)$. 
    
\end{dfn}

\ex{ Let $f : X \to Y$ in $G\mhyph\Set$, $\im f$ be the usual set-theoretic image.
    Show that $\forall g \in G, 
    \res{g}{\im  f} \in Aut_\Set(\im  f) \leq_\Grp Aut_\Set(Y)$,
    i.e. $\im f$ forms a $G$-set from the $G$-action on $Y$.  
}

\begin{dfn} Sub-$G$-Sets.
    Let $X$ be a $G$-set, $Y \subseteq X$. 
    Then the following are equivalent: 
    \begin{enumerate}
        \item $Y$ forms a $G$-set from the $G$-action on $X$.
        \item $Y$ is the image of an injective morphism of $G$-sets.
    \end{enumerate}
    If either of the above is true, 
    we say $Y$ is a \textbf{sub-$G$-set} of $X$,
    denoted $Y \leq_{G\mhyph\Set} X$. 
    If it is clear that these are $G$-sets, 
    we omit the subcript $G\mhyph\Set$.
    
\end{dfn}

%   - Def - Orbit
\begin{dfn} Orbits under an Action. 

    Let $(X, \rho)$ be a $G$-set, $x \in X$. 
    Then the \emph{orbit of $x$ under $\rho$} is defined as
    \[
        Orb_\rho(x) := \{g_\rho(x) \in X \mid g \in G\}
    \]
    The set of orbits is denoted $X / G$.
    If the action is unambiguous, we use $Orb(x)$ instead of $Orb_\rho(x)$. 

\end{dfn}

%   - Ex - Orbits are equivalence classes
\ex{[Important]
    For $x, y \in X$, let $x \sim y := y \in Orb(x)$. 
    Show that $\sim$ is an equivalence relation on $X$ 
    with orbits as equivalence classes, 
    hence deducing $X/G$ is a partition $X$. 
}

\ex{Let $x \in X$. Show that $Orb(x) \leq_{G\mhyph\Set} X$.}

A subgroup $H$ of a group $G$ acts on $G$ as a set: 

%   - Def - Cosets as Orbits
\begin{dfn} Left Cosets. 

    Let $H \leq_\Grp G$. 
    Define 
    \[H \to Aut_\Set(G), h \mapsto (g \mapsto gh^{-1})\]
    Then this is an $H$-action on $G$. \footnote{
        The reader may be wondering why it is $gh^{-1}$, not $gh$.
        Note the following: for $a, b \in H$, 
        $(ab)(g) = g(ab)^{-1} = gb^{-1}a^{-1} = a(b(g)) = (a \circ b)(g)$.
        So we see that the inverse is needed to make our action a morphism of groups.
    }
    For $g \in G$, 
    the \emph{left coset of $H$ represented by $g$} is defined as
    \[gH := Orb(g)\]
    i.e. the orbit of $g$ under this $H$-action. 
    This is all the elements in $G$ reachable by 
    right-multiplication with $H$, hence the notation $gH$. 
    Furthermore, let $G / H$ be the set of all cosets, 
    then $G$ acts naturally on $G / H$ by
    \[G \to Aut_\Set(G / H), g \mapsto (g_0H \mapsto gg_0H)\]
\end{dfn}

The following concept is intimately related to that of orbits. 

%   - Def - Stabiliser Subgroup of an Element
\begin{dfn} Stabiliser Subgroup of an Element. 

    Let $\rho$ be a $G$-action on $X$, $x \in X$. 
    Then the \emph{stabiliser subgroup of $x$} is defined as
    \[
        Stab_\rho(x) := \{g \in G \mid g_\rho(x) = x\} \leq_\Grp G
    \]
    Unfold the definition of ``$\rho$ in $\Grp$" and verify that 
    $Stab_\rho(x)$ is indeed a subgroup. 
    We write $Stab(x)$ instead, if the action is clear. 

\end{dfn}
Now, the main theorem concerning group actions. 
%   - Thm - Orbit Stabiliser
\begin{thm} Orbit-Stabiliser. 
    
    Let $(X,\rho)$ be a $G$-set, $x \in X$.
    Then $G / Stab(x) \iso_{G-Set} Orb(x)$ via the isomorphism of $G$-sets, 
    \[
        G / Stab(x) \to Orb(x), gStab(x) \mapsto g_\rho(x)
    \]
\end{thm}
\begin{proof}
    Well-definedness and bijectivity are easily verified. 
    To show this is a morphism of $G$-sets, let $g \in G$, $g_0Stab(x) \in G / Stab(x)$. 
    Then by the definition of the $G$-actions on $G/Stab(x)$ and $Orb(x)$, 
    we have 
    \begin{figure}[ht]
        \centering
        \begin{tikzcd}
        g_0Stab(x) \arrow[r,"g",symbol=\longmapsto] \arrow[d,symbol=\longmapsto]
        & (gg_0)Stab(x) \arrow[d,symbol=\longmapsto]\\
        g_0(x) \arrow[r,"g",symbol=\longmapsto] 
        & g(g_0(x)) = (gg_0)(x)\\
        \end{tikzcd}
    \end{figure}
    
    %\kern-3em
    So this is indeed a isomorphism of $G$-sets. 
\end{proof}

From this we can immediately deduce the classic theorem due to Lagrange. 

%   - Cor - Lagrange
\begin{cor} Lagrange's Theorem. 

    Let $H \leq_{\Grp} G$ where $G$ is finite. 
    Then $|G| = |G / H||H|.$
    
    The \emph{index of $H$ in $G$} is defined as $[G : H] := |G / H|$. 
\end{cor}
\begin{proof}
    For $g \in G$, $Stab(g) = \{h \in H \mid gh^{-1} = g\} = \{e\}$.
    Then by the orbit-stabiliser theorem, $|gH| = |H/Stab(g)| = |H/{e}| = |H|$.
    Hence by $G / H$ partitioning $G$, 
    \[
        |G| = \sum_{gH \in G/H} |gH| = \sum_{gH \in G/H} |H| = |G/H| |H|
    \]
\end{proof}
%\begin{rmk}
%    Let $H \leq_{\Grp} G$.
%    We already have a multiplication of elements in $G/H$ by elements in $G$
%    via the $G$-action on $G/H$, $a\cdot bH := a(bH) = (ab)H$.
%    One may wonder if we can adapt this multiplication to multiply by \emph{cosets}, 
%    turning $G/H$ into a group. 
%    A plausible way of doing this is defining it as $aH \cdot bH := (ab)H$. 
%    Alas, this is \emph{not} always well-defined. 
%    Hence we have, for subgroups, the notion of \emph{normality}. 
%\end{rmk}
%   - Def - 4 eqv defs of Normal Subgroup
\begin{dfn} 4 Equivalent Definitions of Normal Subgroups, 
Quotient Groups. 
    Let $N \leq_{\Grp} G$. 
    Then the following are equivalent: 
    \begin{enumerate}
        \item $\forall a, b \in G, aNbN = 
        \{an_abn_b \mid n_a, n_b \in N\} = (ab)N.$
        \item $G / N$ forms a group with $aN \cdot bN := (ab)N$.
            Hence $G \to G/N, a \mapsto aN$ is a morphism of groups.
        \item $N$ is the kernel of some morphism of groups from $G$. 
        \item $\forall g \in G, gNg^{-1} = \{gng^{-1} \mid n \in N\} = N$. 
    \end{enumerate}
    If any of the above is true, 
    then we say \emph{$N$ is a normal subgroup of $G$}
    and write $N \normsub_\Grp G$. 
    $G/N$ is called the \emph{quotient of $G$ by $N$}. 
\end{dfn}

\begin{proof}
    ($1 \imp 2$) 
        We check that the multiplication is well-defined. 
        Let $an_a \in aN, bn_b \in bN$. 
        Then $an_abn_b \in aNbN = (ab)N$. 
        Hence $(an_abn_b)N = (ab)N$.
        The group axioms are easily verified
        and the map is clearly a morphism of groups.
        
    ($2 \imp 3$)
        It is clear that $N$ is the kernel of the morphism
        $G \to G/N$.
        
    ($3 \imp 4$)
        Let $g \in G$. 
        Let $\phi : G \to K$ in $\Grp$ with $N = \text{ker }\phi$.
        Then for $n \in N$, it is easily verified that $\phi(gng^{-1}) = e_K$. 
        So $\forall g \in G, gNg^{-1} \subseteq N$. 
        Hence for $g \in G$, $N = g(g^{-1}Ng)g^{-1} \subseteq gNg^{-1}$.
        
    ($4 \imp 1$)
        Let $a, b \in G$. 
        Essentially, $aNbN = abNb^{-1}bN = abNN = abN$. 
        The formalities are left as an exercise for the reader to check. 
        
\end{proof}

\begin{rmk}
    The morphism $G \to G/N$ is usually called the \emph{projection}. 
    One should think of this as taking $G$ partitioned by the cosets of $N$
    and collapsing each coset to a point. 
    The set of these points form the quotient. 
\end{rmk}

\begin{rmk}
    Note that all subgroups of abelian groups are normal. 
    The main significance of this is that 
    when we quotient here we always end up with a group structure.
\end{rmk}

Now we are ready for the main theorems regarding isomorphisms
that we will so often use. 

%   - Thm - 1st iso
\begin{thm} 1st Isomorphism (AKA Orbit-Stabiliser Applied to Groups). 
    
    Let $\phi : G \to H$ in $\Grp$. 
    Then $G / \text{ker } \phi \iso_\Grp \im \phi$ 
    via $g(\text{ker }\phi) \mapsto \phi(g)$.
    This is depicted by the following diagram:
    \begin{figure}[ht]
        \centering
        \begin{tikzcd}[column sep=huge,row sep=large]
        G \arrow{r}{\phi} \arrow{d}{\pi} & \im \phi \leq H\\
        G/\text{ker }\phi 
        \arrow{ur}[anchor=center,rotate=26,yshift=-2.5ex]
        {g\text{ker }\phi \mapsto \phi(g)}
        \end{tikzcd}
    \end{figure}
    We say the diagram \textbf{commutes}. 
\end{thm}
\begin{proof}
    Left as an easy exercise. 
    (Hint : Use the orbit-stabiliser theorem
    on a suitable element in $\im \phi$ with a suitable $G$-action on $\im \phi$.)
\end{proof}
%   - Thm - 3rd iso Part 1 and 2
\begin{thm} 3rd Isomorphism. 
    Let $N \normsub_\Grp G$ with $\pi : G \to G/N$ the usual morphism. Then
    \begin{enumerate}
        \item $\pi : 
        \{M \leq_\Grp G \mid N \subseteq M\} \to \{K \leq_\Grp G/N\}, 
        M \mapsto \pi M = M/N$ is an inclusion-preserving bijection.
        \footnote{This means for all $M_0\subseteq M_1,
        \pi M_0 \subseteq \pi M_1$}
        \item Let $M \leq_\Grp G$, $N \subseteq M$. 
        Then $M \normsub_\Grp G \iff \pi M \normsub_\Grp G/N$. 
        \item Let $M \normsub_\Grp G$, $N \subseteq M$. 
        Then $G / M \iso_\Grp (G / N) / (M / N)$. 
    \end{enumerate}
    
\end{thm}
\begin{proof}
    
    (1) Let $M \leq G$, $N \subseteq M$. 
    $N \normsub G$ implies $\forall g \in G, gNg^{-1} = N$. 
    In particular, this means $\forall m \in M, mNm^{-1} = N$. 
    By definition 4 of normality, $N \normsub M$, 
    i.e. $M/N$ is well-defined as a group. 
    $\pi M$ consists of the cosets of $N$ with elements of $M$ as representatives. 
    This is exactly $M/N$, so $\pi M = M/N$. 
    
    Let $M_0, M_1 \leq G$, $N \subseteq M_0 \subseteq M_1$.
    Then clearly $\pi M_0 \subseteq \pi M_1$. 
    So we have inclusion preservation. 
    
    We now show surjectivity. Let $H \leq G/N$. 
    We claim that $\pi^{-1} H$ is a subgroup of $G$ that maps to $H$. 
    Indeed $\pi(\pi^{-1} H) = H$, 
    and since $\{N\} = \{e_{G/N}\} \subseteq H$, 
    $N = \pi^{-1} \{N\} \subseteq \pi^{-1} H$.
    To show $\pi^{-1} H$ is a subgroup of $G$, 
    let $a, b \in \pi^{-1} H$, i.e. $\pi(a), \pi(b) \in H$. 
    Then $\pi(ab) = \pi(a)\pi(b) \in H$ by $H \leq G/N$, so $ab \in \pi^{-1} H$. 
    Also, $\pi(a^{-1}) = (\pi(a))^{-1} \in H$ again by $H \leq G/N$. 
    Clearly $e_G \in \pi^{-1} H$. Thus $\pi^{-1} H$ is a subgroup of $G$. 
    
    Injectivity is where we shall use the mysterious $N \subseteq M$ condition. 
    Let $M_0, M_1 \leq G$, $N \subseteq M_0$, $N \subseteq M_1$. 
    Assume $\pi M_0 = \pi M_1$. Then $M_0 / N = M_1 / N$. 
    We wish to show $M_0 = M_1$. 
    Since the argument is symmetrical, it suffices to prove $M_0 \subseteq M_1$. 
    So let $m_0 \in M_0$. Then $M_0 / N = M_1 / N$ implies
    $\exists m_1 \in M_1, m_0 N = m_1 N$. 
    Notice $m_0 \in m_0 N = m_1 N \subseteq M_1$.  
    Hence, $m_0 \in M_1$. Thus $M_0 \subseteq M_1$
    and the proof of 1 is done. 

    Parts 2 and 3 are left as an exercise. 
    (Hint for part 3 : Use the 1st isomorphism theorem.)
\end{proof}

Here is the diagram to have in mind:
\begin{figure}[ht]
    \centering
    \begin{tikzcd}
        G \arrow[r,"\pi"]  & G/N   \arrow[r]& (G/N)/(M/N)   \\
        M \arrow[r,"\pi"]   \arrow[u,symbol=\leq] 
        & M/N \arrow[u, swap,symbol=\leq] \arrow[r]
        & \{e_{(G/N)/(M/N)}\} \arrow[u,symbol=\leq]\\
        N \arrow[u, swap,symbol=\leq] \arrow[r,"\pi"] & \{e_{G/N}\} \arrow[u,symbol=\leq]
    \end{tikzcd}
\end{figure}
\begin{rmk}
    As we will see, 
    the idea of morphisms between objects, 
    sub-objects, quotient-objects and the isomorphism theorems
    will appear again and again with various objects in algebra. 
    We next cover the above theory for vector spaces. 
\end{rmk}

% Vector Spaces 
\section{Vector Spaces}
\begin{dfn} Fields (Preliminary Version).

    A \textbf{field} is a triplet $(K,+,\cdot)$ where
    $+, \cdot$ are binary operations such that
    $(K,+)$ is an abelian group with identity $0$,
    $(K^\times,\cdot)$ where $K^\times := K\backslash\{0\}$
    is an abelian group with identity $1 \neq 0$,
    and $\cdot$ \textbf{distributes} over $+$, i.e.
    \[\forall a, b, c \in K, a \cdot (b + c) = a \cdot b + a \cdot c.\]
    $+,\cdot$ are respectively called \textbf{addition} and \textbf{multiplication}, 
    $0, 1$ are respectively the \textbf{additive and multiplicative identity}. 
\end{dfn}

\begin{eg}
    $(\Q,+,\cdot), (\R,+,\cdot), (\C,+,\cdot)$ are all fields. 
    Verify that for all primes $p \in \Z$, $(\Z/p\Z,+,\cdot)$
    is also a field.
\end{eg}

\begin{dfn} Endomorphisms, Automorphisms of Groups.

    Let $(V,+)$ be a group. 
    The set of \textbf{endomorphisms} of $V$ is defined as
    \[ 
        End_\Grp(V) := \{\la : V \to V \text{ in } \Grp\} 
    \]
    Then $End_\Grp(V)$ naturally forms a group, under pointwise addition:
    \[
        End_\Grp(V) \times End_\Grp(V) \to End_\Grp(V), 
    \]
    \[
        (\la, \mu) \mapsto (\la + \mu : x \mapsto \la(x) + \mu(x))
    \]
    with the \emph{zero morphism} $0 : V \to V, 
    x \mapsto 0$ as the group identity\footnote{
    $0$ means different things in different places,
    with sufficient common sense this does not lead to confusion. 
    Here, $0$ is certainly not a ``number" in general 
    (whatever that means).}.
    If $V$ is abelian, then $End_\Grp(V)$ is also abelian. 
    
    The set of \textbf{automorphisms} of $V$, denoted $Aut_\Grp(V)$, 
    is the subset of isomorphisms in $End_\Grp(V)$. 
    Then $Aut_\Grp(V)$ is a group on its own
    as a subgroup of $Aut_\Set(V)$ under function composition. 
    
\end{dfn}

\begin{rmk}
    Note that multiplication for $Aut_\Grp(V)$ also works for $End_\Grp(V)$.
    However, since endomophisms are not required to be bijective, 
    $End_\Grp(V)$ is \emph{not} a group under this multiplication. 
    For example, the zero morphism has no multiplicative inverse. 
    This multiplication \emph{does} distribute over addition in $End_\Grp(V)$: let $f, g, h \in End_\Grp(V), v \in V$, then
    \begin{align*}
        &(f(g + h))(v) = f((g+h)(v)) = f(g(v) + h(v)) \\
        =\;& f(g(v)) + f(h(v)) = (fg)(v) + (fh)(v) = (fg + fh)(v)
    \end{align*}
    It looks like it is \emph{almost} a field. 
    We shall have more to say about structures like this 
    in the next section on \emph{rings}. 
\end{rmk}

%   - Def - Vector Space over a field K 
\begin{dfn} Vector Space over a Field. 

    Let $K$ be a field. 
    Then a \textbf{$K$-vector space} is a pair $(V,\rho)$
    where $V$ is an abelian group under $+$, 
    and $\rho : K \to End_\Grp(V)$ is a morphism of groups for both
    $(K,+) \to (End_\Grp(V),+)$ \emph{and} 
    $(K^\times,\cdot) \to Aut_\Grp(V)$.
    If $\rho$ is unambiguous, then we write $V$ instead of $(V,\rho)$. 
    
    Let $V$ be a $K$-vector space. 
    Then elements in $K$ are referred to as \textbf{scalars}
    and for $\la \in K, v \in V$, applying $\la$ to $v$ 
    (i.e. applying $\la_\rho$ to $v$) is called
    \textbf{scalar multiplication} and is denoted $\la v$. 
    Elements in $V$ are called \emph{vectors}. 

\end{dfn}

\begin{rmk}
    One should think of a $K$-vector space $V$ as $K$ acting on $V$
    in a similar way to a group acting on a set. 
    $V$ is in fact a $G$-set for $G=K^\times$
    \footnote{a $K^\times$-set}, 
    as $\rho : K^\times \to Aut_\Grp(V) \leq_\Grp Aut_\Set(V)$
    is a morphism of groups. 
    The motivation for $\rho$ mapping into $Aut_\Grp(V)$ is
    to have $K^\times$ preserve the "structure" of $V$ as a group. 
    
\end{rmk}

\ex{Unfold the definition of of $V$ being a $K$-vector space to obtain
    the usual axioms of a vector space used in other sources: 
    \begin{enumerate}
        \item Associativity of Vector Addition.
            $\forall a, b, c \in V, a + (b + c) = (a + b) + c$.
        \item Additive Identity of Vector Addition.
            $\exists 0 \in V, \forall a \in V, 0 + a = a + 0 = a$.
        \item Additive Inverse of Vector Addition.
            $\forall a \in V, \exists b \in V, a + b = b + a = 0$. 
        \item Commutativity of Vector Addition.
            $\forall a, b \in V, a + b = b + a$. 
        \item Compatibility of Scalar Multiplication.
            $\forall \la, \mu \in K, \forall a \in V, (\la\mu)(a) = \la(\mu(a))$
        \item Identity for Scalar Multiplication.
            $\forall a \in V, 1(a) = a$.
        \item Distributivity of Scalar Multiplication over Vector Addition.
            \[ 
                \forall \la \in K, \forall a, b \in V, \la(a + b) = \la(a) + \la(b)
            \]
        \item Distributivity of Vectors over Scalar Addition.
            \[ 
                \forall \la, \mu \in K, 
                \forall a \in V, (\la + \mu)(a) = \la(a) + \mu(a) 
            \]
    \end{enumerate}
}

\ex{Let $V$ be a $K$-vector space. Show that $\forall a \in V, 0(a) = 0$, 
    where the first 0 is in $K$ and the second is in $V$. 
}

%   - Def - Linear Map, Isomorphism, Kernel
\begin{dfn} Morphism of $K$-vector Spaces, Isomorphisms, Images, Kernels. 

    Let $V, W$ be $K$-vector spaces
    Then $V, W$ are both $K^\times$-sets by the previous remark. 
    Let $f : V \to W$ a function. 
    Then $f$ is a \textbf{morphism of $K$-vector spaces} (AKA linear map) when 
    $f$ is a morphism of groups \emph{and} a morphism of $K^\times$-sets, 
    or equivalently, for all $a, b \in V, \la \in K,$ 
    \[
        f(a + b) = f(a) + f(b) \text{ and } 
        f(\la(a)) = \la(f(a))
    \]
    Since morphisms of $K$-vector spaces preserve 
    the actions of $K$ on $K$-vector spaces, 
    they are really the only functions worth considering 
    in ``the world of $K$-vector spaces". 
    Thus, we write ``$f : V \to W$ in $\KVec$". 
    
    If in addition $f$ bijects, it is called an \textbf{isomorphism}, and 
    we say $V$ and $W$ are \textbf{isomorphic as $K$-vector spaces}.
    We write $V \iso_{\KVec} W$.
    
    We also have: 
    \begin{itemize}
        \item The \textbf{image} of $f$, 
            $\im f$ := $\{b \in W \mid \exists a \in V, f(a) = b\}$.
        \item The \textbf{kernel} of $f$, 
            $ \ker f$ := $\{ a \in V \mid f(a) = 0 \in W\}$.
    \end{itemize}
\end{dfn}
\ex{Let $U, V, W$ be $K$-vector spaces, $f : U \to V$, $g : V \to W$ both in $\KVec$.
    Show that $g \circ f : U \to W$ in $\KVec$. 
}

%   - ex - Homomorphism Inject iff Kernel Trivial
\ex{Let $f : V \to W$ in $\KVec$. 
    Prove that $f$ injects if and only if $\ker f = \{0\} \subseteq V$.
}

\ex{Let $f : V \to W$ in $\KVec$. 
    Show that \[\im f \leq_\Grp W \;and\;
    \im \; f \leq_{K^\times\mhyph\Set}W\]
    Hence deduce that $\im f$ is a $K$-vector space. 
    Similarly, show that $ \ker f$ forms a $K$-vector space
    from the group multiplication and $K^\times$-action on $V$. 
}
%   - Def - Subspaces
\begin{dfn} Subspaces. 

    Let $V$ be a $K$-vector space, $W \subseteq V$.
    $W$ is a \textbf{subspace} of $V$ when 
    $W \leq_\Grp V$ \emph{and} $W \leq_{K^\times\mhyph\Set} V$.
    We denote this with $W \leq_{\KVec} V$. 
    If it is clear that $W, V$ are $K$-vector spaces,
    we omit the subscript $\KVec$. 
    
\end{dfn}
%   - Ex - Images, Kernels are Subspaces
\begin{eg}
    Images and kernels are subspaces.
\end{eg}
%   - Ex - Arbitrary intersection of subspaces is subspace
\ex{Let $(V_i)_{i \in I}$ be subspaces of $V$, a $K$-vector space. 
    Prove: 
    \[\bigcap_{i \in I} V_i \leq_{\KVec} V\] 
    Hence deduce that 
    \[
        \forall S \subseteq V, \exists ! \< S\> \leq_{\KVec} V, 
        S \subseteq \< S \> \land 
        \forall W \leq_{\KVec} V, S \subseteq W \imp \< S \> \subseteq W. 
    \]
    i.e. $\< S \>$ is the \emph{smallest} subspace containing $S$. 
    $\< S \>$ is called the \textbf{subspace generated by $S$}. 
    Futhermore, show that 
    $\< S \> = \{\sum_{i = 1}^n \la_i s_i \mid n \in \N, \la_i \in K, s_i \in S\}$
    These finite sums are called \textbf{linear combinations of elements in $S$}.
    If $S = \{x\}$ is a singleton set, 
    we write $\< x \>$ for $\< \{x\} \>$. 
}

%   - Def - Quotient Space
\begin{dfn} Quotient Space.

    Let $V$ be a $K$-vector space, $W \leq V$ a subspace.
    So $W \leq_\Grp V$. 
    Since $V$ abelian, $W \normsub_\Grp V$ and we have a group structure on $V / W$.
    \[ 
        (a + W) + (b + W) = (a + b) + W
    \]
    The question is, do we get a $K$-vector space structure?
    Well, note that for $a \in V, \la \in K$, 
    by $W \leq_{K^\times\mhyph\Set} V$,
    \[\la (a + W) := \{\la(a + w) \mid w \in W\} \subseteq (\la a) + W\]
    Hence if we define $\la(a + W) := (\la a) + W$, 
    this is well-defined as a $K^\times$-action,
    making $V/W$ a $K$-vector space.
    Furthermore, this makes the projection $V \to V/W$ 
    a morphism of $K$-vector spaces. 
    This $K$-vector space, $V/W$ is called the \textbf{quotient space of V by W}.
\end{dfn}
%   - Thm - 1st iso
\begin{thm} 1st Isomorphism. 
    
    Let $f : V \to W$ in $\KVec$. 
    Then $V / \ker f \iso_{\KVec} \im f$ via $a + \ker f \mapsto f(a)$.
\end{thm}
\begin{proof}
    By the 1st isomorphism theorem for groups, 
    $a + \ker f \mapsto f(a)$ is an isomorphism of groups. 
    Let $\la \in K$ be a scalar. 
    Then for $a + \ker f \in V/\ker f$, 
    \[
        \la(a + \ker f) = (\la a) + \ker f \mapsto 
        f(\la a) = \la f(a)
    \]
    So the map is also an isomorphism of $K^\times$-sets.
    Hence the map is an isomorphism of $K$-vector spaces. 
\end{proof}
%   - Thm - 3rd iso
\begin{thm} 3rd Isomorphism. 

    Let $W \leq_{\KVec} V$, $\pi : V \to V/W$ be the usual projection. 
    Then 
    \begin{enumerate}
        \item $\pi : 
        \{M \leq_{\KVec} V \mid W \subseteq M\} \to \{N \leq_{\KVec} V/W\}, 
        M \mapsto \pi M = M/W$ is an inclusion-preserving bijection.
        \item Let $M \leq_{\KVec} V$, $W \subseteq M$. 
        Then $V / M \iso_\Grp (V / W) / (M / W)$. 
    \end{enumerate}
\end{thm}
\begin{proof}
    Analogous to the proof for groups
    and thus is left as an exercise to the reader. 
\end{proof}

\begin{rmk}
    Note that for $K$-vector spaces, there is no need 
    for the notion of "normal" subspaces. 
    That has been taken care of by the underlying groups being abelian.
\end{rmk}

% Idea
%   - Def - Direct Sum of Vector Spaces
%   - Eg - Vector Space Generated from a Set (Free Vector Space)
%   - Def - Lin Indep, Spanning, Basis via Inject, Surject, Biject from Free Vector space

% Rings
\section{Rings}
%   - Def - Ring (with Unity), Commutative Ring
\begin{dfn} Rings (with Unity), Commutative Rings
    
    A \textbf{ring} is a triplet $(R,+,\cdot)$ where
    $(R,+)$ is an abelian group and 
    $\cdot : R \times R \to R$ is a multiplication such that:
    \begin{enumerate}
        \item Associative. 
        $\forall a, b, c \in R, a \cdot (b \cdot c) = (a \cdot b) \cdot c$
        \item Identity/Unity. 
        $\exists 1 \in R, \forall a \in R, 1 \cdot a = a \cdot 1 = a$
        \item Left Distributes over Addition. 
        $\forall a, b, c \in R, a\cdot(b + c) = a\cdot b + a\cdot c$
        \item Right Distributes over Addtion.
        $\forall a, b, c \in R, (b + c)\cdot a = b\cdot a + c\cdot a$
    \end{enumerate}
    If it is unambiguous, we write $ab$ for $a\cdot b$ and
    $R$ instead of $(R,+,\cdot)$. 
    We write $1_R$ for $1 \in R$ if it is unclear. 
    If the multiplication is commutative, we call $R$ a \textbf{commutative ring}.
    
    Some people prefer to define rings \emph{without} the axiom of identity/unity,
    hence they will call our definition of rings "rings with unity" instead. 
    We will stay with the convention of rings as rings with multiplicative identity.
    
\end{dfn}

\begin{ex}
    Verify that $(\Z,+,\cdot)$ and any fields are all commutative rings. 
\end{ex}

\begin{ex} Zero Ring.
    
    Let $R$ be a ring where $0 = 1$. Prove that $R = {0}$.
    This ring is called the \textbf{zero ring} and is denoted $\textbf{0}$.
\end{ex}

\begin{ex} Endomorphism Ring of an Abelian Group. 

    Let $(V,+)$ be an abelian group. 
    Recall that $End_\Grp(V)$ is 
    the set of all morphisms of groups from $V$ to itself.
    Verify that $(End_\Grp(V),+,\cdot)$ forms a ring where
    $+$ is pointwise addition and $\cdot$ is function composition. 
    
\end{ex}

%   - Def - Ring homomorphism, Isomorphism, Kernel
\begin{dfn} Morphism of Rings, Isomorphisms, Images and Kernels. 

    Let $(R,+,\cdot), (S,+,\times)$ be rings and $f : R \to S$ be a function. 
    Then $f$ is called a \textbf{morphism of rings} if 
    it is a morphism of groups from $(R,+)$ to $(S,+)$ and
    it respects multiplication, i.e. for $a, b \in R$, 
    \[f(a \cdot b) = f(a) \times f(b)\]
    and respects the multiplicative identities, 
    \[f(1_R) = 1_S\]
    Since these are the functions that preserve the ring structure, 
    they are the only maps to consider in the "world of rings".
    So we denote "$f : R \to S$ in $\Ring$" for $f$ being a morphism of rings. 
    If $f$ also bijects, it is called an \textbf{isomorphism}, and 
    we say $R, S$ are \textbf{isomorphic as rings}
    and write $R \iso_\Ring S$. 
    We also have the following: 
    \begin{enumerate}
        \item The \textbf{image} of $f$ is the set-theoretic image. 
        \item The \textbf{kernel} of $f$, 
        $\ker f := \{a \in R \mid f(a) = 0 \in S\}$
    \end{enumerate}
\end{dfn}

\begin{ex} [Important] Refined Definition of a Vector Space.

    Let $(V,\rho)$ be a $K$-vector space. 
    Verify that this is equivalent to $V$ being an abelian group
    with $\rho : K \to End_\Grp(V)$ as a morphism of rings. 
\end{ex}

\begin{ex} [Important] Modules. 

    Modules generalise vector spaces to over rings.
    That is to say: let $R$ be a ring. 
    Then an \textbf{$R$-module} is a pair $(M,\rho)$ where
    $M$ is an abelian group and $\rho : R \to End_\Grp(M)$ is a morphism of rings. 
    Prove all of the theorems we have developed for vector spaces over a field, 
    now for modules over a ring. 
    $\RMod$ denotes the "world of $R$-modules". 
\end{ex}

\begin{ex} [Important] Rings are Modules over themselves.

    Let $R$ be a ring. 
    Show that $R$ is a module over itself via left-multiplication as an action,
    i.e. \[
        \rho : R \to End_\Grp((R,+)), a \mapsto (x \mapsto ax)
    \]
    is a morphism of rings. 
    If $R$ is a commutative ring, 
    then right-multiplication also gives an action and
    it coincides with the action given by left-multiplication. 
\end{ex}

\begin{ex} [Skippable] Construction of Free Modules. 

    Let $R$ be a ring and $S$ be a set. 
    We proceed to turn $S$ into an $R$-module in the same way as we did free groups.
    
    For a function $\la : S \to R$, define its \textbf{support} to be \[
        \supp \la := \{s \in S \mid \la(s) \neq 0 \in R\}
    \]
    Let $\< S \>$ be the set of functions from $S$ to $R$ with \emph{finite} support.
    Then this is naturally an $R$-module via pointwise addition and
    pointwise left-multiplication by elements in $R$. 
    We have a natural injection $S \to \< S \>$ by
    interpreting $s \in S$ as the function \[
        s : S \to R, x \to \begin{cases}
            1   &, x = s \\
            0   &, x \neq s
        \end{cases}
    \]
    Furthermore, for arbitrary $\la \in \< S \>$, \[
        \la = \sum_{s \in \supp \la} \la(s) s
    \]
    So all elements in $\< S \>$ are linear combinations of ``elements" in $S$. 
    $\< S\>$ is called the \textbf{free $R$-module over $S$}.
    A \emph{free $R$-module} is one that is isomorphic 
    to a free $R$-module over some set. 
    If it is clear what the scalars are, we simply say "free module".
    If $R$ is a field, we say "free vector space" instead. 
    
    In particular, $R^n$ is defined as the free module over a set of $n$ elements.
\end{ex}

\begin{ex} [Skippable] "Characterising" Property of Free Module over a Set. 

    Let $R$ be a ring, $S$ a set, $\iota : S \to \<S\>$ the usual injection.
    Show that $\<S\>$ is the \emph{smallest} $R$-module containing $S$
    in the sense that for any $R$-module $M$ and function $f : S \to M$, 
    there exists a unique morphism of $R$-modules
    $\<f\> : \<S\> \to M$ such that $\<f\> \circ \iota = f$. 
    Diagrammatically, 
    \begin{figure} [ht]
        \centering
        \begin{tikzcd}
        & \<S\> \arrow[dd,"\exists ! \< f \>",dashed]\\
        S \arrow{ru}{\iota} \arrow{rd}{f} & \\
        & M \\
        \end{tikzcd}
    \end{figure}
    
    Hence show that $\<S\>$ is unique up to ismorphism, i.e.
    for any $R$-module with an injection from $S$ into it, 
    it is ismorphic to $\<S\>$. 
\end{ex}

\begin{ex} [Important] Images are Rings. 

    Let $f : R \to S$ be a morphism of rings. 
    Show that $\im f$ forms a ring under the addition and multiplication from $S$. 
\end{ex}

\begin{dfn} Subrings. 

    Let $R$ be a ring, $S \subseteq R$. 
    We leave it as an exercise to figure out the two equivalent definitions
    of \textbf{subrings}, analogous to 
    the definitions of subgroups, sub-$G$-sets, subspaces. 
    
    Images of morphisms are subrings. 
\end{dfn}

\begin{rmk}
    For Galois theory, we will really only need commutative rings. 
    Thus from now on, by rings, we will mean \emph{commutative rings}. 
\end{rmk}

\begin{ex} [Important] Kernels are Submodules.

    Let $f : R \to S$ be a morphism of rings. 
    Show that $\ker f \leq_{\RMod} R$, i.e. $\ker f$ is a submodule of $R$.
    Hence, $f$ injects if and only if $\ker f = \{0\} \subseteq R$.  
\end{ex}

\begin{rmk} 
    Something about how rings are quite badly-behaved.
    Namely, sadly subrings are not what you quotient by
    because it's quotienting by submodules that gives ring structure. 
\end{rmk}

%   - Def - Ideals
\begin{dfn} Ideals and Quotient Rings. 

    Let $R$ be a ring, $S \subseteq R$. 
    Then $S$ is an \textbf{ideal of $R$} when it is a submodule of $R$. 
    This is denoted $S \normsub_\Ring R$. 
    If it is clear that we are "in the world of rings", 
    we write omit the subscript $\Ring$. 
    
    Let $S \normsub_\Ring R$. Then $R/S$ is an $R$-module. 
    Since $S$ is a submodule of $R$ under both left-multiplication \emph{and}
    right-multiplication actions, 
    $R/S$ has \emph{two} scalar multiplications by elements in $R$,
    one from the left and one from the right. 
    That is, for $a, x \in R$, \[
        a (x + S) = (ax) + S = (xa) + S = (x + S) a
    \]
    Furthermore, it respects the cosets of $S$, i.e. for $a + s, s \in S$, 
    \begin{align*}
        (a + s)(x + S) &= a (x + S) + s (x + S) \\
        &= ((ax) + S) + ((sx) + S) \\
        &= (ax) + S = a (x + S)
    \end{align*}
    This gives a well-defined multiplication on $R/S$, \[
        (R/S)\times(R/S) \to R/S, (a+S,b+S) \mapsto ab+S
    \]
    We leave the reader to check that $R/S$ forms a ring (commutative, of course)
    under this multiplication with the already existing addition. 
    $R/S$ is then called the \textbf{quotient ring of $R$ by $S$}. 
    Note that we also get the projection $R \to R/S$ as a morphism of rings. 
\end{dfn}

%   - Ex - Arbitrary intersection of ideals is ideal
\begin{ex} Ideals Generated by a Subset. 
    Let $S \subseteq R$ where $R$ is a ring. 
    Deduce the existence and uniqueness of a \emph{smallest} ideal containing $S$.
    This is called the \textbf{ideal generated by $S$} and 
    is denoted $(S)$. 
    
    Verify that $(S) = 
    \{ \sum_{i \in n} r_i s_i \mid n \in \N, r_i \in R, s_i \in S\}$. 
    If $S = \{x\}$ is a singleton set, we write $(x)$ instead of $(\{x\})$. 
\end{ex}

%   - Thm - 1st iso
\begin{thm} 1st Isomorphism.
    
    Let $f : R \to S$ be a morphism of rings. 
    We remind the reader that we are assuming $R, S$ to be commutative. 
    Then $R/\ker f \iso_\Ring \im f$ via $a + \ker f \mapsto f(a)$. 
\end{thm}
\begin{proof}
    By the 1st isomorphism theorem for modules, 
    with a suitable $R \to End_\Grp((\im f, +))$,
    we have $R/\ker f$ and $\im f$ isomorphic as $R$-modules. 
    We leave it as an exercise for the reader to check that
    this respects the multiplication on $R/\ker f$ and
    hence is an isomorphism of rings.
\end{proof}
%   - Thm - 3rd iso
\begin{thm} 3rd Isomorphism.

    Let $R$ be a ring, $S \normsub R$ an ideal, and
    $\pi : R \to R/S$ the usual projection. 
    Then
    \begin{enumerate}
        \item $\pi : 
        \{M \leq_\Ring R \mid S \subseteq M\} \to \{N \leq_\Ring R/S\}, 
        M \mapsto \pi M = M/S$ is an inclusion-preserving bijection.
        \item Let $M \leq_\Ring R$, $S \subseteq M$. 
        Then $M \normsub_\Ring R \iff \pi M \normsub_\Ring R/S$.
        \item Let $M \normsub_\Ring S$, $S \subseteq M$. 
        Then $R / M \iso_\Ring (R / S) / (M / S)$. 
    \end{enumerate}
\end{thm}
\begin{proof}
    Note that the inclusion-preserving bijection is between \emph{subrings},
    not just ideals. 
    This means there is a bit more work to do, 
    but the proof is overall still analogous to the same theorem for groups.
    So we still leave this as an exercise to the reader. 
\end{proof}

\begin{rmk}
    This concludes our \emph{very} brief introduction to 
    groups, vector spaces and rings as our basic algebraic objects. 
    
    We now cover some elementary results on \emph{polynomial rings} over a field.
\end{rmk}

%- Polynomial Ring over Fields
\section{Polynomial Rings over Fields}
%    - Def - Units of a Comm Ring, Fields (3 eqv def)
\begin{dfn} Fields (Upgraded Definitions).

    Let $K$ be a non-trivial commutative ring. 
    For $x \in K$, 
    \[
        x \in K^\times := \exists y \in K, xy = 1. 
    \]
    $x^{-1} := y$ is unique for fixed x. 
    Then $(K^\times, \cdot)$ is an abelian group. 
    Elements of $K^\times$ are called \textbf{units}. 

    $K$ is a \emph{field} if any of the following are true:
    \begin{enumerate}
        \item $\forall x \in K, x = 0 \lor x \in K^\times$
        \item $\forall \fa \normsub_\Ring K, \fa = (0) \lor \fa = K$
        \item $\forall f : K \to R$ in $\Ring$, $f$ injective $\lor f = 0$
    \end{enumerate}
\end{dfn}
\begin{proof} 
    $(1 \Rightarrow  2)$
    Suppose $\fa \neq (0) \normsub K$ is an ideal, $a \in K$. 
    Then $\exists x \in \fa, x \neq 0$.
    So $1 = x^{-1} x \in \fa$. 
    Hence $a = a\cdot1 \in \fa$. 

    $(2 \Rightarrow 3)$ 
    Let $f : K \to R$ be a morphism of rings. 
    Then $\ker f$ is an ideal of $K$. 
    So if $R$ non-zero, $f(1) = 1 \neq 0$ i.e. $\ker f \neq K$. 
    Thus, $\ker f = (0)$ which implies $f$ injective.  
    If $R$ is the zero ring, then $f = 0$. 

    $(3 \Rightarrow 1)$ 
    Suppose $x \in K$. 
    Then we have the natural morphism of rings $\pi : K \to K/(x)$. 
    If $\pi = 0$, then $(x) = K$ i.e. $\exists\,y \in K, 1 = yx$,
    so $x \in K^\times$. 
    Otherwise, $\pi$ injective implies $(x) = (0)$, hence x = 0. 
\end{proof}

\begin{ex} [Important] Fields are Integral Domains.
    
    Let $K$ be a field. 
    Show that for arbitrary $a, b \in K$, $ab = 0$ implies $a = 0$ or $b = 0$.
    
    In general, rings with the above property are called \textbf{integral domains}.
\end{ex}

%    - Def - K[x], deg on non-zero poly
\begin{dfn} Polynomial Ring over a Field, Degree of Non-Zero Polynomials. 

    Let $K$ be a field. 
    Let $K[X] := \<\N\>$ be the free $K$-vector space over $\N$. 
    For $n \in N$, let $X^n$ denote the image of $n$ 
    under the natural injection $\N \to \<\N\>$. 
    For $0 \in \N$, $1 := X^0$. 
    Then elements of $K[X]$ look generically like \[
        f = \sum_{k \in \N} f_k X^k
    \]
    where there exists $N \in \N$ such that for all $k$ after $N$, $f_k = 0$,
    since only finitely many $f_k$ are non-zero. 
    For $X^n, X^m \in K[X]$, we define multiplication via the addition of $\N$,
    i.e. \[X^n \cdot X^m := X^{n + m}\]
    This extends naturally to all of $K[X]$, \[
    fg = \left(\sum_{k \in \N} f_k X^k\right)\left(\sum_{l \in \N} g_l X^l\right) 
    = \sum_{n \in \N} \left(\sum_{k + l = n} f_k g_l\right) X^n
    \]
    giving $K[X]$ a ring structure.
    
    $K[X]$ is called the \textbf{polynomial ring over $K$}.
    Elements of $K[X]$ are called \textbf{polynomials over $K$}. 
    Note that we have a natural injection $K \to K[X]$ via $k \mapsto k\cdot 1$,
    which we shall denote with just $k$. 
    
    Note that the above construction also works for rings, 
    giving polynomial rings over rings.
    But for Galois theory, we shall focus on those over fields. 
    
    For a \emph{non-zero} polynomial $f = \sum_{k \in \N} f_k X^k\in K[X]$, 
    the \textbf{degree of $f$} is defined as 
    the maximum $k$ such that $f_k \neq 0$ and is denoted $\deg f$. 
    This is well-defined since $f$ is a finite sum of non-zero terms. 
    If $n$ is the degree of $f$, then $f_n$ is called 
    \textbf{leading coefficient of $f$}. 
    In particular, polynomials with leading coefficient 1 are called \textbf{monic}.
\end{dfn}

\begin{ex} [Important] Polynomial Rings are Integral Domains. 
    
    Let $K$ be a field. 
    Show that for non-zero $f, g \in K[X]$, $fg \neq 0$,
    i.e. $K[X]$ is an integral domain. 
\end{ex}

\begin{ex} Units in Polynomials Rings are Constants.
    Let $f \in K[X]^\times$. Show that there exists $k \in K$, $f = k$. 
    Hence deduce for all polynomials $f$, $\deg f = 0$ if and only if $f$ is a unit.
\end{ex}

%    - Lem - deg fg= deg f + deg g
\begin{lem} Degree Turns Multiplication into Addition. 
    
    Let $K$ be a field and $f, g$ non-zero polynomials in $K[X]$. 
    Then \[\deg fg = \deg f + \deg g\]
\end{lem}
\begin{proof}
    The claim is well-defined since $K[X]$ is an integral domain.
    We proceed by double strong induction on the degrees of $f$ and $g$. 
    Suppose the theorem is true for $\deg f < n$ and $\deg g < m$. 
    Let $\deg f = n$, $\deg g = m$. 
    Then $f = \phi + f_n X^n$ where $\deg \phi < n$ and 
    $g = \psi + g_m X^m$ where $\deg \psi < m$. 
    So \[
    fg = \phi \psi + \phi g_m X^m + \psi f_n X^n + f_n g_m X^{n + m}
    \]
    Clearly, when one multiplies by a single term as $g_m X^m, f_n X^n$, 
    $\deg \phi g_m X^m = \deg \phi + m$ and $\deg \psi f_n X^n = \deg \psi + n$.
    By assumption, $\deg \phi\psi = \deg \phi + \deg \psi$.
    Then the degree of the first three terms are all strictly less than $n + m$.
    The last term is non-zero since $K$ being a field and $f_n, g_m \neq 0$
    implies $f_n g_m \neq 0$. 
    So the last term has highest degree and hence
    $\deg fg = \deg f_n g_m X^{n+m} = n + m$. 
\end{proof}
%    - Ex  - deg 0 := -\infty, Why? 
\begin{ex}
    There are sources that define $\deg 0 := -\infty$ where
    $0$ is the zero polynomial in $K[X]$.
    Justify this convention in terms of the result of the previous lemma.
\end{ex}
%    - Lem - Div Algorithm
\begin{lem} Division Algorithm for Polynomials.
    
    Let $K$ be a field and $f, g$ polynomials $\in K[X]$, $g$ non-zero.
    Then there exists unique polynomials $q, r \in K[X]$ such that \[
        f = q g + r \land \deg r < \deg g
    \]
\end{lem}
\begin{proof}
    We proceed by strong induction on the degree of $f$. 
    Assume the theorem is true for $deg f < n$.
    Let $f = \sum_{k = 0}^{n} f_k X^k$ where $n$ is the degree of $f$
    and $g = \sum_{l = 0}^m g_l X^l$ where $m$ is the degree of $g$. 
    If $n < m$, we are done by choosing $q = 0$ and $r = f$.
    So suppose $m \leq n$. 
    Then \[
        \deg (f - (f_n/g_m)X^{n-m} g) < n
    \]
    So by assumption, there exists unique $q_0, r_0$ polynomials such that 
    \[f - (f_n/g_m)X^{n-m} g = q_0 g + r_0\] and $\deg r_0 < \deg g$. 
    Choosing $q = q_0 + (f_n/g_m)X^{n-m}$ and $r = r_0$ proves existence.
    Let $f = q_1 g + r_1$ where $\deg r_1 < \deg g$. 
    Then $f - (f_n/g_m)X^{n-m}g = (q_1 - (f_n/g_m)X^{n-m}) g + r_1$.
    Hence, $q_1 = q_0 + (f_n/g_m)X^{n-m} = q$ 
    and $r_1 = r_0 = r$ by uniqueness of $q_0, r_0$,
    which completes the strong induction. 
\end{proof}

%    - Cor - Ideals in K[x] are generated by 1 element. (K[x] PID)
\begin{ex} Polynomial Rings over Fields are Principal Ideal Domains. 
    
    Let $K$ be a field and $(0) \neq \fa \normsub K[X]$ be a non-zero ideal 
    of the polynomial ring. 
    Show that there exists a unique monic polynomial that generates $\fa$
    and it is of minimal degree in $\fa$. 
    (Hint : Use the division algorithm to derive a contradiction.)
    
    In general, rings where all ideals are generated by one element
    are called \textbf{principal ideal domains} or \textbf{PID} for short.
\end{ex}
%    - Def - Divides
\begin{dfn} Divides.
    
    Let $K$ be a field and $f, g \in K[X]$.
    Then \textbf{$g$ divides $f$} is defined as there existing a polynomial $h$
    such that $f = hg$. 
    Equivalently, $(f) \subseteq (g)$,
    the ideal generated by $g$ contains the ideal generated $f$.
    This is denoted $g \mid f$. 
\end{dfn}

\begin{ex} Degree respects Divides.
    Let $g \mid f$, both non-zero polynomials. Show that $\deg g \leq \deg f$. 
\end{ex}
%    - Ex  - Div Lin Comb
%    - Def - GCD (Monic)
\begin{dfn} Greatest Common Divisor (GCD for short). 

    Let $K$ be a field, $f, g, h$ polynomials over $K$. 
    $h$ is called the \textbf{gcd of $f$ and $g$} when all of the following are true:
    \begin{enumerate}
        \item $h \mid f$ and $h \mid g$
        \item For all other polynomials $p$, 
        $p \mid f$ and $p \mid g$ implies $p \mid h$.
    \end{enumerate}
    Note that $0$ being gcd implies both $f, g = 0$,
    so this is sort of a degenerate case. 
    Otherwise if $h \neq 0$, then WLOG $h$ is monic.
\end{dfn}
%    - Thm - Exists unique GCD via Eucl Alg 
\begin{thm} Existence and Uniqueness of GCD via Euclidean Algorithm. 

    Let $K$ be a field and $f, g$ polynomials over $K$. 
    Then there exists a unique polynomial $(f,g)$ that is the gcd of $f, g$
    up to scaling by units of $K$ 
    and there exists polynomials $\al, \be$ such that $\al f + \be g = (f, g)$.
    
    In general, we pick $(f,g)$ to be zero or monic.
\end{thm}
\begin{proof}
    We first prove uniqueness. 
    Let $h_1, h_2$ both be gcds of $f, g$. 
    Then by definition, $h_1 \mid h_2$ and $h_2 \mid h_1$, i.e.
    there exists $\al, \be$ polynomials such that 
    $h_1 = \al h_2$ and $h_2=\be h_1$.
    If one of $h_1, h_2$ are zero, then both are zero and hence equal.
    Now suppose $h_1 \neq 0$ and $h_2 \neq 0$. 
    Then $h_1 = \al h_2 = \al \be h_1$ which implies $1 = \al \be$ 
    by $h_1 \neq 0$ and $K[X]$ being an integral domain. 
    Hence $\al, \be \in K^\times$, i.e. $h_1, h_2$ differ by scaling by units.
    
    We now prove existence via the euclidean algorithm.
    Note that if either $f, g$ are zero, then the other is the gcd. 
    I.e. WLOG $g = 0$, then $f$ is the gcd of $f, g$, 
    basically because all polynomials divide $0 = g$. 
    So suppose both $f, g$ are non-zero. WLOG $\deg g \leq \deg f$.
    For $n \in \N$, define \begin{align*}
        r_0 &:= g \\
        f &= q_0 r_0 + r_1 \\
        r_n &= q_n r_{n+1} + r_{n+2}
    \end{align*}
    where the second and third line are as in the division algorithm. 
    Since $\deg r_n$ is a strictly monotonically decreasing sequence of naturals,
    it is eventually zero, i.e. there exists $N$ such that $\deg r_N = 0$.
    WLOG let $N$ be the minimal natural such that $\deg r_N = 0$. 
    Then it is easily verified that $r_{N+1} = 0$ and the algorithm ends.
    We leave it as an exercise to the reader to check that
    $r_N$ is the gcd of $f, g$ and 
    the polynomials $\al, \be$ such that $\al f + \be g = r_N$ 
    can be found via back-substituting the equations from the algorithm. 
\end{proof}
%    - Ex - Deduce exists/unique GCD from K[x] PID
\begin{ex} Alternative Proof of Existence and Uniqueness of GCD via PID.
    Let $f, g$ be polynomials over a field $K$.
    Show that for a polynomial $h \in K[X]$, 
    being the gcd of $f, g$ is equivalent to 
    $h$ generating the ideal generated by $\{f, g\}$, i.e. $(h) = (f, g)$.
    Hence show the existence and uniqueness of the gcd from $K[X]$ being a PID.
\end{ex}
%    - Def - Irr, Coprime, Prime
\begin{dfn} Irreducible Elements, Coprime Elements, Prime Elements.
    
    Let $f, g$ be a polynomials over a field $K$. 
    Then \begin{itemize}
        \item $f$ is called \textbf{irreducible} when it is non-zero, non-unit, and
            for all $a, b \in K[X]$, $f = a b$ implies $a$ unit or $b$ unit.
        \item $f, g$ are called \textbf{coprime} when their gcd $(f,g) = 1$.
            Equivalently, the ideal generated by $\{f,g\}$ is the entire $K[X]$.
        \item $f$ is called \textbf{prime} when any of the following 
            equivalent conditions are true: 
            \begin{enumerate}
                \item $f$ is non-zero, non-unit, and for all $a, b \in K[X]$, 
                $f \mid ab$ implies $f \mid a$ or $f \mid b$.
                \item The quotient ring $K[X]/(f)$ is an integral domain.
            \end{enumerate}
    \end{itemize}
    We leave as an exercise for the reader to prove 
    the equivalent definitions of prime. 
\end{dfn}
%    - Lem - Irr Div or Coprime
\begin{lem} Irreducible Elements either Divide or Coprime.
    
    Let $r$ be an irreducible polynomial over a field $K$.
    Then for all polynomials $a$, either $r \mid a$ or $(r,a) = 1$.
\end{lem}
\begin{proof}
    By definition of the gcd, there exists $b \in K[X]$ such that $r = b(r,a)$.
    $r$ irreducible implies $b$ unit or $(r,a)$ unit. 
    If $(r,a)$ is a unit, then WLOG it is 1 and we are done.
    Now suppose $b$ unit. Then by definition of gcd again, 
    there exists $c \in K[X]$ such that $a = c(r,a) = cb^{-1}r$, 
    i.e. $r \mid a$. 
\end{proof}
%    - Thm - Prime iff Irr
\begin{thm} For Polynomial Rings, Prime if and only if Irreducible.
    
    Let $f$ be a polynomial over a field $K$. 
    Then $f$ prime if and only if $f$ irreducible.
\end{thm}
\begin{proof}
    (Prime $\imp$ Irreducible)
    The following works in general for integral domains. 
    Suppose $f$ is prime. Let $f = ab$ where $a, b \in K[X]$. 
    Then clearly $f \mid ab$. 
    So by primeness, $f \mid a$ or $f \mid b$. 
    Since the argument is symmetrical, WLOG $f \mid a$. 
    Then there exists a polynomial $c$ such that $a = cf$,
    so $f = ab = cbf$. 
    $f$ is non-zero since it is prime, so by $K[X]$ being an integral domain,
    $1 = cb$, i.e. $b$ is a unit. 
    
    (Irreducible $\imp$ Prime)
    Suppose $f$ is irreducible. Let $f \mid ab$ where $a, b$ are polynomials. 
    By irreducibility, $f \mid a$ or $(f, a) = 1$.
    If $f \mid a$, we are done. 
    So suppose $(f, a) = 1$. 
    Then there exists polynomials $\al, \phi$ such that $\al a + \phi f = 1$.
    This gives $b = \al a b + \phi b f$ which $f$ clearly divides, and we are done.
\end{proof}
%    - Thm - K[x] unique factorise
\begin{thm} Fundamental Theorem of Arithmetic for Polynomial Rings.
    
    Let $f$ be a non-zero, non-unit polynomial over a field $K$.
    Then there exists irreducible polynomials $r_1, \dots, r_n$ such that
    $f = r_1\cdots r_n$, and they are unique up to reordering and scaling by units.
    
    In general, rings where the "fundamental theorem of arithmetic" hold
    are called \textbf{unique factorisation domains} or \textbf{UFD} for short.
\end{thm}
\begin{proof}
    We first prove uniqueness, which is a standard proof.
    Let $s_1 \cdots s_m = f = r_1 \cdots r_n$ where $s_j$ are irreducibles.
    Then clearly $r_1 \mid s_1 \cdots s_m$. 
    $r_1$ irreducible implies $r_1$ prime, which implies with induction that
    there exists $1 \leq j \leq m$, $r_1 \mid s_j$. 
    By reordering, WLOG $r_1 \mid s_1$. 
    Then by irreducibility of $s_1$ and $r_1$ non-unit, 
    $s_1 = \la_1 r_1$ where $\la_1$ is a unit in $K[X]$. 
    Then $r_1 \cdots r_n = \la_1 r_1 s_2 \cdots s_m$. 
    Since $K[X]$ is an integral domain and $r_1$ is non-zero, 
    we have $r_2 \cdots r_n = \la_1 s_2 \cdots s_m$. 
    Suppose $n \neq m$, WLOG $n < m$. 
    Then by induction, $1 = \la_1 \cdots \la_n s_{n+1} \cdots s_m$,
    which implies $s_m$ is a unit, contradicting the irreducibility of $s_m$.
    Hence $n = m$. Then by induction again, for all $1\leq i \leq n$,
    $s_i = \la_i r_i$ where $\la_i$ are units. 
    
    To prove existence, we proceed by strong induction on the degree of $f$. 
    So suppose the theorem is true for $\deg f < n$. 
    If $f$ is irreducible, then we are done. 
    Suppose $f$ is not irreducible. 
    Then there exists polynomials $a, b$ such that $f = ab$ and 
    neither $a$ nor $b$ are units. 
    That implies $\deg f = \deg a + \deg b > \deg a, \deg b$. 
    Hence by assumption, $a, b$ are both products of irreducibles
    which implies $f$ is a product of irreducibles, completing the induction. 
\end{proof}

\section{A Little More on Vector Spaces}

% Direct Sum of Vector Spaces
\begin{dfn} Direct Sum of Vector Spaces.
    
    The idea is to be able to formally add vectors from different vector spaces
    together in some larger vector space.
    Let $(V_i)_{i\in I}$ be a collection of $K$-vector spaces.
    The \textbf{direct sum}, $\bigoplus_{i\in I}V_i$, 
    will contain finite linear combinations of vectors from the $(V_i)_{i\in I}$.
    It is formalised as follows. 
    Let $\bigsqcup_{i\in I} V_i$ be the disjoint union 
    of the collection of $K$-vector spaces.
    Define the direct sum to be \[
        \bigoplus_{i\in I} V_i := \{f : I \to \bigsqcup_{i\in I} V_i \mid 
        f(i) \in V_i, \supp f \text{ finite}\}
    \]
    Then this is naturally a $K$-vector space under pointwise addition and scaling.
    For all $i \in I$, we have a natural injective morphism of $K$-vector spaces 
    $\iota_i : V_i \to \bigoplus_{i\in I} V_i$ by interpreting
    $v \in V_i$ as the function \[
        v : I \to \bigsqcup_{i\in I} V_i, j \mapsto \begin{cases}
            v   &, j = i\\
            
            0   &, j \neq i
        \end{cases}
    \]
    Then, as desired, all vectors $f$ in the direct sum can be written as
    a finite linear combination \[
        f = \sum_{i \in I} v_i , v_i \in V_i
    \]
\end{dfn}

\begin{thm} The ``Characterising" Property of the Direct Sum.
    
    Let $(V_i)_{i\in I}$ be a collection of $K$-vector spaces.
    Then the direct sum is the \emph{smallest} vector space 
    containing $(V_i)_{i\in I}$ in the sense that
    for any $K$-vector space $W$ and 
    $(f_i : V_i \to W)_{i\in I}$ $K$-vector space morphisms,
    there exists a unique morphism of $K$-vector spaces 
    $f : \bigoplus_{i\in I} V_i \to W$ such that for all $i\in I$, 
    $f_i = f \circ \iota_i$. 
    I.e. the diagram below commutes for all $i \in I$.  
    \begin{figure} [ht]
        \centering
        \begin{tikzcd}
        \bigoplus_{i\in I} V_i \arrow{r}{f} & W\\
        V_i \arrow{u}{\iota_i} \arrow{ru}[sloped, below]{f_i}& 
        \end{tikzcd}
    \end{figure}
    
    Hence if $W$ with $(f_i : V_i \to W)_{i\in I}$ also has the above property,
    $f$ becomes an \emph{isomorphism} between the direct sum and $W$.
    So the above property completely characterises the direct sum. 
\end{thm}
\begin{proof}
    Let $W$ be a $K$-vector space and 
    $(f_i : V_i \to W)_{i\in I}$ $K$-vector space morphisms.
    Let $v$ be a vector in the direct sum, 
    i.e. $v = \sum_{i\in I} v_i$ is a finite sum where $v_i \in V_i$. 
    $f$ being a $K$-vector space morphism and $f \circ \iota_i = f_i$ implies \[
        f(v) = \sum_{i\in I} f(v_i) = \sum_{i\in I} f_i(v_i)
    \]
    This serves as a definition of $f$ and proves uniqueness of $f$. 
    
    Note that since the direct sum itself trivially has the above property, 
    there is a unique morphism from it to itself commuting with all $\iota_i$,
    namely the identity morphism of the direct sum. 
    
    Suppose $W$ also have the above property. 
    Then there exists a unique morphism $g : W \to \bigoplus_{i\in I} V_i$
    such that all $g \circ f_i = \iota_i$. 
    Then it is easily verified that $g \circ f$ is a morphism 
    from the direct sum to itself that commutes with all $\iota_i$.
    But we know this must be the identity, hence $g \circ f = id$. 
    Similarly, $f \circ g = id$. 
    So $\bigoplus_{i\in I} V_i \iso W$ as $K$-vector spaces via $f$. 
\end{proof}
% Free Vector Space as direct sum indexed by set
\begin{ex} Free Vector Space is a Direct Sum.
    
    Let $S$ be a set. 
    Show that $\<S\> \iso_{\KVec} \bigoplus_{s\in S} K$ where the latter is
    the direct sum of copies of $K$ indexed by $S$. 
    This gives another construction for the free vector space over a set. 
    
    In particular, $K^n \iso K \oplus \cdots \oplus K$ $n$-times. 
\end{ex}

% Linear Indep, Spanning, Basis as Inject, Surject, Biject
\begin{dfn} Linear Independence, Spanning, Basis.
    
    Let $V$ be a $K$-vector space and $\{v_i\}_{i\in I}$ a subset of vectors.
    Then via the inclusion $\iota : \{v_i\}_{i\in I} \to V$ and
    the characterising property of free vector spaces, 
    we have a $K$-vector space morphism \[
        \bar\iota : \bigoplus_{\{v_i \mid i\in I\}} K \to V, 
        \sum_{i\in I}\la_i v_i \mapsto \sum_{i\in I}\la_i v_i
    \]
    Then \begin{enumerate}
        \item If $\bar\iota$ injects,
        the subset of vectors $\{v_i\}_{i\in I}$ 
        is called \textbf{linearly independent}. 
        \item If $\bar\iota$ surjects, 
        the subset of vectors $\{v_i\}_{i\in I}$ is called \textbf{spanning}.
        \item If $\bar\iota$ bijects,
        the subset of vectors $\{v_i\}_{i\in I}$ is called a \textbf{basis}.
    \end{enumerate}
    
    These three definitions are a sort of way of measuring the 
    "vector space data" of a subset of vectors.
    Linearly independent is no-redundant data, spanning is too much data, and
    a basis is just the right amount. 
\end{dfn}

\begin{dfn} Dimension of a Vector Space.

    Let $\{v_i\}_{i\in I}$ be a basis of $V$. 
    Note that this is equivalent to $V$ being a free vector space, namely, 
    \[
        V \iso \bigoplus_{\{v_i \mid i\in I\}} K \iso \<\{v_i\}_{i\in I}\> 
    \]
    Then the \textbf{dimension of $V$} is defined as the cardinality of the basis
    and is denoted, $\dim_K V$. 
    
    We will only prove the well-definedness of dimension for finite-bases
    \footnote{Bases is plural for basis.}, 
    since we will only need finite-dimensional vector spaces for Galois theory
    and the proof for general cardinality bases is somewhat tricky. 
\end{dfn}
% Steinitz Exchange / Dimension is Well-defined
\begin{lem} Steinitz Exchange / Finite Dimenion is Well-Defined. 

    Let $S = \{v_i\}_{i=1}^n, T = \{w_j\}_{j=1}^m$ be finite subsets 
    of a $K$-vector space $V$ such that 
    $S$ is linearly independent and $T$ is spanning. 
    Then $|S| \leq |T|$.
    
    Hence, given two finite bases of $V$, they have the same cardinality. 
\end{lem}
\begin{proof}
    The idea is to swap out vectors in $T$ by vectors in $S$, 
    creating an injection from $S$ to $T$. 
    ...
\end{proof}

\end{document}