\documentclass[../book.tex]{subfiles}

\begin{document}

In this chapter we prove the impossiblility of certain constructions
using compasses and straight edges, which the ancient greeks were fond of.

\begin{ex} Operations of Arithmetic using Compasses and Straight Edges.
    
    Let lengths in $\R^2$ represent real numbers.
    \footnote{A length is exactly what you think it is.}
    Suppose we are given a unit length representing 1 and
    two lengths representing real numbers $a, b$.
    Show that using compasses (circles) and straight edges (straight lines),
    we can construct lengths representing $a + b, a - b, a b , a / b$
    where $0 \neq b$. 
    In particular, we can construct lengths representing any $q \in \Q$.
    
    [Hint : For multiplication and division,
    show that you can make right angle triangles with any base and height.
    Hence obtain $ab, a/b$ via similar triangles.]
    
\end{ex}
The above exercise motivates the following definition. 
% Def - Constructible Numbers.
\begin{dfn} Constructible Extensions.
    
    Let $K$ a subfield of $\R$. 
    The cartesian product $K^2$ is a subset of the plane $\R^2$. 
    A \textbf{$K$-line} is defined as the points $(x,y) \in \R^2$ satisfying
    $a x + by = c$ for some $a, b, c \in K$ not all zero.
    A \textbf{$K$-circle} is defined as points $(x,y) \in \R^2$ satisfying
    $(x - a)^2 + (y - b)^2 = c^2$ for some $a, b, c \in K$.
    
    An \textbf{elementary $K$-construction} is 
    the intersection of any of the following pairs (if non-empty): 
    \begin{itemize}
        \item Two distinct $K$-lines
        \item A $K$-line and a $K$-circle
        \item Two distinct $K$-circles
    \end{itemize}
    The components of an elementary construction from $K$ are called
    \textbf{elementarily $K$-constructible}.
    
    Then a $K$-extension $L \subseteq \R$ is called \textbf{constructible} when
    there exists a finite set of generators $z_0,\dots,z_n$ of $L$ such that 
    \begin{enumerate}
        \item $z_0$ is elementarily $K$-constructible.
        \item For $0 < k \leq n$,
        $z_k$ is elementarily $K(z_0,\dots,z_{k-1})$-constructible.
    \end{enumerate}
    A real number $z$ is called \textbf{$K$-constructible} when
    it lies in some constructible $K$-extension. 
    
    In particular, 
    \textbf{constructible numbers} refer to $\Q$-constructible numbers in $\R$.
    
    Note that for extension $K \to L \to M$,
    $K \to L$ and $L \to M$ constructible implies $K \to M$ constructible. 
\end{dfn}

The following is the key theorem regarding constructible numbers. 

% Thm - Constructible Numbers iff Q-ext of degree power 2.
\begin{thm} $\Q$-extension Constructible if and only if Tower of Degree 2's.
    
    Let $L \subseteq \R$ be a $\Q$-extension. 
    Then $L$ is constructible if and only if 
    there exists a subfields $L_0, \dots, L_n$ such that 
    $L_0 = \Q$, $L_n = L$ and $L_i \subseteq L_{i+1}$ with $[L_{i+1} : L_i] = 2$.
    
\end{thm}
\begin{proof}
    \begin{itemize}
    \item (Forward Implication)
    
    We show that for any field $K \subseteq \R$ and real number $z$,
    $z$ is elementarily $K$-constructible implies
    $[K(z) : K] = \deg\min(z,K) \leq 2$.
    From this, the forward implication is clear.
    
    Let $K \subseteq \R$ be a $\Q$-extension and 
    $z$ a real number that is elementarily $K$-constructible.
    
    ($z$ is a component of the intersection of two $K$-lines)
    
    Then by easy computation, $z$ is in $K$, and so $\deg\min(z,K) = 1$.
    
    ($z$ is a component of the intersection of a $K$-line and a $K$-circle)
    
    Let $z = x$ or $y$ in $\R$ where $ax + by = c$ and $(x - u)^2 + (y - v)^2 = w^2$
    with $a, b, c, u, v, w$ in $K$, not all $a, b, c$ are zero.
    If $a = 0$ and $b = 0$, $0 = c$ gives a contradiction.
    So $a$ or $b$ is non-zero, WLOG it is $a$.
    Then $x = (c - by) / a$ and we have \[
        (\frac{c - by}{a} - u)^2 + (y - v)^2 = w^2
    \]
    which gives a quadratic over $K$ with $y$ as a root,
    i.e. $\deg\min(y,K) =$ 1 or 2. 
    Since $K(x) \subseteq K(y)$, $\deg\min(x,K) =$ 1 or 2 as well.
    Thus, $\deg\min(z,K) =$ 1 or 2.
    
    ($z$ is a component of the intersection of two $K$-circles)
    
    Let $z = x$ or $y$ in $\R$ where $(x - a)^2 + (y - b)^2 = c^2$ and
    $(x - u)^2 + (y - v)^2 = w^2$ with $a, b, c, u, v, w$ in $K$.
    Then \begin{align*}
        x^2 - 2ax + a^2 + y^2 - 2by + b^2 &= c^2 \\
        x^2 - 2ux + u^2 + y^2 - 2vy + v^2 &= w^2 \\
        \iff 
        x^2 - 2ax + a^2 + y^2 - 2by + b^2 &= c^2 \\
        2(u - a)x + 2(v - b)y = c^2 - w^2 + u^2 - a^2 + v^2 - b^2 &= A x + B y = C
    \end{align*}
    with $A, B, C$ are in $K$, not all zero.
    So this is the same case as the intersection of a $K$-line and a $K$-circle.
    
    All in all, this shows $z$ elementarily $K$-constructible 
    implies $[K(z) : K] = \deg\min(z,K) \leq 2$.
    
    \item (Reverse Implication)
    
    Now suppose there exists a subfields $L_0, \dots, L_n$ such that 
    $L_0 = \Q$, $L_n = L$ and $L_i \subseteq L_{i+1}$ with $[L_{i+1} : L_i] = 2$.
    We show that every $L_{i+1}$ is a constructible $L_i$-extension.
    
    It can easily be verified that $L_{i+1} = L_i(z_i)$ for some $z_i \in L_{i+1}$.
    Then we have $\deg\min(z_i,L_i) = 2$, 
    i.e. $\min(z_i,L_i) = X^2 + bX + c$ for some $b, c \in L_i$. 
    Then by the quadratic formula \[
        z_i = \frac{-b \pm \sqrt{b^2 - 4c}}{2}
    \]
    So $L_{i+1} = L_i(\sqrt{D})$ where $D = b^2 - 4c \in L_i$.
    We give a finite sequence of elementary $L_i$-constructions yielding $\sqrt{D}$,
    hence showing $L_{i+1}$ is a constructible $L_i$-extension.
    
    % insert pictures of elementary L_i-constructions.
    
    \end{itemize}
\end{proof}

% Cor - Cannot Trisect General Angle
\begin{cor} Impossibility of Angle Trisection.
    
    Let $\theta$ be an angle. 
    Having $\theta$ is the same as having $\cos{\theta}$,
    hence constructibility of $\theta$ is equivalent to 
    constructibility of $\cos{\theta}$.
    
    Then there exists $\theta$ such that 
    $\cos{\theta / 3}$ is not constructible. 
    
\end{cor}
\begin{proof}
    
    By compound angle formula, we have \[ 
        \cos{\theta} = 4\cos{\theta/3}^3 - 3\cos{\theta/3}
    \]
    Let $\theta = 2\pi / 3$. Then $\cos{\theta/3}$ is a root of $8X^3 - 6X + 1$.
    5 is a prime not dividing 8 and 
    $8X^3 - 6X + 1 = 3X^3 - X + 1 \in \F_5[X]$ is irreducible.
    Hence $8X^3 - 6X + 1$ is irreducible in $\Q[X]$ and is equal to
    the minimal polynomial of $\cos{\theta/3}$ over $\Q$. 
    It follows then $\cos{\theta/3}$ cannot lie in a constructible $\Q$-extension,
    i.e. it is not constructible.
    
\end{proof}

\begin{rmk}
    
    The choice of $\theta = 2\pi / 3$ in the above proof
    is equivalent to the impossibility to construct the regular 9-gon. 
    We will have more to say about constructibility of regular $n$-gons 
    in the chapter on cyclotomic extensions.
    
\end{rmk}

% Cor - Cannot Double General Cube
\begin{cor} Impossibility of Doubling a Cube
    
    There exist cubes such that 
    there does not exist cubes with double its volume.
    
\end{cor}
\begin{proof}
    
    Take the unit cube. 
    A cube with double its volume has $2^{1/3}$ as its length.
    To construct such a cube is equivalent to constructing $2^{1/3}$.
    But $X^3 - 2$ has $2^{1/3}$ as a root and is irreducible by Eisenstein's criterion.
    So the minimal polynomial of $2^{1/3}$ is $X^3 - 2$.
    It follows that $2^{1/3}$ cannot lie in a constructible $\Q$-extension.
    
\end{proof}

% Cor - Cannot Square Circle
\begin{cor} Impossibility of Squaring a Circle
    
    There exist circles such that
    there does not exist squares with equal area.
    
\end{cor}
\begin{proof}
    
    Take the circle of area $\pi$. 
    Constructing a square with area $\pi$ is the same as constructing $\sqrt{\pi}$.
    In particular, if such a square is constructible,
    this implies $\sqrt{\pi}$ is algebraic over $\Q$,
    and thus also is $\pi$.
    This is a contradiction with $\pi$ being transcendental.
    \footnote{This is quite a famous theorem. 
    A proof may be found on proofwiki.org.
    }
    
\end{proof}

\end{document}