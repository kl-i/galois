\documentclass[../book.tex]{subfiles}

\begin{document}

%    -  Cor - Fund Thm Alg
\begin{thm}
    For all polynomials $f \in \mathbb{C}[x]$, $\mathbb{C}$ splits $f$.
\end{thm}

To prove this, we need to notice two lemmas first:

%        -  No non-trivial odd deg ext of R
\begin{lem}
    There does not exists non-trivial $\R$-extension $(L,\iota_L)$ 
    such that $[L : \R]$ is odd.
\end{lem}
\begin{proof}
    Assume such $L$ exist, pick $ \al \in L\setminus \iota_L\R$.
    By the tower law of extension degree, 
    it will have a minimal polynomial with odd degree. 
    By the intermediate value theorem, there exist real root $a$ in $\R$.
    Then by irreducibility, $\min(\al,\R) = (X - a)$, 
    i.e. $\al = \iota_L(a) \in \iota_L\R$.
    Since $\al$ was arbitrary, all of $L$ is actually in $\iota_L\R$.
    Hence, $L$ is a trivial extension, which is a contradiction. 
    
\end{proof}
%        -  No non-trivial even deg ext of C 
\begin{lem}
    There does not exists a $\C$-extension $(L,\iota_L)$ 
    such that $[L : \mathbb{C}]=2$.
\end{lem}
\begin{proof}
    Assume such $(L,\iota_L)$ exist and 
    pick $ \al \in L \setminus \mathbb{C}$. 
    It will have a minimal polynomial with degree 2. 
    However $\mathbb{C}$ must split any quadratic polynomial.
    So $\al$ is actually already inside $\iota_L \C$.
    Since $\al$ was arbitrary, $L$ is actually a trivial extension of $\C$,
    i.e. $[L : \C] = 1 \neq 2$, a contradiction. 
\end{proof}

\end{document}