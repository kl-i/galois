\documentclass[../book.tex]{subfiles}

\begin{document}
%    -  Def - char
%    -  Lem - char prime
%    -  Lem - char K = p iff K/Fp
%    -  Lem - K/L -> char K = L
%    -  Lem - Freshmans’ Dream

%    -  Thm - Classification 1.  char K = p -> o(K) = p^n
%    -  Lem - K/{x^p^n - x = 0} 
%    -  Lem - char K = p -> x^p^n - x sep in K[x]
%    -  Thm - Classification 2.  o(K) = p^n -> K splitting x^p^n - x
%    -  Thm - Classification 3.  o(K) = o(L) fin -> K ≅ L
%    -  Thm - Classification 4. Exists o(K) = p^n
%    -  Def - Fp^n
%    -  Thm - Classification 5.  Fp^n = Fp(ω), ω prim-(p^n-1)-root
%    -  Thm - Classification 6. Fp^m/Fp^n iff n|m
%    -  Def - Frob(p^n,K/Fp^n)
%    -  Thm - Classification 7. Fp^nm/Fp^n Galois AND Galois Grp Generated by Frob(p^n)
%    -  Thm - Classification 8. (Fp^n,+) ≅ Cp^n AND (Fp^n*, *) ≅ Cp^n-1

\begin{thm} All Fields contain a Copy of $\F_p$ or $\Q$ and 
Characterisic passes up Extensions.
    
    Let $p \in \Z$ be a prime. 
    Then the ring quotient of $\Z$ by the ideal generated by $p$ 
    is a field with $p$ elements, denoted $\F_p$.
    
    Furthermore, for all fields $K$, the following are true. 
    \begin{enumerate}
        \item If $K$ has characteristic zero, 
        then there is a \emph{unique} ring morphism from $\Q$ into $K$,
        i.e. there is a unique extension from $\Q$ to $K$.
        \item If $K$ has non-zero characteristic,
        then there is a \emph{unique} ring morphism from $\F_p$ into $K$
        for some prime integer $p$,
        i.e. there is a unique extension from $\F_p$ to $K$.
    \end{enumerate}
    Also, if $\iota_L : K \to L$ is an extension,
    then the characteristic of $L$ is the same as that of $K$'s.
    
\end{thm}
\begin{proof}
    
    $\F_p = \Z / (p) $ is a field with $p$ elements is done 
    in the exercise on Fermat's Little theorem.
    
    Let $K$ be a field. 
    Recall that there exists only one ring morphism from $\Z$ to $K$. 
    
    If $K$ has zero characteristic, then this morphism is injective.
    In particular, non-zero integers are mapped to non-zero elements in $K$,
    i.e. multiplicatively invertible elements. 
    Since multiplicative inverses are unique, 
    we have an obvious ring morphism from $\Q$ to $K$, 
    $a / b \mapsto a / b$, where $a, b$ are integers with $b$ non-zero. 
    We leave it as an exercise for the reader to check the uniqueness of this map.
    
    On the otherhand, if $K$ has non-zero characteristic $p$, it must be prime
    and as proven before, we have a ring morphism from $\F_p = \Z / (p)$ to $K$.
    The uniqueness of this map is given by the 1st isomorphism theorem.
    
    Let $\iota_L : K \to L$ be an extension.
    We have a unique ring morphism from $\Z$ to $K$.
    This gives a ring morphism from $\Z \to L$ via composition with $\iota_L$.
    By uniqueness, this is the only ring morphism from $\Z$ to $L$.
    Clearly, the kernel of $\Z \to L$ is equal to the kernel of $\Z \to K$.
    Hence $L$ has the same characteristic as $K$.
    
\end{proof}

\begin{thm} Classification of Finite Fields.
    
    Let $p$ be a prime. Then the following are true. 
    \begin{enumerate}
        \item (Existence)
        
        For every positive power of $p$, $p^n$, 
        the splitting field of $X^{p^n} - X \in \F_p[X]$
        is a field with $p^n$ elements.
        In particular, $F_p$ itself is the splitting field of $X^p - X$. 
        \item (Uniqueness)
        
        All finite fields have non-zero characteristic and 
        those with characteristic $p$ are in fact isomorphic to
        the splitting field of $X^{p^n} - X \in \F_p[X]$ for some $p$-power $p^n$. 
        Since splitting fields are unique up to isomorphism,
        we use $\F_{p^n}$ to denote any of these isomorphic fields
        and refer to it as \textbf{\emph{the} field with $p^n$ elements}. 
        \item (Primitive)
        
        The extension $\iota : \F_p \to \F_{p^n}$ is a simple extension,
        In particular, the Primitive Element theorem holds for finite fields
        and so the Galois correspondence, too. 
        \item (Galois from $\F_p$)
        
        The unique extension $\iota : \F_p \to \F_{p^n}$ is Galois and 
        its Galois group is a cyclic group generated by 
        the \textbf{Frobenius map}, $Frob(p) : x \mapsto x^p$. 
        \item (Galois)
        
        Let $\F_{p^n}$ be a finite field with characteristic $p$. 
        Then for every $d$ that divides $n$, 
        there exists a unique subfield with $p^d$ elements, 
        i.e. a copy of $\F_{p^d}$, giving an extension $\F_{p^d} \to \F_{p^n}$.
        Conversely, if we have a $\F_p$-extension $\F_{p^d} \to \F_{p^n}$,
        then $d$ divides $n$. 
        
        In both cases, 
        $\F_{p^d} \to \F_{p^n}$ is a Galois extension and 
        its Galois group is a cyclic group generated by 
        the $n/d$-th power of the Frobenius map. 
    \end{enumerate}
    
\end{thm}
\begin{proof}
    
    (1)
        
        The derivative of $X^{p^n} - X$ is $-1$, 
        which is coprime to $X^{p^n} - X$ itself.
        So $X^{p^n} - X$ is a separable polynomial. 
        In particular, it has $p^n$ roots. 
        Suppose $a, b$ are both roots.
        Clearly, $ab$ is a root and if $b \neq 0$, then $a / b$ is a root as well.
        Furthermore, by Freshmen's Dream, we have \[
            (a \pm b)^{p^n} = (a^p \pm b^p)^{p^{n-1}} = \cdots
            = a^{p^n} \pm b^{p^n} = a \pm b
        \]
        i.e. the roots of $X^{p^n} - X$ form a field.
        The splitting field must be equal to this, and hence has $p^n$ elements.
        
    (2)
        
        Let $K$ be a finite field. 
        It is easy to show that it has non-zero and hence prime characteristic. 
        So let the characteristic of $K$ be $p$. 
        Then we have a unique extension $\F_p \to K$. 
        Since $K$ is finite as a set, 
        it must have finite dimension as a $\F_p$-vector space.
        So $K$ is isomorphic as a $\F_p$-vector space to $\bigoplus_{i < n} \F_p$ 
        for some positive natural $n$ and hence \[
            |K| = |\bigoplus_{i < n} \F_p| = p^n
        \]
        
        Let $a$ be an element of $K$.
        If it is zero, it is clearly a root of $X^{p^n} - X$.
        If it is non-zero, it is an element of the finite group $(K^\times,\cdot)$,
        so $a^{p^n - 1} = a^{|K^\times|} = 1$.
        Hence $a$ is again a root. 
        Since $K$ has $p^n$ elements, all roots of $X^{p^n} - X$, 
        it contains all the roots of $X^{p^n} - X$,
        and thus must be the splitting field of $X^{p^n} - X$.
        
    (3)
        
        To show that $\F_p \to \F_{p^n}$ is a simple extension,
        note that the group of units $\F_{p^n}$ is a finite group
        satisfying the condition that the number of solutions to $a^n = 1$ 
        is bounded by $n$. 
        Hence, it is a cyclic group with generator $a$ say. 
        Then clearly $\F_{p^n} = \F_p(a)$. 
        
        Now let $\iota_L : K \to L$ be a finite extension where $K$ is finite.
        Then both $K$ and $L$ have the same non-zero characteristic.
        WLOG it's $p$, which gives $L \iso \F_{p^n}$ as a $\F_p$-extension
        for some positive integer $n$. 
        Then since $\F_{p^n}$ is a simple extension of $\F_p$,
        $L$ is also a simple extension of $\F_p$,
        and thus of $K$ as well. 
        
    (4)
        
        Note that for any element $a \in \F_{p^n}$,
        $a$ is fixed by $Frob(p)$ if and only if it is the root of $X^p - X$. 
        But the set of all roots of $X^p - X$ is $\iota \F_p$!
        So clearly $\iota \F_p = \F_{p^n}^{\<Frob(p)\>}$,
        and hence $\iota : \F_p \to F_{p^n}$ is Galois by definition 
        with $\aut{\F_p}{\F_{p^n}} = \<Frob(p)\>$.
        
    (5)
        
        Let $d$ divide $n$.
        Then since the Galois group $\aut{\F_p}{\F_{p^n}}$ is isomorphic to $C_n$,
        there exists a unique subgroup $H$ of $\aut{\F_p}{\F_{p^n}}$ with order $d$. 
        So by the Galois correspondance, 
        $L^H$ is the unique subfield with cardinality $p^d$.
        The extension $L^H \to \F_{p^n}$ is Galois by definition.
        
        The converse is obvious. 
        
        As for the Galois group $\aut{\F_{p^d}}{\F_{p^n}}$,
        by the Galois correspondance, 
        it is a subgroup of order $n/d$ 
        of the cyclic group $\aut{\F_{p^d}}{\F_{p^n}}$.
        Hence, it is also cyclic, and $Frob(p)^{n/d}$ is clearly a generator.
        
\end{proof}
%
%-  Lem - Dedekind’s 
%-  Thm - Deg Sep leq Deg Ext
%    -  Dedekind


\end{document}