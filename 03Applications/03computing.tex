\documentclass[../book.tex]{subfiles}

\begin{document} % This chapter basically from J. Milne's notes on Galois. 

The following interpretation of Galois groups is essential for computation. 

% Def. Galois Group of a Separable Polynomial
\begin{thm} The Galois Group of a Separable Polynomial. 
    
    Let $\iota_L : K \to L$ be a Galois extension. 
    It is easy to check that this is equivalent to $(L,\iota_L)$ being
    the splitting field $(K_f,\iota_f)$ of some separable polynomial $f$ over $K$.
    Let $S_f$ be the roots of $f$ in $K_f$. 
    Consider the permutations $\sigma$ of $S_f$ such that 
    any algebraic equation over $K$ satisfied by $S_f$ is preserved, 
    i.e. for all polynomials $P$ over $K$ with indeterminates indexed by $S_f$,
    \[
        ev_{(a)_{a\in S_f}} (\bar{\iota_f} P) = 0 
        \imp ev_{(\sigma(a))_{a\in S_f}} (\bar{\iota_f} P) = 0
    \]
    These permutations form a group,
    which is defined as the \textbf{Galois group of $f$}, denoted $G_f$.
    
    Then $G_f$ is isomorphic to the Galois group 
    of the splitting field $(K_f,\iota_f)$ of $f$. 
    
\end{thm}

To prove this, we need a generalised 1st isomorphism theorem for rings. 

\begin{lem} Generalised 1st Isomorphism Theorem. 
    
    Let $\phi, \psi : R \to S$ be ring morphisms between rings $R, S$ such that
    the kernel of $\phi$ is a subset of the kernel of $\psi$. 
    Then there exists a unique ring morphism $\bar{\psi}$ 
    from the quotient $R / \ker\phi$ to the image of $\psi$
    such the following diagram commute. \begin{figure}[H]
        \centering
        \begin{tikzcd}
        R \arrow[r,"\pi"] \arrow[rd,"\psi"{swap}] & 
        R / \ker\phi \arrow[d,"\bar{\psi}"] \\
        & 
        \psi R
        \end{tikzcd}
    \end{figure}
    In particular, in conjunction with the 1st isomorphism theorem,
    we can see $\bar{\psi}$ as a morphism from 
    the image of $\phi$ to the image of $\psi$ instead. 
    
\end{lem}
\begin{proof}
    
    By requiring the diagram to commute, 
    we are forced to define $\bar{\psi}(a + \ker\phi) := \psi(a)$.
    This is independent of the choice of $a$ since $\ker\phi \subseteq \ker\psi$. 
    Hence we have existence. 
    Uniqueness is obvious. 
\end{proof}

Now the proof of $G_f \iso \aut{K}{K_f}$.  

\begin{proof} 
    
    We begin by noting 
    the set of all algebraic equations over $K$ satisfied by the roots of $f$ 
    is precisely the kernel of the evaluation map $ev_{(a)_{a\in S_f}}\circ\bar{\iota_f}$
    from the ring of polynomials over $K$ with indeterminates indexed by $S_f$. 
    So the definition of $G_f$ reads \[
        G_f = \{ \sigma \in \aut{\Set}{S_f} \mid 
        \ker (ev_{(a)_{a\in S_f}} \circ \bar{\iota_f}) \subseteq 
        \ker (ev_{(\sigma(a))_{a\in S_f}}\circ \bar{\iota_f} )\}
    \]
    
    Let $\sigma$ be an element of the Galois group of the splitting field $K_f$ of $f$. 
    Then clearly for any polynomial $P$ over $K$ satisfied by $S_f$, \[
        ev_{(\sigma(a))_{a\in S_f}} (\bar{\iota_f} P) 
        = ev_{(\sigma(a))_{a\in S_f}} ( \bar{\sigma}(\bar{\iota_f} P) )
        = \sigma (ev_{(a)_{a\in S_f}} (\bar{\iota_f} P))
        = \sigma (0) = 0
    \]
    So the restriction of $\sigma$ to the roots of $f$ is an element of $G_f$,
    i.e. we have the map \[
        \aut{K}{K_f} \to G_f, \sigma \mapsto \res{\sigma}{S_f}
    \]
    which is clearly a group morphism. 
    
    We claim the above is an isomorphism. 
    To see injectivity, 
    note that since $K_f$ is generated by $S_f$ as a $K$-extension, 
    any $K$-extension automorphism $\sigma$ is determined by its image on $S_f$. 
    For surjectivity, we use the generalised 1st isomorphism theorem. 
    Let $\sigma$ be a permutation of $S_f$ in the Galois group of $f$. 
    Then the fact that 
    $\ker (ev_{(a)_{a\in S_f}} \circ \bar{\iota_f}) \subseteq 
    \ker (ev_{(\sigma(a))_{a\in S_f}}\circ \bar{\iota_f} )\}$
    gives a ring morphism $\bar{\sigma}$ that makes the following diagram commute.
    \begin{figure} [H]
        \centering
        \begin{tikzcd} [sep = huge]
        K[\{X_a\}_{a\in S_f}] \arrow[r,"(ev_{(a)_{a\in S_f}} \circ \bar{\iota_f})"] 
        \arrow[rd,"(ev_{(\sigma(a))_{a\in S_f}}\circ \bar{\iota_f} )"{swap}] & 
        (ev_{(a)_{a\in S_f}} \circ \bar{\iota_f}) K[\{X_a\}_{a\in S_f}] 
        = K(S_f) = K_f
        \arrow[d,"\bar{\sigma}"] 
        \\
        & 
        (ev_{(\sigma(a))_{a\in S_f}}\circ \bar{\iota_f} ) K[\{X_a\}_{a\in S_f}] 
        = K(S_f) = K_f 
        \end{tikzcd}
    \end{figure}
    Since $K_f$ is generated by $S_f$ and the image of $S_f$ under $\sigma$ is $S_f$,
    we have the equalities in the diagram above. 
    It is then easy to see that $\bar{\sigma}$ is a $K$-extension automorphism
    of $K_f$ that when restricted to $S_f$, gives $\sigma$,
    i.e. we have surjectivity. 
    
\end{proof}

\section{Discriminant and the Alternating Group}

\begin{rmk}
    Let $f$ be a separable polynomial over $K$, $K_f$ its splitting field,
    $S_f$ its roots and $G_f$ its Galois group. 
    Then $G_f$ identifies with a subgroup of the symmetric group on $|S_f|$ elements.
    We now give a characterisation for when $G_f$ consists of even permutations. 
\end{rmk}

% Def. Discriminant of a Polynomial.
\begin{dfn} Discriminant of a Polynomial. 
    
    Let $f$ be a polynomial over a field $K$
    and let $(K_f,\iota_f)$ be its splitting field
    where $f$ factorises as \[
        f = \iota_f(\la) \prod_{i < \deg f} (X - a_i)
    \]
    Define \[
        \Delta(f) := \prod_{i < j < \deg f} (a_i - a_j)
    \]
    Then the \textbf{discriminant of $f$} is defined as \[
        D(f) := \Delta(f)^2
    \]
    It is easy to see that $D(f)$ is zero if and only if $f$ is separable. 
    Note that $\Delta(f)$ might differ by a sign depending on
    the labeling of the roots of $f$, but $D(f)$ is always the same. 
    
\end{dfn}

\begin{ex}
    Verify that the discriminant of a real quadratic $aX^2 + bX + c$
    is indeed the usual $b^2 - 4ac$. 
\end{ex}

% Thm. Discriminant in Base Field, 3 eqv of Galois subgroup Alternating
\begin{thm} 3 Equivalent Characterisations of Galois Group being Alternating.
    
    Let $f$ be a separable polynomial over $K$, $(K_f,\iota_f)$ its splitting field,
    $S_f$ the set of its roots and $G_f$ its Galois group. 
    Then the discriminant of $f$ is inside $K$ and 
    the subgroup of even permutations in $G_f$ corresponds 
    to the subfield $K(\Delta(f))$.
    Thus, the following are equivalent: 
    \begin{enumerate}
        \item $G_f$ is a subgroup of the alternating group $A_{|S_f|}$
        \item $\Delta(f)$ is in $K$.
        \item $D(f)$ is a square in $K$. 
    \end{enumerate}
    
\end{thm}
\begin{proof}
    
    Notice that any permutation $\sigma$ in the Galois group of $f$, 
    we have \[
        \sigma \Delta(f) = sign(\sigma) \Delta(f)
    \]
    The result now follows easily from the Galois correspondance. 
    
\end{proof}

% Over Q
\section{Galois Groups of Integer Coefficient Polynomials over the Rationals}

We first give some ways of checking irreducibility over $\Q$.

% Lem. For Cubics, via no roots.
\begin{ex} For up to Degree 3, via No Roots.
    
    Let $f$ be a polynomial over a field $K$
    with $\deg f \leq 3$ and no roots in $K$. 
    Show that $f$ is irreducible.
    
\end{ex}
% Lem. For Q, rational roots.
\begin{thm} Rational Roots.
    
    Let $f = \sum_{i \leq n} f_k X^k$ be a polynomial over $\Q$ 
    with $n = \deg f$ and $f_k$ integers. 
    Let $x \in \Q$ be a root of $f$. 
    WLOG $x = p/q$ where $p, q$ are integers with $q$ non-zero and $p, q$ coprime. 
    Then $p$ divides $f_0$ and $q$ divides $f_{n}$. 
    
\end{thm}
\begin{proof}
    
    Since $x$ is a root of $f$, we have \[
        0 = f_0 + f_1 \left( \frac{p}{q} \right) + \cdots 
        + f_{n-1} \left( \frac{p}{q} \right)^{n-1}
        + f_n \left( \frac{p}{q} \right)^n
    \]
    which we can rewrite as    
    \[
        - f_0 q^{n} = f_1 p q^{n - 1} + \cdots 
        + f_{n - 1} p^{n - 1} q + f_{n} p^{n}
    \]
    So $p$ divides $f_0 q^n$ and $q$ divides $f_n p^n$. 
    It is easy to deduce from coprimeness that 
    $p$ divides $f_0$ and $q$ divides $f_n$.
    
\end{proof}

\begin{rmk}
    
    The rational roots theorem can be used to check 
    whether cubics from with integer coefficients have roots,
    and hence whether they are irreducible. 
    
\end{rmk}

% Lem. For Q, Gauss.
\begin{lem} Gauss.
    
    Let $f = \sum_{i \leq n} f_k X^k$ be a polynomial over $\Q$ 
    with $n = \deg f$ and $f_k$ integers. 
    Suppose $f$ is irreducible inside the subring $\Z[X]$ 
    of polynomials with integer coefficients. 
    Then $f$ is irreducible inside $\Q[X]$. 
    
\end{lem}
\begin{proof}
    
    We go by the contrapositive. 
    Suppose there exists non-unit, non-zero polynomials $g, h$ over $\Q$
    such that $f = g h$. 
    It is easy to check that there exists integers $m, n$ such that 
    $m g$ and $n h$ are inside $\Z[X]$. 
    Let $g_0 = mg$ and $h_0 = nh$. 
    Then $m n f = g_0 h_0$ is in $\Z[X]$.
    
    Let $m n = \prod_{i < N} p_i$ where $p_i$ are prime numbers in $\Z$. 
    Consider $p_0$. By sending everything to $\F_{p_0}[X]$, 
    we have $0 = g_0$ or $0 = h_0$ in $\F_{p_0}[X]$. 
    WLOG $0 = g_0$. Then there exists polynomials $g_1, h_1$ with integer coefficients
    such that $g_0 = p_0 g_1$ and $h_1 = h_0$,
    i.e. \[
        \frac{mn}{p_0} f = g_1 h_1
    \]
    Continuing in this way, we can divide out all of $mn$ after some $k$-times,  
    giving $f = g_k h_k$ for some integer-coefficient polynomials $g_k, h_k$. 
    
\end{proof}
From the Gauss lemma, we deduce the following quick way of checking irreducibility. 
% Lem. For Q, via Eisenstein Criterion. 
\begin{thm} Eisenstein's Criterion.
    
    Let $f = \sum_{i \leq k} f_i X^i$ be a polynomial over $\Q$ 
    with integer coefficients and $\deg f = k$.
    Suppose there exists a prime $p$ such that \begin{enumerate}
        \item $p$ divides $f_0, \dots, f_{k-1}$.
        \item $p^2$ \emph{does not} divide $f_0$.
        \item $p$ \emph{does not} divide $f_k$.
    \end{enumerate}
    Then $f$ is irreducible. 
    
\end{thm}
\begin{proof}
    
    We go by contradiction. 
    Suppose $f = gh$ where $g, h$ are non-zero, non-unit polynomials over $\Q$. 
    By Gauss lemma, WLOG $g, h$ have integer coefficients. 
    Let $g = \sum_{i \leq m} g_i X^i$ and $h = \sum_{i \leq n} h_i X^i$
    where $m$ and $n$ are respectively degrees of $g$ and $h$. 
    Then $p$ divides $f_0 = g_0 h_0$. 
    Since $p^2$ does not divide $g_0 h_0$, 
    $p$ divides \emph{exactly one} of $g_0, h_0$.
    WLOG $p$ divides $g_0$ and not $h_0$. 
    Then since $p$ divides $f_1 = g_0 h_1 + g_1 h_0$, $p$ divides $g_1$.
    By continuing in this way through the coefficients of $f$,
    we deduce that $p$ divides all of $g_0, \dots, g_m$. 
    In particular, $p$ divides $g_m h_n = f_k$ which is a contradiction. 
    
\end{proof}

We also have the following useful trick. 

% Lem. For Q, via modulo non-dividing prime. 
\begin{thm} Modulo Non-Dividing Prime. 
    
    Let $f$ be a polynomial over $\Q$ with integer coefficeints. 
    Suppose there exists a prime $p$ 
    \emph{not} dividing the leading coefficient of $f$
    with $f$ irreducible when seen as a polynomial over $\F_p$. 
    Then $f$ is irreducible in $\Q[X]$. 
    
\end{thm}
\begin{proof}
    
    From the Gauss lemma, proving $f$ irreducible inside $\Z[X]$ is sufficient. 
    We go via the contrapositive. 
    Suppose $f = gh$ where $g$ and $h$ are non-zero, non-unit polynomials over $\Z$
    and let $p$ be a prime not dividing the leading coefficient of $f$.
    Then $p$ does not divide the leading coefficients of $g$ and $h$ either. 
    In particular, $g, h$ are non-zero when seen as polynomials over $\F_p$. 
    Hence $f$ is not irreducible inside $\F_p[X]$.
    
\end{proof}

The following gives a condition for irreducible polynomials of prime order
to have symmetric groups as their Galois groups.

% Thm. Irr, Prime Degree w 2 Complex roots gives S_p Galois Group
\begin{thm} Irreducible Prime Degree with 2 Complex Roots 
gives Symmetric Galois Group. 
    
    Let $f$ be an irreducible polynomial over $\Q$ with prime degree $p$.
    Suppose $f$ has exactly $p-2$ roots in $\R$. 
    Then its Galois group is the symmetric group on $p$ elements. 
    \[ G_f = S_p \]
    
\end{thm}
\begin{proof}
    
    We will show that $G_f$ contains a $2$-cycle and a $p$-cycle,
    which generates $S_p$, so $G_f = S_p$.
    
    We first show the existence of a $2$-cycle. 
    Since $f$ has exactly $p-2$ roots in $\R$, 
    as a polynomial over $\R$, it must have an irreducible quadratic factor. 
    Hence, by the quadratic formula, 
    the missing two roots of $f$ are the roots of said quadratic factor inside $\C$,
    i.e. $\C$ splits $f$. 
    In particular, the non-real roots are complex conjugates. 
    WLOG we can see the splitting field $\Q_f$ of $f$ as a subfield of $\C$.
    Consider the automorphism of $\C$ that is complex conjugation. 
    This can be seen as a $\Q$-extension automorphism of $\Q_f$
    which swap the two non-real roots and leave the real roots alone, 
    i.e. a $2$-cycle. 
    
    Now for the $p$-cycle, take a root $a$ of $f$ inside $\Q(f)$.
    Since $f$ is irreducible, it is a unit multiple of $\min(a,\Q)$. 
    So we have \[
        |G_f| = [\Q_f : \Q] = [\Q_f : \Q(a)] [\Q(a) : \Q]
        = [\Q_f : \Q(a)] \deg \min(a,\Q) = [\Q_f : \Q(a)] p 
    \]
    i.e. $p$ divides the order of the Galois group of $f$. 
    Then by Cauchy's theorem, there exists a permutation of order $p$ inside $G_f$. 
    It is easy to deduce from $p$ being prime that this permutation is a $p$-cycle. 
    
\end{proof}

Irreducibility of a polynomial is also related 
to the orbits of roots under the action of the polynomial's Galois group.
% Irreducibility and Transitivity
% Thm. Orbits correspond to Irreducible Factors
\begin{thm} Orbits of Roots correspond to Irreducible Factors.
    
    Let $f$ be a separable polynomial over a field $K$,
    $K_f$ its splitting field, $S_f$ the set of its roots, and $G_f$ its Galois group.
    Let $f = \prod_{i<n} f_i$ where $f_i$ are irreducible in $K[X]$.
    Then each irreducible factor $f_i$ corresponds 
    to an orbit $s_i$ with cardinality $\deg f_i$ of the roots of $f$
    under the action from $G_f$. 
    
    In particular, if $f$ is irreducible in $K[X]$, 
    then $S_f$ has only one orbit,
    i.e. the $G_f$-action on $S_f$ is transitive.
    
\end{thm}
\begin{proof}
    
    Let $f_i$ be an irreducible factor of $f$. 
    Let $s_i$ be the set of roots of $f_i$.
    The result then follows from all of $s_i$ being mutual Galois conjugates. 
\end{proof}

\begin{cor} Existence of Cycle Shapes in Galois Groups of Polynomials 
over Finite Fields.
    
    Let $f$ be a separable polynomial over a finite field $\F_{p^n}$ and 
    $L$ its splitting field. 
    Note that the Galois group of $f$ is then cyclic with 
    $\sigma = \mathrm{Frob}(p)^n$ as a generator.
    Suppose $f$ factorises into irreducibles $f_1, \dots, f_m$. 
    Then $\sigma$ has cycle shape of the degrees of $f_1,\dots,f_m$.
\end{cor}
\begin{proof}
    Since $\aut{\F_{p^n}}{L}$ is generated by $\sigma$, 
    the orbits of the $f$'s roots under the action from $\aut{\F_{p^n}}{L}$
    are the same as the disjoint cycles of $\sigma$. 
    The result then follows from the previous theorem. 
\end{proof}
\begin{rmk}
    
    There is a way to lift the above result up to Galois extensions of $\Q$.
    However, since the proof requires quite a bit of machinery from 
    algebraic number theory, we only quote the result. 
    
\end{rmk}
\begin{thm} Dedekind. 
    
    Let $f$ be a polynomial over $\Q$ with integer coefficients. 
    Suppose $p$ is a prime integer such that $\bar{f}$, the image of $f$ in $\F_p[X]$,
    is separable and factorises into irreducibles $f_1, \dots, f_m$.
    Then $f$ is separable and its Galois group contains 
    a permutation with cycle shape of the degrees of $f_1,\dots,f_m$.
    
\end{thm}

\end{document}