\documentclass[../../book.tex]{subfiles}

\begin{document}

This section is all about subgroups of prime powers. 

% Def - Order of an Element
\begin{dfn} Order of an Element.
    
    Let $g$ be an element of a group $G$. 
    Then the \textbf{order of $g$} is defined as 
    the minimal positive integer $o(g)$ such that $g^{o(g)} = e$. 
    If no such integer exists, we say $g$ has \textbf{infinite order}. 
    Note that if $o(g)$ exists, 
    it is equal to the cardinality of the subgroup generated by $g$.
    
\end{dfn}
\begin{ex} Order divides Candidate Orders.
    
    Show that if the order of $g$ exists, then
    for any positive integer $n$ such that $g^n = e$, 
    $o(g)$ divides $n$. 
    
\end{ex}

Groups with cardinality the power of a prime $p$ are called $p$-groups. 
The following is \emph{the} key lemma about $p$-Groups 
from which all of Sylow's theorems will be deduced. 

% Lem - Fixed points of p-Groups
\begin{dfn} Fixed Points in the Action of a $p$-Group. 
    
    Let $X$ be a $G$-set where $G$ is a $p$-group with cardinality $p^n$.
    Let $Fix(X)$ denote the set of elements of $X$ with orbit cardinality $1$. 
    Then $|X| = |Fix(X)|$ in $\F_p$. 
    
\end{dfn}
\begin{proof}
    
    For any $x \in X$, since $Stab(x)$ is a subgroup of $G$, 
    Lagrange's theorem dictates that $Stab(x)$ must have cardinality of power $p$.
    Hence by the Orbit-Stabiliser theorem, 
    $Orb(x)$ must also have power $p$ cardinality. 
    For $k \leq n$, let $m_k$ denote the number of orbits of $X$ with size $p^k$. 
    Since orbits partition $X$, we have \[
        |X| = m_0 + m_1 p + m_2 p^2 + \cdots + m_n p^n
    \]
    But of course, $m_0 = |Fix(x)|$ so we are done. 
\end{proof}

% Def - Normaliser Subgroup
\begin{dfn} Normaliser of a Subgroup. 
    
    Let $H$ be a subgroup of a group $G$. 
    Then the \textbf{normaliser of $H$} is defined as the smallest normal subgroup
    containing $H$, denoted $N_G(H)$. 
    It is in fact equal to the set of all $g$ such that $g^{-1}Hg \subseteq H$
    where $g^{-1} H g = \{ g^{-1}hg \mid h \in H \}$. 
    
\end{dfn}

% Thm - Sylow 1
\begin{thm} Sylow I. 
    
    Let $G$ be a finite group and $p$ a prime with divides $|G|$. 
    Suppose $p^n$ is the maximal power of $p$ dividing $|G|$. 
    Then there exists subgroups $H_1, \cdots, H_n$ such that
    $|H_i| = p^i$ and each $H_i$ is a normal subgroup of $H_{i+1}$. 
    Subgroups like $H_n$ with maximal prime power are called 
    \textbf{Sylow $p$-subgroups}.
    
\end{thm}
\begin{proof}
    
    We show the existence of the subgroups by induction. 
    
    (Cauchy's Theorem / Existence of $H_1$)
    
    We show that since $p$ divides $|G|$, 
    there exists an element $g$ with order $p$.
    \footnote{This is Cauchy's theorem.}
    Hence, $H_1 := \<g\>$ is a subgroup of order $p$.  
    
    To show such an element exists, we use a clever group action. 
    Let $X$ be the set of $p$-tuples $(g_1,\dots,g_p)$ of elements from $G$ 
    such that $g_1\cdots g_p = e$. 
    It is easy to show that if $g_1 \cdots g_p = e$, then $g_2 \cdots g_p g_1 = e$,
    and hence more generally any cyclic permutation also yields $e$. 
    This defines an action on $X$ from the cyclic group of order $p$. 
    Now here is the key: let $(g_1,\cdots,g_p)$ be in $Fix(X)$. 
    Since the action is cycling elements, we have \[
        (g_1,\cdots,g_{p-1},g_p) = (g_2,\cdots,g_p,g_1)
    \]
    i.e. $g_1 = g_2 = \cdots = g_p = g$ for some element $g$ 
    with $g^p = g_1\cdots g_p = e$. 
    Thus, the fixed points correspond to the elements we are looking for! 
    We know $(e,\cdots,e)$ is already a fixed point, 
    so if we can show there are more than one fixed points, we are done. 
    For that, we use the key lemma on 
    $X$ with the action from the cyclic group of order $p$,
    which gives $|X| = |Fix(X)| \in \F_p$. 
    Note that no matter what $g_1,\dots,g_{p-1}$ are, 
    as long as $g_p = g_{p-1}^{-1}\cdots g_1^{-1}$,
    we have $(g_1,\dots,g_p)$ is in $X$. 
    Hence $|X| = |G|^{p-1}$, which $p$ divides.
    Thus, $p$ divides $|Fix(X)|$, which is non-zero, so $1 < p \leq |Fix(X)|$. 
    
    (Existence of $H_{i+1}$ from $H_i$)
    
    Suppose we have the existence of the $i$-th subgroup $H_i$
    with order $p^i$ strictly less than the maximal power $p^n$. 
    Let $\pi_i$ be the projection of $N_G(H_i)$ to $N_G(H_i) / H_i$.
    The plan is to show that $p$ divides the cardinality of $N_G(H_i) / H_i$
    and hence by Cauchy's theorem, 
    we have a subgroup $\bar{H_i}$ of the quotient with cardinality $p$. 
    Then by the 3rd and 1st isomorphism theorem, 
    $H_i$ is normal in $\pi_i^{-1}\bar{H_i}$ which has order $p^{i+1}$,
    i.e. $H_{i+1} = \pi_i^{-1}\bar{H_i}$ works. 
    
    To show $p$ divides $|N_G(H_i) / H_i|$, we again use a clever action.
    Consider $G / H_i$, the set of all left cosets of $H_i$.
    Recall the $G$ has a natural action on $G / H_i$, \[
        G \mapsto \aut{\Set}{G / H_i}, g \mapsto (g_0H \mapsto g g_0H)
    \]
    This gives us an $H_i$-action on $G / H_i$. 
    $H_i$ is a $p$-group, so by the key lemma,
    we have $|G / H_i| = |Fix(G / H_i)| \in \F_p$.
    Since $|H_i|$ is not the maximal power of $p$ dividing $|G|$,
    $p$ divides $|G / H_i|$ as well, and hence $0 = |Fix(G / H_i)| \in \F_p$. 
    But note that \[
        gH_i \in Fix(G / H_i) 
        \iff \forall h \in H_i, h gH_i = gH_i
    \]
    \[
        \iff g^{-1} H_i g \subseteq H_i
        \iff g \in N_G(H_i) 
        \iff gH_i \in N_G(H_i) / H_i
    \]
    i.e. fixed cosets are precisely cosets of $H_i$ inside its normaliser. 
    Of course, the number of such cosets is equal to $|N_G(H_i) / H_i|$.
    Thus $p$ divides the cardinality of $N_G(H_i) / H_i$ and we are done. 
    
\end{proof}

% Thm - Sylow 2
\begin{thm} Sylow II.
    
    Let $G$ be a finite group with a prime $p$ dividing $|G|$.
    Let $Syl_p(G)$ denote the set of Sylow $p$-subgroups of $G$. 
    Note that for $A \in Syl_p(G)$ and $g \in G$, 
    $g^{-1} A g$ is also a Sylow $p$-subgroup,
    i.e. we have the following action. \[
        G \to \aut{\Set}{Syl_p(G)}, g \mapsto (A \mapsto g^{-1} A g)
    \]
    Then $Syl_p(G)$ has only one orbit under this action.
    In particular, there is one Sylow $p$-subgroup if and only if 
    there exists a normal Sylow $p$-subgroup. 
    
    In general, actions with only one orbit are called
    \textbf{transitive actions}. 
\end{thm}
\begin{proof}
    
    Let $A$ and $B$ be Sylow $p$-subgroups of $G$. 
    We wish to show there exists a $g \in G$ such that $B = g^{-1} A g$. 
    Note that for a $g \in G$, 
    \[
        B = g^{-1} A g \iff B = g^{-1} A g B \iff g B = A g B
    \]
    i.e. for all $a \in A$, $a g B = g B$. 
    Thus, by using the $A$-action \[
        A \to \aut{\Set}{G / B}, a \mapsto (gB \mapsto a gB)
    \]
    we see the fixed points of this action give us our desired element $g$. 
    Since $A$ is a $p$-group, by the key lemma we have
    $p$ divides $|G / B| - |Fix(G / B)|$.
    Then $B$ having cardinality maximal power $p$ implies 
    $p$ does not divide $|G / B|$, 
    and hence does not divide $|Fix(G / B)|$.
    In particular, the number of fixed points is non-zero. 
    
\end{proof}

% Thm - Sylow 3
\begin{thm} Sylow III.
    
    Let $G$ be a finite group with $|G| = k p^n$ where $p$ is prime
    and $0 < n$ the maximal power of $p$ dividing $|G|$. 
    Let $Syl_p(G)$ be the set of Sylow $p$-subgroups of $G$ and $n_p$ its cardinality.
    Then the following are true. \begin{enumerate}
        \item For any Sylow $p$-subgroup $A$, $n_p = [ G : N_G(A)]$.  
        \item $n_p = 1$ in $\F_p$. 
        \item $n_p$ divides $k$. 
    \end{enumerate}
    
\end{thm}
\begin{proof}
    
    (1)
    In Sylow II, we proved that the $G$-action on the Sylow $p$-subgroups
    that is $g \mapsto (A \mapsto g^{-1} A g)$ is transitive. 
    This implies for any Sylow $p$-subgroup $A$, its orbit has cardinality $n_p$.
    The result then follows from the Orbit-Stabiliser theorem. 
    
    (2)
    We will obtain the result by yet again using the key lemma. 
    Let $A$ be a Sylow $p$-subgroup of $G$. 
    Recall that we already have a $G$-action on the set of Sylow $p$-subgroups.
    This naturally gives us an $A$-action \[
        a \mapsto (B \mapsto a^{-1} B a)
    \]
    Then the key lemma says $n_p = |Fix(Syl_p(G))|$ inside $\F_p$. 
    We claim there is only one fixed point, $A$ itself. 
    
    Suppose $B$ is a Sylow $p$-subgroup that is fixed by the $A$-action.
    Then $A$ is inside the normaliser of $B$,
    i.e. $A$ and $B$ are both Sylow $p$-subgroups of $N_G(B)$.
    However, since $B$ is normal inside $N_G(B)$, by Sylow II,
    it must be the only Sylow $p$-subgroup of $N_G(B)$. 
    Thus, $A = B$ as desired.
    
    (3)
    Follows from (1) and (2) with elementary number theory. 
    
\end{proof}

\end{document}