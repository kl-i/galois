\documentclass[../../book.tex]{subfiles}

\begin{document}

The Galois correspondence gives us a way of exchanging results
between the world of groups and the world of extensions. 
Thus, in the following preliminary sections, 
we develop some group-theoretic results which we will be using. 

\begin{dfn} Symmetric Group, Permutations, Cycle Decomposition and Cycle Shape.
    
    Let $n$ be a natural number. 
    Then the \textbf{symmetric group on $n$-elements} is defined as
    the automorphism group of any set of cardinality $n$. 
    It is denoted $S_n$ and has $n$ factorial many elements. 
    WLOG we can use the automorphism group of the set $\{i\}_{i<n}$.
    An element $\sigma$ of $S_n$ is called a \textbf{permutation on $n$ elements},
    or simply a \textbf{permutation}. 
    
    For a permutation $\sigma$ of $S_n$, 
    the subgroup generated by $\sigma$ naturally acts on $\{i\}_{i<n}$. 
    Let the orbits of $\{i\}_{i<n}$ under the $\<\sigma\>$-action 
    be $\O_1,\dots,\O_s$. 
    For each orbit $\O_i$, by picking an element $x \in \O_i$, 
    one can see that, $\res{\sigma}{\O_i}$, 
    the restriction of $\sigma$ to $\O_i$ \emph{cycles} the elements of $\O_i$. 
    Let $\sigma_i$ be the extension
    of $\res{\sigma}{\O_i}$ to the whole of $\{i\}_{i<n}$
    by defining it as identity outside $\O_i$. 
    Then it is easy to see that we have \[
        \sigma = \sigma_n \circ \cdots \circ \sigma_1
    \]
    This is called the \textbf{cycle decomposition of $\sigma$}
    with the individual $\sigma_i$ called \textbf{cycles}. 
    
    Cycle decomposition gives us a way of notating permutations 
    called \emph{cycle notation}.
    Let $\sigma$ be a permutation with cycle decomposition 
    $\sigma = \sigma_n \circ \cdots \circ \sigma_1$. 
    For each cycle $\sigma_i$ with corresponding orbit $\O_i$,
    we write \[
        ( x_i \thickspace \sigma_i(x_i) \thickspace \sigma_i^2(x_i) 
        \thickspace \cdots \thickspace \sigma_i^{|\O_i| - 1}(x_i))
    \]
    where $x_i$ is any element of $\O_i$. 
    So the notation for $\sigma$ is \[
        ( x_n \thickspace \sigma_n(x_n) \thickspace \sigma_i^2(x_n) 
        \thickspace \cdots \thickspace \sigma_n^{|\O_n| - 1}(x_n))
        \cdots 
        ( x_1 \thickspace \sigma_1(x_1) \thickspace \sigma_1^2(x_1) 
        \thickspace \cdots \thickspace \sigma_1^{|\O_1| - 1}(x_1))
    \]
    
    An important characteristic of a permutation $\sigma$ is 
    its \textbf{cycle shape}, which is defined as 
    the list of the cardinalities of the orbits of $\{i\}_{i<n}$ 
    under the $\<\sigma\>$-action in ascending sizes. 
    \footnote{
        The requirement of ascending is arbitrary. 
        We are only interested in the list of numbers. 
    }
    
\end{dfn}

\begin{eg}
    
    Consider the permutation on $5$ elements, \[
        0 \mapsto 3, 1 \mapsto 4, 2 \mapsto 1, 3 \mapsto 0, 4 \mapsto 2
    \]
    This has cycle notation $(0\thickspace3) (1\thickspace4\thickspace2)$
    and cycle shape $[2, 3]$. 
    
\end{eg}

\begin{thm} The Order of a Permutation. 
    
    Let $\sigma$ be a permutation on $n$ elements 
    with cycle decomposition $\sigma = \sigma_n \circ \cdots \circ \sigma_1$. 
    Then the order of an individual cycle $\sigma_i$ is 
    the cardinality of corresponding orbit $\O_i$.
    Consequently, the order of $\sigma$ is the lowest common multiple
    of the cardinalities of its orbits. 

\end{thm}
\begin{proof}
    
    Left as an exercise to the reader. 
    [Hint : Use the fact that the cycles $\sigma_i$ commute with each other.]
    
\end{proof}

Another important feature of permutations is their \emph{sign}. 

\begin{dfn} Signs of Permutations, Even and Odd Permutations, Alternating Group. 
    
    Let $\sigma$ be a permutation on $n$ elements. 
    Define the polynomial in $n$ indeterminates
    \[
        P(X_i)_{i<n} := \prod_{i<j<n} (X_i - X_j)
    \]
    Then the \textbf{sign} or \textbf{parity} of $\sigma$ is defined as \[
        sign(\sigma) := \frac{P(X_{\sigma(i)})_{i<n}}{P(X_i)_{i<n}}
    \]
    Since $P(X_i)_{i<n}$ and $P(X_{\sigma(i)})_{i<n}$ have the same factors,
    only differing in signs, $sign(\sigma)$ is $\pm 1$. 
    One can see that the sign of $\sigma$ counts
    the number of pairs $(i,j)$ such that $i < j$ and $\sigma(i) > \sigma(j)$. 
    Permutations of sign $+1$ are called \textbf{even} 
    and otherwise called \textbf{odd}.
    
    Let $\sigma, \rho$ be permutations in $S_n$. 
    Then we have \[
        sign(\rho \circ \sigma) = \frac{P(X_{\rho(\sigma(i))})_{i<n}}{P(X_i)_{i<n}}
        = \frac{P(X_{\rho(\sigma(i))})_{i<n}}{P(X_{\sigma(i)})_{i<n}}
        \frac{P(X_{\sigma(i)})_{i<n}}{P(X_i)_{i<n}} 
        = sign(\rho) sign(\sigma)
    \]
    where the second equality comes from $\sigma$ being simply
    a relabeling of the set $\{i\}_{i<n}$. 
    This shows $sign : S_n \to (\{-1,1\},\times)$ is a group morphism. 
    It is easy to see this morphism is surjective. 
    Hence by the 1st isomorphism theorem and Lagrange's theorem, 
    the kernel of taking the sign must be a subgroup containing half of $S_n$. 
    That is to say, the set of even permutations on $n$ elements
    form a group of $n!/2$ elements.
    This is defined as the \textbf{alternating group on $n$-elements}, denoted $A_n$.
\end{dfn}

\end{document}