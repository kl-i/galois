\documentclass[../book.tex]{subfiles}
\begin{document}

%\begin{dfn} Permutations, Cycle Decomposition, Cycle Shape, Sign. 
%    
%    Let $X$ be a finite set. 
%    Then an element $\sigma$ of its automorphism group 
%    is called a \textbf{permutation} of $X$. 
%    In this case, the subgroup generated by $\sigma$ naturally acts on $X$. 
%    Let the orbits of $X$ under the $\<\sigma\>$-action be $s_1,\dots,s_n$. 
%    For each orbit $s_i$, 
%    by picking an element $a \in s_i$, 
%    one can see that, $\res{\sigma}{s_i}$, 
%    the restriction of $\sigma$ to $s_i$ \emph{cycles} the elements of $s_i$. 
%    Let $\sigma_i$ be the extension
%    of $\res{\sigma}{s_i}$ to the whole of $X$
%    by defining it as identity outside $s_i$. 
%    Then it is easy to see that we have \[
%        \sigma = \sigma_n \circ \cdots \circ \sigma_1
%    \]
%    This is called the \textbf{cycle decomposition of $\sigma$}
%    and uniquely determines the permutation $\sigma$. 
%    
%    The \textbf{cycle shape of $\sigma$} is defined as 
%    the list of the cardinalities 
%    of the orbits of $X$ under the $\<\sigma\>$-action in ascending sizes. 
%    \footnote{
%        The requirement of ascending is arbitrary. 
%        We are only interested in the list of numbers. 
%    }
%    The \textbf{sign of $\sigma$} is defined as \[
%        sign(\sigma) := (-1)^{\text{Number of orbits with even cardinality}}
%    \]
%    
%\end{dfn}

\begin{dfn} Symmetric Group, Permutations, 
Signs of Permutations, Alternating Group. 
    
    Let $n$ be a natural number. 
    Then the \textbf{symmetric group on $n$-elements} is defined as
    the automorphism group of any set of cardinality $n$. 
    It is denoted $S_n$ and has $n$ factorial many elements. 
    WLOG we can use the automorphism group of the set $\{i\}_{i<n}$.
    Then an element $\sigma$ of $S_n$ is called a \textbf{permutation}.
    Define the polynomial in $n$ indeterminates
    \[
        P(X_i)_{i<n} := \prod_{i<j<n} (X_i - X_j)
    \]
    Then the \textbf{sign} or \textbf{parity} of $\sigma$ is defined as \[
        sign(\sigma) := \frac{P(X_{\sigma(i)})_{i<n}}{P(X_i)_{i<n}}
    \]
    Since $P(X_i)_{i<n}$ and $P(X_{\sigma(i)})_{i<n}$ have the same factors,
    only differing in signs, $sign(\sigma)$ is $\pm 1$. 
    One can see that the sign of $\sigma$ counts
    the number of pairs $(i,j)$ such that $i < j$ and $\sigma(i) > \sigma(j)$. 
    Permutations of sign $+1$ are called \textbf{even} 
    and otherwise called \textbf{odd}.
    
    Let $\sigma, \rho$ be permutations in $S_n$. 
    Then we have \[
        sign(\rho \circ \sigma) = \frac{P(X_{\rho(\sigma(i))})_{i<n}}{P(X_i)_{i<n}}
        = \frac{P(X_{\rho(\sigma(i))})_{i<n}}{P(X_{\sigma(i)})_{i<n}}
        \frac{P(X_{\sigma(i)})_{i<n}}{P(X_i)_{i<n}} 
        = sign(\rho) sign(\sigma)
    \]
    where the second equality comes from $\sigma$ being simply
    a relabeling of the set $\{i\}_{i<n}$. 
    This shows $sign : S_n \to (\{-1,1\},\times)$ is a group morphism. 
    It is easy to see this morphism is surjective. 
    Hence by the 1st isomorphism theorem and Lagrange's theorem, 
    the kernel of taking the sign must be a subgroup containing half of $S_n$. 
    That is to say, the set of even permutations on $n$ elements
    form a group of $n!/2$ elements.
    This is defined as the \textbf{alternating group on $n$-elements}, denoted $A_n$.
    
\end{dfn}

\end{document}