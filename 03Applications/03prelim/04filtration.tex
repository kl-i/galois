\documentclass[../../book.tex]{subfiles}

\begin{document}
The goal of this section is to introduce \emph{solvability} of groups,
which plays an essential role in the insolvability of general quintics. 

% Def - Normal Series, Factor Groups, Length of Series
% Def - Refinement, Proper Refinement, Equivalence, Composition Series.
\begin{dfn} Normal Series, Factor Groups, Length 
and (Proper) Refinements of a Series, Equivalence of Series, Composition Series.
    
    Let $G$ be a group and 
    $S = (S_i)_{i\leq n}$ a finite sequence of subgroups.
    Then $S$ is called a \textbf{normal series} when the following conditions are met.
    \begin{enumerate}
        \item $S_0$ is the trivial subgroup and $S_n$ is $G$. 
        \item For all $i < j \leq n$, $S_i \subseteq S_j$. 
        \item For all $i < n$, $S_i \normsub S_{i+1}$.
    \end{enumerate}
    For a normal series $S$, 
    its \textbf{factor groups}, or simply \textbf{factors},
    are groups in the sequence of quotients $(S_{i+1}/S_i)_{i<n}$,
    which we shall denote with $Fac(S)$.
    
    The \textbf{length} of a normal series $S$ is
    the number of strict inclusions in $S$, which we shall denote with $|S|$.
    Equivalently, it's the number of non-trivial factor groups of $S$. 
    
    Let $S$ and $T$ be normal series of $G$. 
    Then $S \leq T$ is defined as $S$ being a subsequence of $T$. 
    We say $T$ is a \textbf{refinement of $S$}.
    It is easy to see that if $S \leq T$, then $|S| \leq |T|$.
    We say $T$ is a \textbf{proper refinement of $S$}
    when it is a refinement with strictly longer length. 
    
    For a normal series $S$, 
    the following are equivalent by the 3rd isomorphism theorem.
    \begin{enumerate}
        \item $S$ has no proper refinements.
        \item For a group, it is called \textbf{simple} when
        the only normal subgroups are itself and the trivial subgroup.
        Then the factor groups of $S$ are all simple. 
    \end{enumerate}
    If any of the above is true, $S$ is called a \textbf{composition series} of $G$. 
    
    For two normal series $S, T$ of $G$, 
    they are called \textbf{equivalent} when 
    there exists a bijection between $Fac(S)$ and $Fac(T)$
    that maps factors to isomorphic factors. 
    This unsurprisingly forms an equivalence relation 
    on the set of all normal series of a group $G$. 
    
\end{dfn}

\begin{thm} Existence of Composition Series for Finite Groups.
    
    Let $G$ be a finite group.
    Then it has a composition series.
    
\end{thm}
\begin{proof}
    
    Let $S$ be a normal series of $G$.
    Clearly the length of $S$ cannot be larger than the cardinality of $G$
    and $\<e\> \normsub G$ is a normal series of $G$.
    Hence the set of lengths of normal series of $G$ is a non-empty
    bounded subset of the naturals
    and thus must have a maximum. 
    Let $\hat{S}$ be a normal series of $G$ with maximal length. 
    It cannot have a proper refinement, so it must be a composition series.
    
\end{proof}

\begin{rmk}
    
    The motivation for considering normal series up to equivalence is that
    the data of a normal series is really in its factor groups.
    As we will now see, all finite groups have \emph{unique} composition series.
    But first, the theorem that you might have been wondering about.
    
\end{rmk}

% Thm - 2nd Iso.
\begin{thm} 2nd Isomorphism.
    
    Let $A, B$ be subgroups of a group $G$ such that
    $B$ is in the normaliser of $A$. 
    Define \[
        AB := \{ ab \mid a \in A, b \in B\}
    \]
    Then $AB$ is a subgroup of $G$ and we have the following isomorphism of groups. 
    \[
            \frac{AB}{A} \iso \frac{B}{A \cap B}
    \]
    
\end{thm}
\begin{proof}
    
    $AB$ being a subgroup and 
    the normality of $A \normsub AB$. $A \cap B \normsub B$
    is left as an exercise to the reader.
    
    As for the isomorphism, it is easily deduced from the 1st isomorphism theorem.
    
\end{proof}

% Lem - Butterfly / Zassenhaus.
\begin{lem} Butterfly / Zassenhaus.
    
    Let $G$ be a group with subgroups $A, A^*, B, B^*$ such that 
    $A^* \normsub A$ and $B^* \normsub B$. 
    Then \begin{itemize}
        \item $A^* (A \cap B^*) \normsub A^* (A \cap B)$
        \item $(A^* \cap B) B^* \normsub (A \cap B) B^*$
        \item \[
            \frac {A^* (A \cap B)}{A^* (A \cap B^*)}
            \iso \frac{A \cap B}{(A^* \cap B)(A \cap B^*)}
            \iso \frac{(A \cap B)B^*}{(A^*\cap B)B^*}
        \]
    \end{itemize}
\end{lem}
\begin{proof}
    
    Since the second isomorphism is entire analogous to the first,
    but we have two proofs of the first,
    we give different proofs for the two isomorphisms.
    Note that $A^* \normsub A$ and $B^* \normsub B$
    implies $A^* \cap B \normsub A \cap B$ and $A \cap B^* \normsub A \cap B$
    which in turn implies $(A^* \cap B)(A \cap B^*)$ is a subgroup
    by the 2nd isomorphism theorem,
    and it is normal in $A \cap B$. 
    
    (the first isomorphism)
    
    We first show that $A^* (A \cap B^*) \normsub A^* (A \cap B)$. 
    Let $a, \al \in A^*$, $b \in A \cap B$ and $\be \in A \cap B^*$. 
    Then \[
        ab \al\be (ab)^{-1} = ab \al\be b^{-1}a^{-1}
        = a (b\al b^{-1}) 
        [ (b\be b^{-1}) a^{-1} (b\be b^{-1})^{-1} ]
        (b\be b^{-1})
    \]
    where $a$, $b\al b^{-1}$ and the square bracket term are in $A^*$
    because $b\be b^{-1}$ is in $A \cap B^*$.
    Hence for all $x \in A^* (A \cap B)$, 
    $x (A^* (A \cap B^*)) x^{-1} \subseteq A^* (A \cap B^*)$,
    i.e. we have the normality. 
    
    We then invite the reader to stare intensely
    until they realise the following are true \[
        A^* (A \cap B) = A^* (A \cap B^*) (A \cap B) \,\,\,\,\,\,\,\,\,\,
        A \cap B \cap A^* (A \cap B^*) = (A^* \cap B)(A \cap B^*)
    \]
    and thus deduced from the the second isomorphism theorem, \[
        \frac {A^* (A \cap B)}{A^* (A \cap B^*)}
        = \frac {A^* (A \cap B^*)(A \cap B)}{A^* (A \cap B^*)}
        \iso \frac{A \cap B}{A \cap B \cap A^*(A \cap B^*) }
        = \frac{A \cap B}{(A^* \cap B)(A \cap B^*)}
    \]
    
    (the second isomorphism)
    
    We construct an explicit group morphism 
    from $(A \cap B)B^*$ to the quotient $(A \cap B) / (A^* \cap B)(A \cap B^*)$
    that is surjective and has $(A^* \cap B)B^*$ as the kernel,
    thus concluding the result from the 1st isomorphism theorem. 
    
    The group morphism is \[
        \phi : (A \cap B)B^* \to \frac{A \cap B}{(A^* \cap B)(A \cap B^*)},
        a b \mapsto a (A^* \cap B)(A \cap B^*)
    \]
    We must show this is well defined.
    Let $a, \al \in A \cap B$ and $b, \be \in B^*$ such that $ab = \al \be$. 
    Then $\al = a b \be^{-1}$ where $b \be^{-1} = a^{-1} \al \in A \cap B^*$
    which is a subset of $(A^* \cap B)(A \cap B^*)$.
    Hence $\al$ and $a$ are in the same coset of $(A^* \cap B)(A \cap B^*)$,
    i.e. the group morphism is well-defined as a function. 
    Furthmore, $a b \al \be = a \al (\al^{-1} b \al) \be$ 
    where $\al^{-1} b \al$ is in $B^*$ since $B^*$ is normal in $B$.
    Thus $a b \al \be = a \al (\al^{-1} b \al) \be$ is mapped to 
    $(a \al)(A^* \cap B)(A \cap B^*)$,
    proving the function to be a group morphism. 
    
    We leave surjectivity and $\ker \phi = (A^* \cap B) B^*$ 
    as an easy exercise for the reader to check. 
    
\end{proof}

% Thm - Schreier Refinement. 
\begin{thm} Schreier Refinement. 
    
    Let $S, T$ be normal series of a group $G$. 
    Then there exists normal series $\hat{S}, \hat{T}$ such that
    $S \leq \hat{S}$, $T \leq \hat{T}$ and $\hat{S}, \hat{T}$ equivalent. 
    
\end{thm}
\begin{proof}
    
    Let $S = (S_i)_{i \leq n}$ and $T = (T_j)_{j \leq m}$. 
    For $i < n, j \leq m$, define \[
        S_{i,j} := S_i (S_{i+1} \cap T_j)
    \]
    and for $i \leq n, j < m$, define \[
        T_{i,j} := (S_{i} \cap T_{j+1}) T_j
    \]
    Then by the Butterfly lemma, 
    $S_{i,j} \normsub S_{i,j+1}$ and $T_{i,j} \normsub T_{i+1,j}$
    with \[
        \frac{S_{i,j+1}}{S_{i,j}} \iso \frac{T_{i+1,j}}{T_{i,j}}
    \]
    Thus we have the following two normal series \begin{align*}
        \<e\> = S_{0,0} \normsub \cdots \normsub S_{0,m} 
        = S_{1,0} \normsub \cdots \normsub S_{1,m}
        = S_{2,0} \normsub \cdots \normsub S_{n-2,m}
        = S_{n-1,0} \normsub \cdots \normsub S_{n-1,m} = G
        \\
        \<e\> = T_{0,0} \normsub \cdots \normsub T_{n,0} 
        = T_{0,1} \normsub \cdots \normsub T_{n,1} 
        = T_{0,2}\normsub \cdots \normsub T_{n,m-2}
        = T_{0,m-1} \normsub \cdots \normsub T_{n,m-1} = G
    \end{align*}
    which are equivalent and are clearly refinements of $S, T$ respectively. 
    
\end{proof}

% Thm - Uniqueness and Every Normal Series has Composition Refinement.
\begin{thm} Uniqueness of Composition Series.
    
    Let $G$ be a group with a composition series $S$
    Then any other composition series $T$ is equivalent to $S$.
    Furthermore, any normal series $R$ has a refinement that is a composition series.
    
\end{thm}
\begin{proof}
    
    Easily deduced from the Schreier Refinement theorem.
    
\end{proof}

We arrive at our goal.

% Def - Solvability for Finite Groups, 3 Eqv.
\begin{dfn} Solvability of a Finite Group.
    
    Let $G$ be a finite group.
    Then the following are equivalent: \begin{enumerate}
        \item There exists a normal series of $G$ with abelian factor groups.
        \item There exists a normal series of $G$ 
        with factor groups of prime cardinality or one.
        \item There exists a normal series of $G$ with cyclic factor groups.
    \end{enumerate}
    If any of the above is true, we say $G$ is \textbf{solvable}. 
    
\end{dfn}
\begin{proof}
    
    ($1 \imp 2$)
        
        Let $S$ be a normal series of $G$ with abelian factor groups.
        Since $G$ is finite, it has a composition series,
        so there exists a refinement $\hat{S}$ of $S$ that is a composition series.
        It is not difficult to deduce from the 3rd isomorphism theorem
        that the factor groups of $\hat{S}$ are abelian, too.
        Suppose there is a non-trivial factor $\hat{S}_{i+1} / \hat{S}_i$ 
        with non-prime cardinality. 
        Then $|\hat{S}_{i+1} / \hat{S}_i|$ has a prime factor $p$ 
        strictly smaller than itself. 
        By Cauchy's theorem, 
        there then exists a subgroup $T$ of cardinality $p$ in the factor.
        Since the factor group is abelian, $T$ is normal in the factor group
        and from the 3rd isomorphism theorem we can deduce that \[
            S_i \normsub \pi^{-1} T \normsub S_{i+1}
        \]
        where $\pi^{-1} T$ is the preimage of $T$ under the projection to the factor.
        This gives a proper refinement of $\hat{S}$, 
        contradicting with it being a composition series. 
        Thus all factor groups are either trivial or have prime cardinality. 
        
    ($2 \imp 3$) Easily deduced from Lagrange's theorem.
    
    ($3 \imp 1$) Trivial.
\end{proof}

A useful result.
% Lem - Solvable iff Exists Solvable Factor.
\begin{lem} Solvable if and only if Exists Solvable Factor.
    
    Let $G$ be a group.
    If $G$ is solvable then for all normal subgroups $N$,
    $N$ and $G / N$ are solvable.
    On the other hand if there exists a normal subgroup $N$ such that 
    $N$ and $G / N$ are solvable,
    then $G$ is solvable.
    
\end{lem}
\begin{proof}
    
    Left as an exercise. 
    [Hint : The Schreier Refinement theorem
    and the 3rd isomorphism theorem are your friends.]
    
\end{proof}

\end{document}