\documentclass[../../book.tex]{subfiles}

\begin{document}

The Galois correspondence gives us a way of exchanging results
between the world of groups and the world of extensions. 
In the following preliminary sections, 
we develop some group-theoretic results which we will be using. 

\begin{dfn} Order of an Element.
    
    Let $G$ be a group. 
    For an element $g \in G$, the \textbf{order of $g$} is defined as 
    the smallest positive integer $o(g)$ such that $g^{o(g)} = e$. 
    If no such $o(g)$ exists, we say $g$ has \emph{infinite order}. 

\end{dfn}
    
\begin{thm} Basic Properties of the Order of an Element.
    
    Let $G$ be a group and $g$ an element of $G$ such that $o(g)$ exists.
    Then the following are true: \begin{enumerate}
        \item For all other positive integers $n$, 
        $g^n = e$ implies $o(g)$ divides $n$.
        \item $o(g)$ is the cardinality of the subgroup generated by $g$.
        \item If $G$ is finite, then $g^{|G|} = e$.
    \end{enumerate}
\end{thm}
\begin{proof}
    Left as exercises. 
\end{proof}

\begin{ex} Fermat's Little Theorem.
    
    Let $n$ be an integer. 
    Then the quotient ring $\Z / (n)$ has $n$ elements.
    For an integer $a$, let $\bar{a}$ denote its projection in $\Z / (n)$.
    Then $\bar{a} \in \{\bar{0}, \bar{1}, \dots, \bar{n-1}\}$. 
    Show that $\bar{a}$ is a unit in the ring $\Z / (n)$
    if and only if $a, n$ are coprime. 
    In particular, if $n = p$ is prime, then $\Z / (p)$ is a field
    and for all $\bar{a} \in \Z/(p)$, $\bar{a}^p = \bar{a}$. 
    
\end{ex}
    
\begin{dfn} Cyclic Group.
    
    Let $G$ be a group. 
    Then $G$ is called a \textbf{cyclic group} when 
    there exists an element $g \in G$ such that $\<g\> = G$,
    i.e. $G = \{ g^n \mid n \in \Z\}$.
    We say \textbf{$g$ generates $G$} or \textbf{$g$ is a generator of $G$}.
    
    For any natural $n$, $C_n$ denotes the \textbf{cyclic group of order $n$},
    which is anything isomorphic to a cyclic group of cardinality $n$. 
\end{dfn}

\begin{lem} Basic Results about Finite Cyclic Groups and a Number-Theoretic Result.
    
    Let $G$ be a finite cyclic group with a generator $g$. 
    Elements of $G$ are $g^k$ where $0 \leq k < |G|$.
    Then \begin{enumerate}
        \item For any positive integer $n$, 
        the \textbf{totient of $n$} is defined as the number of positive integers 
        less than $n$ and coprime to $n$, denoted $\phi(n)$. 
        
        Let $g^k$ be an element of $G$. Then $o(g) = |G| / (|G|,k)$.
        Hence, the number of generators of $|G|$ is $\phi(|G|)$.
        
        \item Let $H$ be a subgroup of $G$. Then $H$ is cyclic, too. 
        \item For all $d$ dividing $|G|$,
        there exists a unique subgroup $H_d$ of cardinality $d$. 
        \item Let $n = |G|$. Then \[
            n = \sum_{d \mid n} \phi(d)
        \]
    \end{enumerate}
\end{lem}
\begin{proof}
    \begin{enumerate}
        \item Clearly, $(g^k)^{|G|/(|G|,k)} = e$. 
        We now show it is minimal. 
        Let $(g^k)^n = e$. 
        Then $g^{kn} = e$ implies $|G|$ divides $kn$ since $g$'s order is $|G|$. 
        That is to say, there exists an integer $m$ such that $|G| m = k n$. 
        Dividing by $(|G|,k)$ on both sides gives \[
            \frac{|G|}{(|G|,k)} m = \frac{k}{(|G|,k)} n
        \]
        We leave it to the reader to show that 
        $|G| / (|G|,k)$ and $k / (|G|, k)$ are coprime, 
        and hence $|G| / (|G|,k)$ divides $n$. 
        In particular, it is smaller than or equal to $n$.
        
        It is then clear that $g^k$ is a generator if and only if $(|G|,k) = 1$.
        Hence number of generators is the totient of $|G|$. 
        
        \item 
        % If $H = \<e\>$, then $H$ is trivially cyclic.
        % So suppose there exists a non-identity element $g^k$ in $H$.
        % Since non-empty subsets of $\N$ have a minimal element, 
        % WLOG $k$ is the minimal positive integer such that $g^k$ is in $H$.
        % We leave it to the reader to check that for any other element $g^m$ in $H$,
        % $k$ divides $m$, and hence $g^k$ generates $H$. 
    
        Define the set of powers of elements in H as
        \[ P:=\{k \in \Z \mid g^k \in H\} \]
        It is easy to verify $P$ is an ideal.
        As $\Z$ is a principal ideal domain,
        we know that $P$ is generated by some $k \in \Z$.
        We leave it to the reader to show that $g^k$ is a generating element of $H$.
        
        \item By assumption, there exists a positive integer $n_d$ such that $d n_d = |G|$.
        We leave it to the reader to verify that $g^{n_d}$ has order $d$,
        hence the subgroup generated by it has cardinality $d$. 
        This shows existence. 
        
        Let $H$ be another subgroup of order $d$. 
        Then there exists a $g^{m_d}$ that generates $H$. 
        Then $g^{dm_d} = e$ implies $|G|$ divides $dm_d$. 
        It is easy to check that $n_d$ divides $m_d$.
        Hence $g^{n_d}$ is in $H_d$, i.e. all of $H$ is in $H_d$. 
        Both are sets of order $d$, so $H = H_d$.
        
        \item The orders of elements in $G$ give a partition of $G$,
        and since orders of elements must divide $|G| = n$, we have \[
            n = \sum_{d \mid n} |\{x \in G \mid o(x) = d\}|
        \]
        Let $d$ be one of such orders. 
        Then all elements of order $d$ generate subgroups of order $d$,
        so by part 3, they are generate the same cyclic subgroup $H_d$. 
        Generators of $H_d$ are clearly order $d$, 
        so elements are order $d$ if and only if they generate $H_d$.
        Since $H_d$ is finite cyclic, 
        the number of generators of $H_d$ is equal to $\phi(d)$. 
        This gives \[
            n = \sum_{d \mid n} |\{x \in G \mid o(x) = d\}|
            = \sum_{d \mid n} \phi(d)
        \]
    \end{enumerate}
\end{proof}

\begin{thm} A Condition for Cyclic.
    
    Let $G$ be a finite group such that for all $d$ dividing $|G|$, 
    the number of solutions $x^d = e$ is less than or equal to $d$. 
    Then $G$ is cyclic.
\end{thm}
\begin{proof}
    
    We again partition $G$ by the orders of elements. 
    Let $G_d$ be the set of elements in $G$ with order $d$. 
    Then we have \[
        |G| = \sum_{d \mid |G|} |G_d|
    \]
    We show that $|G_d| \leq \phi(d)$. 
    Note that $G_d$ is a subset of the elements that satisfy $x^d = e$. 
    Suppose $G_d$ is empty. Then we are done.
    Suppose $G_d$ is non-empty. 
    Then there exists a non-identity element $g$ with order $d$.
    Consider the subgroup $\<g\>$ generated by $g$. 
    All elements in $\<g\>$ satisfy $x^d = e$.
    So $\<g\>$ is a $d$-element subset of the solutions to $x^d = e$,
    which has at most $d$-elements.
    Hence $\<g\>$ \emph{is} the set of solutions to $x^d = e$.
    In particular, all other elements of order $d$ must be in $\<g\>$.
    Generators of $\<g\>$ are also order $d$, 
    and $\<g\>$ is finite cyclic, so we have $|G_d| = \phi(d)$. 
    
    We then have \[
        |G| = \sum_{d \mid |G|} |G_d| \leq \sum_{d \mid |G|} \phi(d) = |G|
    \]
    Thus $|G_d| = \phi(d)$ for all $d$ dividing $|G|$. 
    In particular, $|G_{|G|}| = \phi(|G|) \neq 0$, 
    so we have an element of order the cardinality of $G$,
    i.e. $G$ is cyclic.
\end{proof}


\end{document}