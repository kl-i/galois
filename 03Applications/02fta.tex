\documentclass[../book.tex]{subfiles}

\begin{document}

In this chapter, we give a proof of the fundamental theorem of algebra,
which states that all polynomials over $\C$ split in $\C$. 
To prove this, we need two lemmas first.
%        -  No non-trivial odd deg ext of R
\begin{lem}

    There does not exists non-trivial $\R$-extension $(L,\iota_L)$ 
    such that $[L : \R]$ is odd.
\end{lem}
\begin{proof}
    Assume such $L$ exist, pick $ \al \in L\setminus \iota_L\R$.
    By the tower law of extension degree, 
    it will have a minimal polynomial with odd degree. 
    By the intermediate value theorem, there exist real root $a$ in $\R$.
    Then by irreducibility, $\min(\al,\R) = (X - a)$, 
    i.e. $\al = \iota_L(a) \in \iota_L\R$.
    Since $\al$ was arbitrary, all of $L$ is actually in $\iota_L\R$.
    Hence, $L$ is a trivial extension, which is a contradiction. 
    
\end{proof}
%        -  No non-trivial even deg ext of C 
\begin{lem}

    There does not exists a $\C$-extension $(L,\iota_L)$ 
    such that $[L : \mathbb{C}]=2$.
\end{lem}
\begin{proof}
    Assume such $(L,\iota_L)$ exist and 
    pick $ \al \in L \setminus \mathbb{C}$. 
    It will have a minimal polynomial with degree 2. 
    However $\mathbb{C}$ must split any quadratic polynomial.
    So $\al$ is actually already inside $\iota_L \C$.
    Since $\al$ was arbitrary, $L$ is actually a trivial extension of $\C$,
    i.e. $[L : \C] = 1 \neq 2$, a contradiction. 
\end{proof}
\begin{thm} Fundamental Theorem of Algebra. 
    
    Let $f$ be a polynomial over $\C$.
    Then $f$ splits in $\C$.
    
\end{thm}
\begin{proof}
    
    Let $\iota_L : \C \to L$ be the splitting field of $f$. 
    We seek to show that this is a trivial extension.
    
    First, we note that it suffices to show $[L : \C]$ is a power of 2,
    for if it is, then Sylow's theorem in conjunction with the Galois correspondence
    gives a chain of extensions \[
        \C \iso L_0 \subseteq \cdots \subseteq L_n = L
    \]
    where $[L_{i+1} : L_i] = 2$.
    Then all these extensions must be trivial, 
    since there does not exists a $\C$-extension of degree 2. 
    
    Next, consider the extension $\iota_L : \R \to L$,
    where we viewed $\R$ as a subset of $\C$. 
    This may not be a normal extension, 
    so take the normal closure $\iota_N \circ \iota_L : \R \to N$ instead.
    The Galois correspondence then implies 
    the $\aut{\C}{L}$ is isomorphic to $\aut{\C}{N} / \aut{L}{N}$,
    where $\aut{\C}{N}$ is a subgroup of $\aut{\R}{N}$.
    It then suffices to show $\aut{\R}{N}$ has cardinality power of 2. 
    
    Suppose $[N : \R] = |\aut{\R}{N}| = k 2^n$ where 
    $2^n$ is the maximal power of 2 
    dividing the cardinality of the Galois group of $N$ over $\R$.
    This implies $k$ is odd.
    Then by Sylow's theorem, we have a chain of subgroups, \[
         \aut{N}{N} = H_0 \subseteq \cdots \subseteq H_n \subseteq \aut{\R}{N}
    \]
    where $H_i$ has cardinality $2^i$.
    By the Galois correspondence, we then have a chain of subextensions, \[
        \R \to N^{H_n} \subseteq \cdots \subseteq N^{H_0} = N
    \]
    In particular, $[N^{H_n} : \R] = [\aut{\R}{N} : H_n] = k$ which is odd.
    Since all $\R$-extensions of odd degree are trivial, 
    $k = 1$ and we are done. 
    
\end{proof}

\end{document}