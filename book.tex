\documentclass[oneside]{book}

\usepackage[left=1in,right=1in]{geometry}
\usepackage{amsmath, amsthm, amssymb, marvosym, verbatim, tikz, tikz-cd, float}
\usepackage{parskip}
\usepackage{mdframed}
\usepackage{perpage}
\usepackage{subfiles}
\usepackage{hyperref}
\usepackage{enumitem}

\MakePerPage{footnote}

%theorems
\newtheoremstyle{definitionstyle}
{20pt}% above thm
{10pt}% below thm
{}% body font
{}% space to indent
{\bf}% head font
{.}% punctuation between head and body
{ }% space after head
{\thmname{#1}\thmnumber{ #2}\thmnote{ (#3)}}

\newtheoremstyle{lemmastyle}%
{20pt}% above thm
{5pt}% below thm
{\it}% body font
{}% space to indent
{\bf}% head font
{.}% punctuation between head and body
{ }% space after head
{\thmname{#1}\thmnumber{ #2}\thmnote{ (#3)}}

\newtheoremstyle{exercisestyle}%
{10pt}% above thm
{5pt}% below thm
{\it}% body font
{}% space to indent
{\it}% head font
{.}% punctuation between head and body
{ }% space after head
{\thmname{#1}\thmnumber{ #2}\thmnote{ (#3)}}

\newtheoremstyle{claimstyle}%
{5pt}% above thm
{5pt}% below thm
{\it}% body font
{}% space to indent
{\it}% head font
{:}% punctuation between head and body
{ }% space after head
{\thmname{#1}\thmnumber{ #2}\thmnote{ (#3)}}

%\newtheoremstyle allows creating space before and after theorems
%[chapter] gives numberings according to chapter
%[thm] gives continuous numberings for all three types

\theoremstyle{lemmastyle}
\newtheorem{thm}{Theorem}[chapter]
\newtheorem{lem}[thm]{Lemma}
\newtheorem{cor}[thm]{Corollary}

\theoremstyle{definitionstyle}
\newtheorem{definition}[thm]{Definition}
\newtheorem{eg}[thm]{Example}
\newtheorem*{rmk}{Remark}

\theoremstyle{exercisestyle}
\newtheorem*{ex}{Exercise}

\theoremstyle{claimstyle}
\newtheorem*{clm}{Claim}

%tikzcd
\tikzset{
  symbol/.style={
    draw=none,
    every to/.append style={
      edge node={node [sloped, allow upside down, auto=false]{$#1$}}}
  }
}
% allows for substituting symbols for arrows in tikz comm-diagrams.

%commands
\mathchardef\mhyph="2D
%\newcommand{\ex}[1]{\textit{Exercise. #1}}
\newcommand\res[2]{{% we make the whole thing an ordinary symbol
  \left.\kern-\nulldelimiterspace % automatically resize the bar with \right
  #1 % the function
  \vphantom{\big|} % pretend it's a little taller at normal size
  \right|_{#2} % this is the delimiter
}}

\newcommand{\<}{\langle}
\renewcommand{\>}{\rangle}
\newcommand{\supp}{\mathrm{supp }}

\newcommand{\N}{\mathbb{N}}
\newcommand{\Z}{\mathbb{Z}}
\newcommand{\Q}{\mathbb{Q}}
\newcommand{\R}{\mathbb{R}}
\newcommand{\C}{\mathbb{C}}
\newcommand{\F}{\mathbb{F}}
\newcommand{\al}{\alpha}
\newcommand{\be}{\beta}
\newcommand{\ga}{\gamma}
\newcommand{\de}{\delta}
\newcommand{\ep}{\varepsilon}
\newcommand{\ka}{\kappa}
\newcommand{\la}{\lambda}
\newcommand{\om}{\omega}
\newcommand{\im}{\textnormal{im}\;}

\newcommand{\Hom}[3]{\mathrm{Hom}_{#1}(#2, #3)}
\newcommand{\mor}[3]{\mathrm{Mor}_{#1}(#2, #3)}
\newcommand{\End}[2]{\mathrm{End}_{#1}#2}
\newcommand{\aut}[2]{\mathrm{Aut}_{#1}#2}
\newcommand{\Set}{\mathbf{Set}}
\newcommand{\Top}{\mathbf{Top}}
\newcommand{\Grp}{\mathbf{Grp}}
\newcommand{\Ring}{\mathbf{Ring}}
\newcommand{\RMod}{R\mhyph\mathbf{Mod}}
\newcommand{\KVec}{K\mhyph\mathbf{Vec}}
\renewcommand{\mod}[1]{#1\mhyph\mathbf{Mod}}
\renewcommand{\vec}[1]{#1\mhyph\mathbf{Vec}}

\newcommand{\imp}{\Rightarrow}
\newcommand{\iso}{\cong}
\newcommand{\normsub}{\trianglelefteq}
\newcommand{\norm}[1]{\Vert#1\Vert}
\renewcommand{\bar}[1]{\overline{#1}}
\newcommand{\id}[1]{\mathrm{id}_{#1}}

\newcommand{\gal}[2]{\mathrm{Gal}_{#1}(#2)}
\newcommand{\Orb}{\mathrm{Orb}}
\newcommand{\Stab}{\mathrm{Stab}}

\renewcommand{\O}{\mathcal{O}}

\newcommand{\fa}{\mathfrak{a}}
\newcommand{\fb}{\mathfrak{b}}
\newcommand{\fp}{\mathfrak{p}}
\newcommand{\fq}{\mathfrak{q}}

\renewcommand{\thempfootnote}{\fnsymbol{mpfootnote}}
\renewcommand{\thefootnote}{\fnsymbol{footnote}}

\renewcommand{\tiny}[1]{\scalebox{0.5}{$#1$}}

\newcommand{\dolater}{\Coffeecup \hspace{3pt}}

\newcommand{\bigspace}{\mspace{12mu}}

%Boxes
\mdfdefinestyle{Definitions}{
    leftmargin=0cm,
    rightmargin=0cm,
    linecolor=gray!70,
    topline=false,
    bottomline=false,
    rightline=false,
    backgroundcolor=gray!4,
    footnoteinside=true}

\newenvironment{dfn}
    {\begin{mdframed}
        [style=Definitions, skipabove=1em, skipbelow=1em]
        \begin{definition}}
        {\end{definition}
    \end{mdframed}}




\begin{document}
\title{Notes on Galois Theory}
\author{Names}
\date{Summer 2019}
\maketitle

% Aim: Readable for First Year Math Student
% key word in def \emph{}
% Job 1: type up all thm + def assuming knowledge
% Job 2: Adding prelim ideas (groups, rings, iso thm)
% Job 3: the "talk" around the theory
\tableofcontents
\newpage

\part{Preface}
    \chapter{Markers}
        \subfile{01Preface/01markers}

\part{Theory}
    
    \chapter{Motivation}
        \subfile{02Theory/01motivation_alt}
        
    \chapter{Preliminaries}
        
        \section{Groups}
            \subfile{02Theory/02prelim/01groups}
        
        \section{Vector Spaces}
            \subfile{02Theory/02prelim/02vectorspaces}
        
        \section{Rings}
            \subfile{02Theory/02prelim/03rings}
        
        \section{Polynomial Rings over Fields}
            \subfile{02Theory/02prelim/04polynomial}
        
        \section{More on Vector Spaces : Dimension}
            \subfile{02Theory/02prelim/05morevector}

    \chapter{Field Extensions}
        \subfile{02Theory/03extensions}

    \chapter{Minimal Polynomials and Simple Extensions}
        \subfile{02Theory/04minimal}

    \chapter{Splitting Fields}
        \subfile{02Theory/05splitting}

    \chapter{Normal Extensions}
        \subfile{02Theory/06normal}

    \chapter{Seperable Extensions}
        \subfile{02Theory/07seperable_alt}
        
    \chapter{Primitive Element Theorem}
        \subfile{02Theory/08primitive}

    \chapter{Galois Correspondance}
        \subfile{02Theory/09galois_alt2}
        
\part{Applications}
    
    \chapter{Preliminaries}
        
        \section{Order of Elements and Cyclic Groups}
            \subfile{03Applications/03prelim/01cyclic}
        
        \section{Permutations}
            \subfile{03Applications/03prelim/02permute}
        
        \section{Sylow's Theorems}
            \subfile{03Applications/03prelim/03sylow}
            
        \section{Normal Series of Groups}
            \subfile{03Applications/03prelim/04filtration}
            
    \chapter{Finite Fields}
        \subfile{03Applications/01finite}
    
    \chapter{Fundamental Theorem of Algebra}
        \subfile{03Applications/02fta}
    
    \chapter{Impossible Constructions in Greek Geometry}
        \subfile{03Applications/04greeks}
        
    \chapter{Computing Galois Groups}
        \subfile{03Applications/03computing}
    
    \chapter{Cyclotomic Extensions}
        \subfile{03Applications/09cyclotomic}

    \chapter{Radical Extensions and Solvability}
        \subfile{03Applications/10radical}

    
\end{document}