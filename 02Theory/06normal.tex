\documentclass[../book.tex]{subfiles}

\begin{document}
%    - Def - Normal Ext, 5 Eqv def (Except 4 - > 2)
\begin{dfn} Normal Extensions. 
    Let $\iota : K \to L$ be a finite extension. 
    Then the following are equivalent: 
    \begin{enumerate}
        \item For all irreducible polynomials $f \in K[X]$ that have root in $L$,
        $L$ splits $f$. 
        \item For all elements $a \in L$, $L$ splits $\min(a,K)$. 
        \item There exists generators $a_1, \dots, a_n \in L$ such that
        for all generators $a_i$, there exists a polynomial $f_i \in K[X]$
        where $a_i$ is a root and $L$ splits $f_i$.
        \item There exists some polynomial $f \in K[X]$ such that 
        $L$ is the splitting field of $f$. 
        \item For all $K$-extension $\iota_M : K \to M$ and 
        $K$-extension morphisms $\phi, \psi : L \to M$, $\phi L = \psi L$. 
    \end{enumerate}
    If any of the above are true, 
    $\iota : K \to L$ is called a \textbf{normal extension}.
    We also sometimes say \textbf{$L$ is normal over $K$}. 
\end{dfn}
\begin{proof}
    ($1 \imp 2$) 
        Since finite extensions are algebraic, the second statement is well-defined.
        By minimal polynomials are irreducible and have a root in $L$, we are done. 
        
    ($2 \imp 3$)
        $L$ is a finite extension so 
        there exists $a_0,\dots,a_{n-1} \in L$ such that $L = K(a_0,\dots,a_{n-1})$.
        For each generator $a_i$, pick $f_i = \min(a,K)$. 
    
    ($3 \imp 4$)
        Let $f = \prod_{i\in n} f_i$. 
        $L$ clearly splits $f$ and 
        since the roots of $f$ include the generators $a_0,\dots,a_{n-1}$,
        $L$ is generated by the roots of $f$. 
        
    ($4 \imp 1$) 
        Let $g$ be an irreducible polynomial over $K$ with a root $a \in L$. 
        We do not know whether $g$ splits in $L$, 
        but we know there exists a finite extension $\iota_M : L \to M$
        that splits $\bar\iota g$, i.e. contains all the roots of $g$. 
        We will show in that in this larger field, 
        all the roots of $g$ are already in $L$. 
        
        Let $a_i$ be any root of $g$ in $M$.  
        The proof is contained in the following diagram. 
        \begin{figure} [H]
            \centering
            \begin{tikzcd}
            & 
            K(a) \arrow[r,"\subseteq"] &
            L \arrow[r,"\iota_M"] &
            \iota_M L \arrow[rd,"\subseteq"{sloped,above}] 
            \arrow[dd,"\subseteq"] &
            \\
            K \arrow[ru,"\iota"{sloped,above}] 
            \arrow[rd,"\iota_M \circ \iota"{sloped,below}] &
            & 
            & 
            & 
            M \\
            & 
            K(a_i) \arrow[rr,"\subseteq"] \arrow[uu,"\phi",dashed] & 
            & 
            L(a_i) \arrow[ru,"\subseteq"{sloped,below}] 
            \arrow[luu,"\bar\phi",dashed] &
            \end{tikzcd}
        \end{figure}
        We will show that $\iota_M L = L(a_i)$, i.e. the root $a_i$ is already in $L$.
        Note that by $\iota_M : L \to L(a_i)$ being 
        an injective $K$-vector space morphism, 
        it suffices to show $\dim_K L = \dim_K L(a_i)$.
        
        We already have $\dim_K L \leq \dim_K L(a_i)$.
        For the other inequality, note that by definition, 
        $L$ splits $f$ and $L$ is generated by the roots of $f$. 
        This clearly implies $L(a_i)$ splits $\overline{\iota_M \circ \iota} f$
        and is generated by the roots of $\overline{\iota_M \circ \iota} f$, i.e.
        $L(a_i)$ is the splitting field of $\overline{\iota_M\circ\iota} f$.
        Since $a$ and $a_i$ are galois conjugates, 
        there exists a $K$-extension morphism $\phi : K(a_i) \to K(a), a_i \mapsto a$.
        So then $\phi : K(a_i) \to L$ is a $K(a_i)$-extension that splits
        $\overline{\iota_M\circ\iota} f$. 
        Hence by the minimal property of splitting fields, 
        there exists a $K(a_i)$-extension morphism $\bar\phi : L(a_i) \to L$. 
        This is clearly a $K$-extension morphism. 
        In particular, $\bar\phi$ is an injective $K$-vector space morphism
        between finite dimensional $K$-vector spaces $L(a_i)$ and $L$. 
        Hence $\dim_K L(a_i) \leq \dim_K L$. 
        
    ($4 \imp 5$)
        This is image invariance of splitting fields.
    
    ($5 \imp 1$)
        We require the machinery of \emph{normal closures}, which we will investigate below.
\end{proof}

For now, we restrict the definition of normality to 1 to 4. 

\begin{eg} Non-Normal Extensions. 
    
    The $\Q$-extension $\iota : \Q \to \Q(\sqrt[3]{2})$ is \emph{not} normal, 
    since \[
        \bar\iota\min(\sqrt[3]{2},\Q) = X^3 - 2 
        = (X - \sqrt[3]{2})(X^2 + \sqrt[3]{2}X + \sqrt[3]{2}^2)
    \]
    does not factorise further. 
    In a sense, the other roots $\sqrt[3]{2}\omega, \sqrt[3]{2}\omega^2$ 
    are "missing" from the field $\Q(\sqrt[3]{2})$. 
    
    This can be fixed by adding elements to the extension until 
    the extension is normal.
    The "smallest" such extension will be the \emph{normal closure}.
\end{eg}
%    - Lem - Normality Inheritance
\begin{lem} Normality Lifts up. 

    Let $K \overset{\iota_L}{\to} L \overset{\iota_M}{\to} M$ be extensions
    and $f$ a polynomial over $K$. 
    Then $\iota_M\circ\iota_L : K \to M$ is the splitting field of $f$
    implies $\iota_M : L \to M$ is the splitting field of $\bar\iota_L f$. 
    Consequently, $\iota_M\circ\iota_L : K \to M$ normal 
    implies $\iota_M : L \to M$ normal. 
    
\end{lem}
\begin{proof}
    Suppose $\iota_M\circ\iota_L : K \to M$ is the splitting field of $f$. 
    Clearly, $M$ splits $\bar\iota_L f$. 
    Let $S_f$ be the roots of $f$ in $M$. 
    $S_f$ is also the roots of $\bar\iota_L f$ in $M$. 
    Then the smallest subfield of $M$ containing $\iota_M L$ and $S_f$ 
    also contains $\iota_M(\iota_L K)$, 
    i.e. $M = K(S_f) \subseteq L(S_f) \subseteq M$.
    Hence $M$ is generated by $S_f$ as an $L$-extension. 
\end{proof}
%    - Def - Normal Closure
\begin{dfn} Normal Closure. 
    
    Let $K \overset{\iota_L}{\to} L \to \overset{\iota_N}{\to} N$ be extensions. 
    Then $(N,\iota_N)$ is called 
    a \textbf{normal closure of $(L,\iota_L)$} when
    it is \emph{a smallest} normal $K$-extension containing $\iota_L : K \to L$
    in the sense that $\iota_N\circ\iota_L : K \to N$ is normal and 
    for any $\iota_M : L \to M$ such that $\iota_M\circ\iota_L : K \to M$ is normal,
    there exists a $L$-extension morphism $\bar\iota_M : N \to M$.
    Diagrammatically, 
    \begin{figure} [H]
        \centering
        \begin{tikzcd}
        & 
        & 
        N \arrow[dd,"\bar\iota_M"] \\
        K \arrow[r,"\iota_L"] &
        L \arrow[ru,"\iota_N"{sloped,above}] \arrow[rd,"\iota_M"{sloped,above}] &
        \\
        & 
        &
        M
        \end{tikzcd}
    \end{figure}
    
    Note that we have not proved normal closures to be unique yet,
    hence the emphasis on "a normal closure" not "the normal closure". 
\end{dfn}
%    - Thm - Min Prop of Normal Closure (Changed to Existence & Uniqueness)
\begin{thm} Existence and Uniqueness of Normal Closure of Finite Extensions.
    
    Let $\iota_L : K \to L$ be a finite $K$-extension. 
    Then there exists an $L$-extension $\iota_N : L \to N$ such that
    $(N,\iota_N)$ is a normal closure of $(L,\iota_L)$.
    Furthermore, $N$ is a finite extension and unique up to isomorphism, i.e.
    any other normal closure of $\iota_L : K \to L$ is
    isomorphic to $N$ as a $K$-extension. 
    Thus we denote $N$ as $N(L/K)$ and 
    refer to anything isomorphic to it as \emph{the} normal closure of $(L,\iota)$.
\end{thm}
\begin{proof}
    Since $L$ is a finite dimensional $K$-vector space,
    by existence of a basis,
    let $a_0,\dots,a_{n-1}$ be a finite set of generators of $L$,
    i.e. $L = K(a_0,\dots,a_{n-1})$.
    Let $f = \prod_{i\in n} \min(a_i,K)$. 
    There exists a $K$-extension $\iota_{\tilde{N}} : K \to \tilde{N}$ 
    that splits $f$. 
    Let $N$ be the $K$-subextension of $\tilde{N}$ generated by the roots of $f$, 
    i.e. the splitting field of $f$. 
    Since for all generators $a_i$ of L, $N$ splits $\min(a_i,K)$,
    we have a $K$-extension $\iota_N : L \to N$ by embedding via conjugates.
    Clearly, $\iota_N\circ\iota_L : K \to N$ is normal. 
    We will now show that it has the minimal property of normal closures. 
    
    Let $\iota_M : L \to M$ be an $L$-extension such that 
    $\iota_M\circ\iota_L : K \to M$ is normal. 
    By lifting normality of $N$, 
    $\iota_N : L \to N$ is the splitting field of $\bar\iota_L f$. 
    For a generator $a_i$ of $L$, $\min(a_i,K) = \min(\iota_M(a_i),K)$. 
    So normality of $M$ over $K$ implies $M$ splits $f$,
    consequently splitting $\bar\iota_L f$. 
    Hence by the minimal property of splitting fields,
    there exists an $L$-extension morphism $\bar\iota_M : N \to M$
    such that $\bar\iota_M\circ\iota_N = \iota_M$. 
    Thus, $(N,\iota_N)$ is a normal closure of $\iota_L : K \to L$.
    
    Now let $(M,\iota_M)$ be another normal closure of $(L,\iota_L)$. 
    Then by applying minimal property of normal closure twice,
    we have $L$-extension morphisms $\bar\iota_M : N \to M$ 
    and $\bar\iota_N : M \to N$.
    These are also $K$-extension morphisms.
    Since $N$ is a finite $K$-extension and 
    $\bar\iota_N$ is an injective morphism of $K$-vector spaces from $M$ to $N$,
    we have the dimension of $M$ as finite and less than equal to that of $N$. 
    Similarly, the dimension of $N$ is less than equal to that of $M$.
    So $\dim_K N = \dim_K M$ and hence $\bar\iota_M : N \to M$ is actually a bijection.
    Thus $N$ and $M$ are isomorphic as $L$-extensions. 
\end{proof}
%    - Thm - L is “in” N(L/K)
\begin{rmk}
    The following is the main property of normal extensions 
\end{rmk}
\begin{thm}
    
    Let $K \overset{\iota_L}{\to} L \overset{\iota_N}{\to} N$ be extensions
    where $(N,\iota_N\circ\iota_L)$ is normal. 
    Then for all $K$-extension morphisms 
    $f : (L,\iota_L) \to (N,\iota_N\circ\iota_L)$,
    there exists a $L$-extension morphism $\bar{f} : (N,\iota_N) \to (N,f)$.
\end{thm}

%    - Thm - Embeddings into Normal Ext give Automorphisms
%\begin{thm} Normal Extensions Differ by Extension Automorphisms. 
    
%    Let $\iota_N : L \to N$ be a normal $K$-extension. 
%    Then for any other normal $K$-extension to $N$, $f : K \to N$, 
%    there exists a $K$-extension automorphism $\bar{f} : (N,\iota_N) \to (N,f)$.
%    Diagrammatically, 
%    \begin{figure} [H]
%        \centering
%        \begin{tikzcd}
%        K \arrow[r,"\iota_N"{sloped,above},yshift=0.5ex] 
%        \arrow[r,"f"{sloped,below},yshift=-0.5ex] & 
%        N \arrow[loop right,"\bar f"]
%        \end{tikzcd}
%    \end{figure}
%\end{thm}
%\begin{proof}
%    $(N,\iota_N)$ normal implies there is a finite set of generators $a_1,\dots,a_n$
%    such that for each $a_i$, $(N,\iota_N)$ splits $\min(a_i,K)$.
%    But then by normality of $(N,f)$, $(N,f)$ splits $\min(a_i,K)$.
%    FALSE. $\bar\iota\min(a_i,K) \neq \bar{f}\min(a_i,K)$. 
%    Hence by embedding via conjugates, we have a $K$-extension morphism
%    $\bar{f} : (N,\iota_N) \to (N,f)$. 
    
%    The fact that it bijects follows from injective $K$-vector space morphism
%    between finite dimensional vector spaces are bijective. 
%\end{proof}

We are now ready to prove $5 \imp 1$. 
%    - Thm - Image Invariance -> Splits irr with root
\begin{thm} Image Invariance gives Splitting Irreducible Polynomials with Roots.
    
    Let $\iota_L : K \to L$ be a $K$-extension such that
    for all $K$-extensions $\iota_M : K \to M$ and 
    $K$-extension morphisms $\phi, \psi : L \to M$, $\phi L = \psi L$. 
    Then for all irreducible polynomials $f \in K[X]$ that have a root $a \in L$, 
    $L$ splits $f$. 
\end{thm}
\begin{proof}
    Let $f$ ba an irreducible polynomial over $K$ with a root $a \in L$. 
    We do not know whether $L$ contains all the roots of $f$, 
    but we do know that $L$'s normal closure definitely does. 
    So let $\iota_N : L \to N(L/K)$ be the normal closure of $L$. 
    Then $f$ is an irreducible polynomial with $\iota_N(a)$ as a root.
    So by normality, $N$ splits $f$. 
    Let $a_i$ be an arbitrary root of $f$ in $N$. 
    We seek to show that the root $a_i$ is already in $\iota_N L$. 
    The following diagram is the situation. 
    \begin{figure} [H]
        \centering
        \begin{tikzcd}
        K \arrow[r,"\iota_L"] \arrow[rd,"\iota_N\circ\iota_L"{sloped,below}] & 
        K(a) \arrow[r,"\subseteq"] \arrow[d,"\phi",dashed] & 
        L \arrow[r,"\iota_N"] & 
        N \arrow[loop right,"\bar\phi",dashed] \\
        & 
        K(a_i) \arrow[rru,"\subseteq"{sloped,below}] & 
        & 
        \end{tikzcd}
    \end{figure}
    Since $a_i$ is a galois conjugate of $a$, 
    there exists a $K$-extension morphism 
    $\phi : (K(a),\iota_L) \to (K(a_i),\iota_N\circ\iota_L)$ mapping $a \mapsto a_i$. 
    This implies $\phi\circ\iota_L = \iota_N\circ\iota_L$, i.e.
    $(N,\phi\circ\iota_L) = (N,\iota_N\circ\iota_L)$ as $K$-extensions.
    By lifting normality, $(N,\iota_N)$ and $(N,\phi)$ are both normal $K(a)$-extensions.
    Thus, there exists a $K(a)$-extension morphism $\bar\phi : (N,\iota_N) \to (N,\phi)$
    via finite normal extensions differing by automorphisms. 
    It follows that $\bar\phi\circ\iota_N : (L,\iota_L) \to (N,\iota_N\circ\iota_L)$
    is a $K$-extension morphism. 
    Since $\iota_N$ and $\bar\phi\circ\iota_N$ are 
    both $K$-extension morphisms from $L$ to $N$,
    $\iota_N L = \bar\phi(\iota_N L)$.
    But of course, 
    \[ a_i = \phi(a) = \bar\phi(\iota_N(a)) \in \iota_N L\]
    Thus, the root $a_i$ is actually already in $\iota_N L$.
    This completes the proof. 
\end{proof}
\end{document}