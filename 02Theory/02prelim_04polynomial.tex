\documentclass[../book.tex]{subfiles}
\begin{document}
%    - Def - Units of a Comm Ring, Fields (3 eqv def)
\begin{dfn} Fields (Upgraded Definitions).

    Let $K$ be a non-trivial commutative ring. 
    For $x \in K$, 
    \[
        x \in K^\times := \exists y \in K, xy = 1. 
    \]
    $x^{-1} := y$ is unique for fixed $x$. 
    Then $(K^\times, \cdot)$ is an abelian group. 
    Elements of $K^\times$ are called \textbf{units}. 

    $K$ is a \emph{field} if any of the following equivalent conditions are true:
    \begin{enumerate}
        \item $\forall x \in K, x = 0 \lor x \in K^\times$
        \item $\forall \fa \normsub_\Ring K, \fa = (0) \lor \fa = K$
        \item $\forall f : K \to R$ in $\Ring$, $f$ injective $\lor f = 0$
    \end{enumerate}
\end{dfn}
\begin{proof} 
    $(1 \Rightarrow  2)$
    Suppose $\fa \neq (0) \normsub K$ is an ideal, $a \in K$. 
    Then $\exists x \in \fa, x \neq 0$.
    So $1 = x^{-1} x \in \fa$. 
    Hence $a = a\cdot1 \in \fa$. 

    $(2 \Rightarrow 3)$ 
    Let $f : K \to R$ be a morphism of rings. 
    Then $\ker f$ is an ideal of $K$. 
    So if $R$ non-zero, $f(1) = 1 \neq 0$ i.e. $\ker f \neq K$. 
    Thus, $\ker f = (0)$ which implies $f$ injective.  
    If $R$ is the zero ring, then $f = 0$. 

    $(3 \Rightarrow 1)$ 
    Suppose $x \in K$. 
    Then we have the natural morphism of rings $\pi : K \to K/(x)$. 
    If $\pi = 0$, then $(x) = K$ i.e. $\exists\,y \in K, 1 = yx$,
    so $x \in K^\times$. 
    Otherwise, $\pi$ injective implies $(x) = (0)$, hence x = 0. 
\end{proof}

\begin{ex} [Important] Fields are Integral Domains.
    
    Let $K$ be a field. 
    Show that for arbitrary $a, b \in K$, $ab = 0$ implies $a = 0$ or $b = 0$.
    
    In general, rings with the above property are called \textbf{integral domains}. Although in general integral domains are not fields, but one can prove that \textbf{finite} integral domain are field. (Prove this!)
\end{ex}

%    - Def - K[x], deg on non-zero poly
\begin{dfn} Polynomial Ring over a Field, Degree of Non-Zero Polynomials. 

    Let $K$ be a field. 
    Let $K[X] := \<\N\>$ be the free $K$-vector space generated by $\N$.\footnote{
    A free vector space is constructed in exactly the same way as a free module.} 
    For $n \in \N$, let $X^n$ denote the image of $n$ 
    under the natural injection $\N \to \<\N\>$ 
    defined in the free modules exercise. 
    For $0 \in \N$, $1 := X^0$. 
    Then an arbitrary element $f \in K[X]$ looks like \[
        f = \sum_{k \in \N} f_k X^k
    \]
    where $f_k$ are scalars in $K$ and
    since only finitely many $f_k$ are non-zero,
    there exists $N \in \N$ such that for all $k$ after $N$, $f_k = 0$. 
    For $X^n, X^m \in K[X]$, we define multiplication via the addition of $\N$,
    i.e. \[X^n \cdot X^m := X^{n + m}\]
    This extends naturally to all of $K[X]$, \[
    fg = \left(\sum_{k \in \N} f_k X^k\right)\left(\sum_{l \in \N} g_l X^l\right) 
    = \sum_{n \in \N} \left(\sum_{k + l = n} f_k g_l\right) X^n
    \]
    giving $K[X]$ a ring structure.
    
    $K[X]$ is called the \textbf{polynomial ring over $K$}.
    Elements of $K[X]$ are called \textbf{polynomials over $K$}. 
    Note that we have a natural injection $K \to K[X]$ via $k \mapsto k\cdot 1$,
    which we shall denote with just $k$. 
    These are called \textbf{constants}. 
    
    Note that the above construction also works for rings, 
    giving polynomial rings over rings.
    But for Galois theory, we shall focus on those over fields. 
    
    For a \emph{non-zero} polynomial $f = \sum_{k \in \N} f_k X^k\in K[X]$, 
    the \textbf{degree of $f$} is defined as 
    the maximum $k$ such that $f_k \neq 0$ and is denoted $\deg f$. 
    This is well-defined since $f$ is a finite sum of non-zero terms. 
    If $n$ is the degree of $f$, then $f_n$ is called 
    \textbf{leading coefficient of $f$}. 
    In particular, polynomials with leading coefficient 1 are called \textbf{monic}.
\end{dfn}

\begin{ex} [Important] Polynomial Rings over a Field are Integral Domains. 
    
    Let $K$ be a field. 
    Show that for non-zero $f, g \in K[X]$, $fg \neq 0$,
    i.e. $K[X]$ is an integral domain. 
\end{ex}

%    - Lem - deg fg= deg f + deg g
\begin{lem} Degree of Product is Sum of Degree. 
    
    Let $K$ be a field and $f, g$ non-zero polynomials in $K[X]$. 
    Then \[\deg fg = \deg f + \deg g\]
\end{lem}
\begin{proof}
    The claim is well-defined since $K[X]$ is an integral domain.
    We proceed by double strong induction on the degrees of $f$ and $g$. 
    Suppose the theorem is true for $\deg f < n$ and $\deg g < m$. 
    Let $\deg f = n$, $\deg g = m$. 
    Then $f = \phi + f_n X^n$ where $\deg \phi < n$ and 
    $g = \psi + g_m X^m$ where $\deg \psi < m$. 
    So \[
    fg = \phi \psi + \phi g_m X^m + \psi f_n X^n + f_n g_m X^{n + m}
    \]
    Clearly, when one multiplies by a single term as $g_m X^m, f_n X^n$, 
    $\deg \phi g_m X^m = \deg \phi + m$ and $\deg \psi f_n X^n = \deg \psi + n$.
    By assumption, $\deg \phi\psi = \deg \phi + \deg \psi$.
    Then the degree of the first three terms are all strictly less than $n + m$.
    The last term is non-zero since $K$ is integral means $f_n, g_m \neq 0$
    implies $f_n g_m \neq 0$. 
    So the last term has highest degree and hence
    $\deg fg = \deg f_n g_m X^{n+m} = n + m$. 
\end{proof}
%    - Ex  - deg 0 := -\infty, Why? 
\begin{ex}
    There are sources that define $\deg 0 := -\infty$ where
    $0$ is the zero polynomial in $K[X]$.
    Justify that this convention intuitively works using the result of the previous lemma.
\end{ex}
\begin{ex} Equivalent Conditions for Units in Polynomial Rings over a Field.

    Let $f$ be a non-zero polynomial over a field $K$. 
    Show that the following are equivalent: \begin{enumerate}
        \item $f$ is a unit.
        \item $\deg f = 0$
        \item $f$ is a constant, i.e. $f = k$ where $k \in K$. 
    \end{enumerate}
\end{ex}
%    - Lem - Div Algorithm
\begin{lem} Division Algorithm for Polynomials.
    
    Let $K$ be a field and $f, g$ polynomials $\in K[X]$, $g$ non-zero.
    Then there exists unique polynomials $q, r \in K[X]$ such that \[
        f = q g + r \;\;\emph{ and }\;\; \deg r < \deg g
    \]
\end{lem}
\begin{proof}
    We proceed by strong induction on the degree of $f$. 
    Assume the theorem is true for $deg f < n$.
    Let $f = \sum_{k = 0}^{n} f_k X^k$ where $n$ is the degree of $f$
    and $g = \sum_{l = 0}^m g_l X^l$ where $m$ is the degree of $g$. 
    If $n < m$, we are done by choosing $q = 0$ and $r = f$.
    So suppose $m \leq n$. 
    Then \[
        \deg (f - (f_n/g_m)X^{n-m} g) < n
    \]
    So by assumption, there exists unique $q_0, r_0$ polynomials such that 
    \[f - (f_n/g_m)X^{n-m} g = q_0 g + r_0\] and $\deg r_0 < \deg g$. 
    Choosing $q = q_0 + (f_n/g_m)X^{n-m}$ and $r = r_0$ proves existence.
    Let $f = q_1 g + r_1$ where $\deg r_1 < \deg g$. 
    Then $f - (f_n/g_m)X^{n-m}g = (q_1 - (f_n/g_m)X^{n-m}) g + r_1$.
    Hence, $q_1 = q_0 + (f_n/g_m)X^{n-m} = q$ 
    and $r_1 = r_0 = r$ by uniqueness of $q_0, r_0$,
    which completes the strong induction. 
\end{proof}

%    - Cor - Ideals in K[x] are generated by 1 element. (K[x] PID)
\begin{ex} Polynomial Rings over Fields are Principal Ideal Domains. 
    
    Let $K$ be a field and $(0) \neq \fa \normsub K[X]$ be a non-zero ideal 
    of the polynomial ring. 
    Show that there exists a unique monic polynomial that generates $\fa$
    and it is of minimal degree in $\fa$. 
    (Hint : Use the division algorithm to derive a contradiction.)
    
    In general, rings where all ideals are generated by one element
    are called \textbf{principal ideal domains} or \textbf{PID} for short.
\end{ex}
%    - Def - Divides
\begin{dfn} Divides.
    
    Let $K$ be a field and $f, g \in K[X]$.
    Then \textbf{$g$ divides $f$} if there exists a polynomial $h$
    such that $f = hg$. 
    Equivalently, $(f) \subseteq (g)$,
    the ideal generated by $g$ contains the ideal generated $f$.
    This is denoted $g \mid f$. 
\end{dfn}

\begin{ex} Degree respects Divides.
    Let $g \mid f$, both non-zero polynomials. Show that $\deg g \leq \deg f$. 
\end{ex}
%    - Ex  - Div Lin Comb
%    - Def - GCD (Monic)
\begin{dfn} Greatest Common Divisor (GCD for short). 

    Let $K$ be a field, $f, g, h$ polynomials over $K$. 
    $h$ is called the \textbf{gcd of $f$ and $g$} when both of the following are true:
    \begin{enumerate}
        \item $h \mid f$ and $h \mid g$
        \item For all other polynomials $p$, 
        $p \mid f$ and $p \mid g$ implies $p \mid h$.
    \end{enumerate}
    Note that $0$ being gcd implies both $f, g = 0$,
    so this is sort of a degenerate case. 
    Otherwise if $h \neq 0$, then WLOG $h$ is monic.
\end{dfn}
%    - Thm - Exists unique GCD via Eucl Alg 
\begin{thm} Existence and Uniqueness of GCD via Euclidean Algorithm. 

    Let $K$ be a field and $f, g$ polynomials over $K$. 
    Then there exists a unique polynomial $(f,g)$ that is the gcd of $f, g$
    up to scaling by units of $K$ 
    and there exists polynomials $\al, \be$ such that $\al f + \be g = (f, g)$.
    
    In general, we pick $(f,g)$ to be zero or monic.
\end{thm}
\begin{proof}
    We first prove uniqueness. 
    Let $h_1, h_2$ both be gcds of $f, g$. 
    Then by definition, $h_1 \mid h_2$ and $h_2 \mid h_1$, i.e.
    there exists $\al, \be$ polynomials such that 
    $h_1 = \al h_2$ and $h_2=\be h_1$.
    If one of $h_1, h_2$ are zero, then both are zero and hence equal.
    Now suppose $h_1 \neq 0$ and $h_2 \neq 0$. 
    Then $h_1 = \al h_2 = \al \be h_1$ which implies $1 = \al \be$ 
    by $h_1 \neq 0$ and $K[X]$ being an integral domain. 
    Hence $\al, \be \in K^\times$, i.e. $h_1, h_2$ differ by scaling by units.
    
    We now prove existence via the euclidean algorithm.
    Note that if either $f, g$ are zero, then the other is the gcd, 
    because all polynomials divide $0$. 
    So suppose both $f, g$ are non-zero. WLOG $\deg g \leq \deg f$.
    For $n \in \N$, define (until $n+2 = 0$) $r_n$ such that\begin{align*}
        r_0 &= g \\
        f &= q_0 r_0 + r_1 \\
        r_n &= q_n r_{n+1} + r_{n+2}
    \end{align*}
    where the second and third line are as in the division algorithm. 
    Since the $\deg r_n$ are strictly monotonically decreasing and
    bounded above by $\deg r_0$, it must be a finite set of naturals.
    So the algorithm terminates. 
    Let $N$ be such that $\deg r_N$ is the minimum of the aforementioned set. 
    We leave it as an exercise to the reader to check that
    $r_N$ is the gcd of $f, g$.
    The polynomials $\al, \be$ such that $\al f + \be g = r_N$ 
    can be found via back-substituting the equations from the algorithm. 
\end{proof}
%    - Ex - Deduce exists/unique GCD from K[x] PID
\begin{ex} Alternative Proof of Existence and Uniqueness of GCD via PID.

    Let $f, g$ be polynomials over a field $K$.
    Show that for a polynomial $h \in K[X]$, 
    being the gcd of $f, g$ is equivalent to 
    $h$ generating the ideal generated by $\{f, g\}$, i.e. $(h) = (f, g)$.
    Hence show the existence and uniqueness of the gcd from $K[X]$ being a PID.
\end{ex}
%    - Def - Irr, Coprime, Prime
\begin{dfn} Irreducible Elements, Coprime Elements, Prime Elements.
    
    Let $f, g$ be a polynomials over a field $K$. 
    Then \begin{itemize}
        \item $f$ is called \textbf{irreducible} when it is non-zero, non-unit, and
            for all $a, b \in K[X]$, $f = a b$ implies $a$ unit or $b$ unit.
        \item $f, g$ are called \textbf{coprime} when their gcd $(f,g) = 1$.
            Equivalently, the ideal generated by $\{f,g\}$ is the entire $K[X]$.
        \item $f$ is called \textbf{prime} when any of the following 
            equivalent conditions are true: 
            \begin{enumerate}
                \item $f$ is non-zero, non-unit, and for all $a, b \in K[X]$, 
                $f \mid ab$ implies $f \mid a$ or $f \mid b$.
                \item The quotient ring $K[X]/(f)$ is an integral domain.
            \end{enumerate}
    \end{itemize}
    We leave as an exercise for the reader to prove that
    the definitions of prime are equivalent. 
\end{dfn}
%    - Lem - Irr Div or Coprime
\begin{lem} Irreducible Elements either Divide or Coprime.
    
    Let $r$ be an irreducible polynomial over a field $K$.
    Then for all polynomials $a$, either $r \mid a$ or $(r,a) = 1$.
\end{lem}
\begin{proof}
    By definition of the gcd, there exists $b \in K[X]$ such that $r = b(r,a)$.
    $r$ irreducible implies $b$ unit or $(r,a)$ unit. 
    If $(r,a)$ is a unit, then WLOG it is 1 and we are done.
    Now suppose $b$ unit. Then by definition of gcd again, 
    there exists $c \in K[X]$ such that $a = c(r,a) = cb^{-1}r$, 
    i.e. $r \mid a$. 
\end{proof}
%    - Thm - Prime iff Irr
\begin{thm} For Polynomial Rings, Prime if and only if Irreducible.
    
    Let $f$ be a polynomial over a field $K$. 
    Then $f$ prime if and only if $f$ irreducible.
\end{thm}
\begin{proof}
    (Prime $\imp$ Irreducible)
    The following works in general for integral domains. 
    Suppose $f$ is prime. Let $f = ab$ where $a, b \in K[X]$. 
    Then clearly $f \mid ab$. 
    So by primeness, $f \mid a$ or $f \mid b$. 
    Since the argument is symmetrical, WLOG $f \mid a$. 
    Then there exists a polynomial $c$ such that $a = cf$,
    so $f = ab = cbf$. 
    $f$ is non-zero since it is prime, so by $K[X]$ being an integral domain,
    $1 = cb$, i.e. $b$ is a unit. 
    
    (Irreducible $\imp$ Prime)
    Suppose $f$ is irreducible. Let $f \mid ab$ where $a, b$ are polynomials. 
    By irreducibility, $f \mid a$ or $(f, a) = 1$.
    If $f \mid a$, we are done. 
    So suppose $(f, a) = 1$. 
    Then there exists polynomials $\al, \phi$ such that $\al a + \phi f = 1$.
    This gives $b = \al a b + \phi b f$ which $f$ clearly divides, and we are done.
\end{proof}
%    - Thm - K[x] unique factorise
\begin{thm} Unique Factorisation for Polynomial Rings.
    
    Let $f$ be a non-zero, non-unit polynomial over a field $K$.
    Then there exists irreducible polynomials $r_1, \dots, r_n$ such that
    $f = r_1\cdots r_n$, and they are unique up to reordering and scaling by units.
    
    In general, rings where the unique factorisation into irreducibles hold
    are called \textbf{unique factorisation domains} or \textbf{UFD} for short.
\end{thm}
\begin{proof}
    We first prove uniqueness, which is a standard proof.
    Let $s_1 \cdots s_m = f = r_1 \cdots r_n$ where $s_j$ are irreducibles.
    Then clearly $r_1 \mid s_1 \cdots s_m$. 
    $r_1$ irreducible implies $r_1$ prime, which implies with induction that
    there exists $1 \leq j \leq m$, $r_1 \mid s_j$. 
    By reordering, WLOG $r_1 \mid s_1$. 
    Then by irreducibility of $s_1$ and $r_1$ non-unit, 
    $s_1 = \la_1 r_1$ where $\la_1$ is a unit in $K[X]$. 
    Then $r_1 \cdots r_n = \la_1 r_1 s_2 \cdots s_m$. 
    Since $K[X]$ is an integral domain and $r_1$ is non-zero, 
    we have $r_2 \cdots r_n = \la_1 s_2 \cdots s_m$. 
    Suppose $n \neq m$, WLOG $n < m$. 
    Then by induction, $1 = \la_1 \cdots \la_n s_{n+1} \cdots s_m$,
    which implies $s_m$ is a unit, contradicting the irreducibility of $s_m$.
    Hence $n = m$. Then by induction again, for all $1\leq i \leq n$,
    $s_i = \la_i r_i$ where $\la_i$ are units. 
    
    To prove existence, we proceed by strong induction on the degree of $f$. 
    So suppose the theorem is true for $\deg f < n$. 
    Let $\deg f = n$. 
    If $f$ is irreducible, then we are done. 
    Suppose $f$ is not irreducible. 
    Then there exists polynomials $a, b$ such that $f = ab$ and 
    neither $a$ nor $b$ are units. 
    Clearly, $a, b$ cannot be zero. 
    That implies $n = \deg f = \deg a + \deg b > \deg a, \deg b$. 
    Hence by assumption, $a, b$ are both products of irreducibles
    which implies $f$ is a product of irreducibles, completing the induction. 
\end{proof}

% Field of fractions? Specifically to construct field of rational polynomials 

\end{document}