\documentclass[../book.tex]{subfiles}

\begin{document}


%    - Eg - Splitting field of x^3 - 2
%        - Computation of the general form of elements
%        - 
%        - The Galois Grp / the trick
%        - subgrp tree
%        - subfield tree / technique for finding fixed subfields
%    - Thm - G-equivariance of Gal(L/ -)

% maybe some friendliness?
Concepts touched on in this chapter will be proven later on,
so if you see things that aren't obvious or 
you don't know the definitions of don't be put off.

\begin{eg} 

Consider $\Q$ as a field.
$X^3 - 2 \in \Q[X]$ has no solutions in $\Q$.
However, $\Q \subset \C$ and inside $\C[X]$, 
$X^3 - 2 = (X - r)(X - r\omega)(X - r\omega^2)$ where 
$r = \sqrt[3]{2}$, $\omega = exp(2\pi i / 3)$. 
We restrict our focus to the smallest subfield containing $\Q$ 
and these roots, denoted $\Q(r, r\omega, r\omega^2)$. 

Intuitively, $\Q(r, r\omega, r\omega^2)$ is what you get when you put
$\Q, r, \omega$ in a box with the axioms of a field and shake it, 
giving all possible arithmetic expressions. 

\begin{clm} 
    \[ 
        \Q(r, r\omega, r\omega^2) 
        = \{a + b r + c r^2 + d\omega + er\omega + fr^2\omega
        \mid a, \dots, f \in \Q\} 
    \]
\end{clm}

We will do this in two steps: 
First, throw $r$ in i.e. compute $\Q(r)$. 
Then, repeat by throwing $\omega$ into $\Q(r)$.

\begin{proof} 
    We first prove 
    $\Q(r) = \{f(r) \mid f \in \Q[X] \} =\{a + br + cr^2 \mid a, b, c \in \Q \}$ 
    by showing it is a field and it is the smallest containing $\Q$ and $r$
    (this is the definition of $\Q(r)$). 
    
    By the division algorithm on polynomials, 
    $\{f(r) \mid f \in \Q[X] \} =\{a + br + cr^2 \mid a, b, c \in \Q \}$.
    So we only show the first equality.
    Note that $ 0 = r^3 - 2 = g(r)$ where $g$ is irreducible in the ring $\Q[X]$ 
    (you cannot factorise it further without ending up with elements outside $\Q$).
    So we have  $\forall f \in \Q[X], g \mid f \,\lor\, (f, g) = 1$.

    Case $g \mid f$, $f(r) = 0$. 

    Case $(f, g) = 1$, $\exists \la,\,\mu\in\Q[X], \la f + \mu g = 1$. 
    Evaluating everything at $r$ gives $\la(r)f(r) = 1$. 
    Hence $f(r)=0$ or it is a unit. 
    This shows $\{f(r) \mid f \in \Q[X] \}$ is a field. 
    
    The least-property of $\{f(r) \mid f \in \Q[X] \}$ is trivial. (Why?)

    \ex{
        Adapt the proof above to show that 
        the smallest field containing $\Q(r)$ and $\omega$, 
        $\Q(r)(\omega) = \{a + b\omega \mid a, b \in \Q(r)\}$. 
        Hence complete the proof by convincing yourself that
        $\Q(r, \omega) := \Q(r)(\omega) = \Q(r, r\omega, r\omega^2)$. 
    }
\end{proof}

\ex{
    Show that $\Q(r, r\omega, r\omega^2)$ is a $\Q$-vector space 
    with $\{1, r, r^2, \omega, r\omega, r^2\omega \}$ as a basis. 
}

Now introducing maps. 
For a function $\sigma : \Q(r, r\omega, r\omega^2) \to 
\Q(r, r\omega, r\omega^2) $, 
we say $\sigma$ is a \emph{morphism of fields} if: 
\begin{enumerate}
    \item $\forall a, b \in \Q(r, r\omega, r\omega^2), 
        \sigma(a + b) = \sigma(a) + \sigma(b)$
    \item $\forall a, b \in \Q(r, r\omega, r\omega^2), 
        \sigma(ab) = \sigma(a)\sigma(b)$
    \item $\sigma(1) = 1$
\end{enumerate}
If $\sigma$ also bijects, it is called an \emph{isomorphism}.

\ex{
Let L = $\Q(r, r\om, r\om^2)$. 
Show that the set of all isomorphisms from $L$ to itself 
form a group under composition. 
This is denoted $Aut(L)$, the \emph{automorphism group of $L$}.
} 

Consider the following subgroup of $Aut(L)$, 
the \emph{Galois group of $L$ over $\Q$}
\[Gal(L/\Q) := \{\sigma \in Aut(L) \mid \forall x \in \Q, \sigma(x) = x\}\]

\begin{clm}
    $Gal(L/\Q) \iso D_6$. 
\end{clm}

\begin{proof} 
    We begin by noting that the condition $\forall x \in \Q, \sigma(x) = x$ 
    implies $\sigma$ is a $\Q$-linear map. 
    Since $L$ is a finite dimensional $\Q$-vector space, 
    $\sigma$ is determined the images of $1, r, r^2, \omega, r\omega, r^2\omega$.
    Furthermore, since we have the following equations: 
    \[ \sigma(1) = 1 \quad \sigma(r^2) = \sigma(r)^2 \]
    \[ \sigma(\omega) = \sigma(r^2 \cdot r\omega / 2) 
        = \sigma(r)^2\sigma(r\omega) / 2 \]
    \[ \sigma(r^2\omega) = \sigma(r) \sigma(r\omega) \]
    So $\sigma$ is actually determined by the images of $r, r\omega, r\omega^2$. 
    We proceed by doing cases on 
    $\sigma(r), \sigma(r\omega), \sigma(r\omega^2)$. 

    The rest of the proof hinges on a trick, which we will formalise later. 
    It is the following: 
    \begin{align*}
        0 &= \sigma(0) = \sigma(r^3 - 2) 
            = \sigma(r^3) - \sigma(2) = \sigma(r)^3 - 2 \\
        &= (\sigma(r) - r) (\sigma(r) - r\omega) (\sigma(r) - r\omega^2) 
            \imp \sigma(r) = r, r\omega, r\omega^2 
    \end{align*}

    The same follows for $\sigma(r\omega)$ and $\sigma(r\omega^2)$. 
    Since $\sigma$ is bijective, thus injective,  
    we see that $\sigma \in Gal(L/\Q)$ simply permute the three roots. 
    Going backwards, extending linearly and mutiplicatively yields 
    isomorphisms again. 
    Hence, $Gal(L/\Q)$ is isomorphic to $S_3 \iso D_6$.  
\end{proof}
Since $Gal(L/\Q) \iso S_3$, 
we will denote an element $\sigma$ by the triplet of images 
$(\sigma(r), \sigma(r\om), \sigma(r\om^2) )$.

Let us now consider subgroups of $Gal(L/\Q) \iso D_6$.
Letting $\phi, \psi$ denote a flip and rotation that generate $D_6$, 
we have the following subgroups: 
\begin{figure}[ht]
    \centering
    \begin{tikzcd}
        & D_6 & & \\
        \langle\psi\rangle \arrow[ru,"2"]&\langle\phi\rangle \arrow[u,"3"]&
        \langle\phi\psi\rangle \arrow[lu,"3"]
        &\langle\psi\phi\rangle \arrow[llu,"3"]\\
        &\langle e \arrow[lu,"3"] \arrow[u,"2"] \arrow[ru,"2"]\rangle
        \arrow[rru,"2"]& & 
    \end{tikzcd}
\end{figure}

The arrows denote subgroup and the numbers are the index of the subgroups. 
Let $(r, r\omega^2, r\omega) \mapsto \phi$ and 
$(r\omega,r\omega^2,r) \mapsto \psi$ be an isomorphism $Gal(L/\Q) \iso D_6$. 
Then there is a way we can obtain subfields of $L$ containing $\Q$. 
For $H \leq Gal(L/\Q)$, we define the \emph{subfield fixed by $H$} to be
\[L^H := \{z \in L \mid \forall \sigma \in H, \sigma(z) = z\} \]
\ex{Show that $L^H$ is indeed a field inside $L$, containing $\Q$.}

\begin{clm}
    \[ L^{\langle \psi \rangle} = 
    \{ a + b\omega \mid a, b \in \Q \}\]
\end{clm}

\begin{proof}
    Let $z = \alpha + \beta r + \gamma r^2 
    + \delta \omega + \mu r\omega + \nu r^2\omega$, 
    $\alpha, \dots, \nu \in \Q$, $\psi (z) = z$. 
    Then \begin{align*}
        & \alpha + \beta r + \gamma r^2 
        + \delta \omega + \mu r\omega + \nu r^2\omega \\
        =\;&\psi(\alpha + \beta r + \gamma r^2 
        + \delta \omega + \mu r\omega + \nu r^2\omega) \\
        =\;& \alpha + \beta \psi(r) + \gamma \psi(r)^2 + \delta\psi(r\omega)/\psi(r)
        + \mu \psi(r\omega) + \nu \psi(r)\psi(r\omega) \\
        =\;& \alpha + \beta r \om + \gamma r^2 \om^2 + \delta r\om^2 / r\om
        + \mu r \om^2 + \nu r\om r\om^2 \\
        =\;& \al -\mu r + (\nu - \ga)r^2 + \de\om + (\be - \mu)r\om - \ga r^2 \om
        \end{align*}
    By the elements being a basis, we have from linear independence
    $(\al,\be,\ga,\de,\mu,\nu)$ $= (\al,-\mu,\nu-\ga,\de,\be-\mu,-\ga)$  
    Hence, $z = \al + \de\om$. 
    Since $\psi$ generates $\langle \psi \rangle$, $\psi^2(z) = z$. 
    So $L^{\langle \psi \rangle} = \{ \al + \de\om \mid \al, \de \in \Q\}$ 
    as desired. 
\end{proof}
\ex{By adapting the proof above, compute the other subfields fixed by
the other subgroups of the Galois group, thus obtaining the following diagram.}
\begin{figure}[ht]
    \centering
    \begin{tikzcd} [column sep = 0.1em]
        & \Q \arrow[dl,"2"] \arrow[d,"3"] \arrow[dr,"3"] \arrow[drr,"3"] & & \\
        \Q(\om) \arrow[dr,"3"] & \Q(r) \arrow[d,"2"] & 
        \Q(r\om^2) \arrow[dl,"2"] & \Q(r\om) \arrow[dll,"2"]\\
        & \Q(r,r\om,r\om^2)& &
    \end{tikzcd}
\end{figure}

Here, the arrows denote subfields and the numberings denote
the dimension of the superfield as a vector space over the subfield. 
But somehow, it appears the diagram of subfields looks strikingly similar to
the diagram of subgroups of the Galois group! 
This is the basically Galois correspondance. 

\begin{thm} Basically Galois Correspondance (for Extensions of $\Q$). 

    Let $\Q \subseteq L$ where $L$ is a field. 
    Then under ``nice" conditions, 
    there is a order-reversing bijection between
    subgroups of the Galois group $Gal(L/\Q)$ and 
    the subfields of $L$ containing $\Q$. 
    This bijection also preserves indices and dimension. 
    
\end{thm}

As we will see, ``nice" means $L$ is a \emph{Galois extension} over $\Q$. 
The correspondance also behaves nicely with respect to natural group actions
of $Gal(L/\Q)$. 

\ex{
    Let $M$ be a field such that $\Q \subseteq M \subseteq L$. 
    Then for all $\sigma \in Gal(L/\Q)$, $\sigma M$ is also a field.
    This defines an action of $Gal(L/\Q)$ on subfields of $L$ containing $\Q$. 
    Compute the orbits of this action.
    Then, let $Gal(L/\Q)$ act on its subgroups via conjugation. 
    Compute the orbits of this action (i.e. the conjugacy classes). 
    Hence, see that the bijection also preserves orbits. 
}
\begin{figure}[ht]
    \centering
    \begin{tikzcd}[column sep = 0.1em]
        & D_6 & & & 
        & \Q \arrow[dl,"2"] \arrow[d,"3"] 
        \arrow[dr,"3"] \arrow[drr,"3"] & &\\
        \color{blue}\langle\psi\rangle \arrow[ru,"2"]&
        \color{red}\langle\phi\rangle \arrow[u,"3"]&
        \color{red}\langle\phi\psi\rangle \arrow[lu,"3"]&
        \color{red}\langle\psi\phi\rangle \arrow[llu,"3"]& 
        \color{blue}\Q(\om) \arrow[dr,"3"] & 
        \color{red}\Q(r) \arrow[d,"2"] & 
        \color{red}\Q(r\om^2) \arrow[dl,"2"] & 
        \color{red}\Q(r\om) \arrow[dll,"2"]\\
        &\langle e \arrow[lu,"3"] \arrow[u,"2"] \arrow[ru,"2"]\rangle
        \arrow[rru,"2"]& & & & \Q(r,r\om,r\om^2)& &
    \end{tikzcd}
    \caption{Same colour indicates same orbit.}
\end{figure}

\end{eg}

\end{document}