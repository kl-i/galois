\documentclass[../book.tex]{subfiles}

\begin{document}

Now we are ready to talk about the building blocks of Galois Theory.

%    - Def - Field Ext (as an algebra), Deg Ext, Fin Ext
\begin{dfn} Field Extension, Morphism of Extensions, Extension Degree.

    Let $K$ be a field. 

    A $K$-extension is a pair $(L,\iota)$ where $L$ is a field 
    and $\iota : K \to L$ is a ring morphism. 
    Since all ring morphisms of rings from a field are injective,
    one can see $K$ as ``sitting inside" $L$ via $\iota$.\footnote{
        i.e. elements in $K$ bijective with 
        elements in the iamges of $\iota$. 
    }
    We sometimes write ``$\iota : K \to L$" instead of $(L,\iota)$.
    If we say ``$\iota : K \to L$ is an extension", we mean
    $L$ is a $K$-extension.
    
    For a $K$-extension $(L,\iota)$, 
    $L$ is naturally a $K$-vector space with the following action: 
    \[
        K \to End_\Grp((L,+)), k \mapsto (l \mapsto \iota(k) l)
    \]
    i.e. scalar multiplication for $k \in K, l \in L$ is $kl := \iota(k)l$.
    The \textbf{degree} of $\iota : K \to L$ is then defined as
    the dimension of $L$ as a $K$-vector space, denoted as \[
        [L : K] := \dim_K L
    \]
    The extension is called a \textbf{finite extension} when $[L:K]$ is finite. 
    
    Let $\iota_L : K \to L, \iota_M : K \to M$ be $K$-extensions and
    $f : L \to M$ a ring morphism.
    Then $f$ is a \textbf{morphism of $K$-extensions} when 
    any of the following equivalent conditions are true: 
    \begin{enumerate}
        \item $f$ is a $K$-vector space morphism.
        \item $f$ is the identity on the images of $K$, i.e.
        $f\circ\iota_L = \iota_M$
    \end{enumerate}
    We write $f : (L,\iota) \to (M,\iota_M)$. 
    If it is clear what the extensions are, we just write $f : L \to M$. 
    If $f$ bijects, it is called an \textbf{isomorphism} and
    we say $(L,\iota_L)$ and $(M,\iota_M)$ are \textbf{isomorphic as $K$-extensions}.
    A morphism from an extension to itself is called an \textbf{endomorphism}.
    If it is additionally an isomorphism, it is called an \textbf{automorphism}. 
    
    With seeing $K$ as sitting inside field extensions, 
    $K$-extension morphisms are the functions that ``preserve $K$" 
    inside the extensions. 
\end{dfn}

\begin{rmk}
    
    Note that something like $K \to K[X], \la \mapsto \la$ is \emph{not}
    a field extension since the polynomial ring $K[X]$ is not a field. 
    The following result gives 

\end{rmk}

%\begin{dfn} Field Extension, Morphism of Extensions,Extension Degree,
%Finite Extension.

%Let $K$ be a field. Then
%a \textbf{field extension} is a pair 
%We write ``$\iota : K \to L$" instead of $(L,\iota)$.
%Note that since all morphisms of rings from a field is injective,
%one can see $K$ as ``sitting inside" $L$ via $\iota$. 

%Let $\iota : K \to L$ be a field extension. Then 
%$L$ becomes a $K$-vector space with scalar multiplication $kl:=i(k)l$. (Check this!)
%Then the \textbf{degree of $L$ over $K$} is the dimension of $L$ 
%as a $K$-vector space,
%\[
%    [L : K] := \dim_K L
%\]
%The extension is called finite if $[L : K]$ is finite.  
%\end{dfn}

%    - Thm - Tower Law of Deg Ext
\begin{thm} Tower Law of Extension Degree. \\
Let $\iota_0 : K \to L$ be a finite $K$-extension and 
$\iota_1 : L \to M$ a finite $L$-extension.
Then $\iota_1 \circ \iota_0 : K \to M$ is a finite $K$-extension, and 
\[
    [M : K]=[M : L][L : K]
\]
\end{thm}
\begin{proof}
The first assertion holds since compositions 
of ring morphisms are also ring morphisms. 
For the multiplicativity of degree, 
let $B_M=\{y_1,y_2,\cdots,y_n\} \subseteq M$ be a basis of $M$ 
as a vector space over $L$, 
and $B_L=\{x_1,x_2,\cdots,x_m\} \subseteq L$ a basis of $L$ 
as a vector space over $K$.
We claim that $B = \{x_iy_j:1 \leq i \leq n, 1\leq j \leq m\}$ 
is a basis of $M$ over $K$.

\begin{itemize}
    \item Spanning: 
        Let $z \in M$. 
        Then there exists $\lambda_i \in L$ such that 
        $z=\displaystyle\sum_{i=1}^{n} \lambda_iy_i$, 
        and for each $\lambda_i \in L$, there exists ${\mu_i}_j \in K$ 
        such that $\lambda_i=\displaystyle\sum_{j=1}^{m} {(\mu_i)}_j x_j$. 
        Hence,\[
            z = \sum_{i=1}^{n} \lambda_iy_i = 
            \sum_{i=1}^{n} \sum_{j=1}^{m} {(\mu_i)}_j x_j y_i,
        \]
        which is spanned by $\{x_iy_j\}$.
    \item Linear Independence: 
        Let $\lambda_{ij} \in K$ such that 
        $\displaystyle\sum_{ij}\lambda_{ij} x_iy_j=0$. 
        Then \[\sum_j(\sum_i \lambda_{ij} x_i)y_j=0\]
        So by $B_M$ linear independence, 
        we have $\sum \lambda_{ij}x_i=0$, 
        which means $\lambda_{ij}=0$ 
        since $B_L$ is also linearly independent.
\end{itemize}
Clearly, $|B| \leq nm$. Assume that $x_i y_j = x_k y_l \in B$, 
then by linear independance of $B_M$, 
we have $x_i = x_k = 0$, a contradiction with the linear independence of $B_L$. 
Thus $|B| = nm$.
%The fact the $|B| = nm$ comes from the $L$-linear independence of $B_M$. 
\end{proof}

%    - Def - Subfield Generated by a Subset, Construction and Min Prop
\begin{dfn} Subfields, Subextensions generated by Subsets, Simple Extensions. 

    Let $L$ be a field and $M \subseteq L$.
    Then the following are equivalent: 
    \begin{enumerate}
        \item $M$ forms a field with addition and multiplication from $L$.
        \item $M$ is the image of a ring morphism from some field into $L$. 
    \end{enumerate}
    If any of the above are true, $M$ is called a \textbf{subfield} of $L$. 
    We leave as an exercise to the reader to check that for any subset $S \subseteq L$,
    there exists a unique \emph{smallest} subfield of $L$ containing $S$.
    This subfield is called the \textbf{subfield generated by $S$}, denoted $\<S\>$. 

    Let $\iota : K \to L$ be a $K$-extension, $S \subseteq L$. 
    Then the \textbf{$K$-subextension of $L$ generated by $S$} or 
    \textbf{$K$ adjoin $S$} is defined to be the subfield generated by $\iota K \cup S$, 
    \[ K(S) := \< \iota K \cup S\> \]
    If $S = \{a\}$ is a singleton set, we write $K(a)$ instead
    and call it a \textbf{simple extension}. 
    In this case, $a$ is sometimes called the \textbf{primitive element}. 
    If $L = K(S)$, then we say \textbf{$L$ is generated by $S$ as a $K$-extension}.
    
    In general, $M \subseteq L$ is called a \textbf{$K$-subextension of $(L,\iota_L)$}
    when any of the following equivalent statements are true:
    \begin{enumerate}
        \item $M$ is a subfield of $L$ that contains $\iota_L K$, i.e.
        $M$ forms a $K$-extension using $\iota_L$.  
        \item $M$ is the image of a $K$-extension morphism into $(L,\iota_L)$. 
    \end{enumerate}

\end{dfn}



\end{document}