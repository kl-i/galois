\documentclass[../book.tex]{subfiles}

\begin{document}
%    - Thm - Nice properties of (-)[x] (is a functor)
\begin{lem} Extensions induce Ring Morphisms on the Polynomial Ring. 

    Let $\iota_0 : K \to L, \iota_1 : L \to M$ be extensions. 
    Then we have a natural morphism of rings $\bar{\iota_0} : K[X] \to L[X]$
    and $\bar{\iota_1} : L[X] \to M[X]$ such that 
    $\bar{\iota_1 \circ \iota_0} = \bar{\iota_1} \circ \bar{\iota_0}$. 
    \begin{figure}[ht]
        \centering
        \begin{tikzcd}
        K \arrow[r,"\iota_0"] \arrow[dr,"\iota_1 \circ \iota_0" below left] &
        L \arrow[d,"\iota_1"]& & 
        K[X] \arrow[r,"\bar{\iota_0}"] \arrow[dr,"\bar{\iota_1 \circ \iota_0} 
        =    \bar{\iota_1}\circ\bar{\iota_0}" below left] & 
        L[X] \arrow[d,"\bar{\iota_1}"] \\
        & M & & & M[X]
        \end{tikzcd}
        \label{fig:polyring_functor}
    \end{figure}
\end{lem}
\begin{proof}
    $\bar{\iota_0} : K[X] \to L[X], 
    \sum_{k=0}^n \la_kX^k \mapsto \sum_{k = 0}^n \iota_0(\la_k)X^k$. 
    
    (Check the composition-preserving property.)
\end{proof}
%    - Def - Evaluation Homomorphism. 
\begin{lem} Evaluation Morphism. 

    Let $K$ be a field, $a \in K$. 
    Then the \emph{evaluation at a} 
    \[ev_a : K[X] \to K, 
    f = \sum_{k\in\deg f} f_k X^k \mapsto \sum_{k\in\deg f} f_k a^k\]
    is a morphism of rings. 
    We often write "$f(a)$" for $ev_a (f)$. 
    Afterall, elements in polynomial rings are meant to be polynomials! 

    Furthermore, given an extension $\iota : K \to L$, 
    we have $\iota \circ ev_a = ev_{\iota a} \circ \bar{\iota}$,
    i.e. the following diagram commutes. 
    \begin{figure}[ht]
    \centering
    \begin{tikzcd}
    K[x] \arrow[r,"\bar{\iota}"] \arrow[d,"ev_a"]
    & L[X] \arrow[d,"ev_{\iota a}"]\\
    K \arrow[r,"\iota"]& L\\
    \end{tikzcd}
    \label{fig:ev}
\end{figure}
\end{lem}
\begin{proof} Both statements are straightforward to check. 
\end{proof}

%    - Def - Alg Element, Alg ext
\begin{dfn} Algebraic and Transcendental Elements, Algebraic Extensions.

    Let $\iota : K \to L$ be a field extension and $a \in L$. 
    Then we have the ring morphism \[
        ev_a \circ \bar\iota : K[X] \to L, 
        f \mapsto (\bar\iota f)(a) 
    \]
    i.e. takes polynomials over $K$ and evaluates them at $a$ inside $L$.
    $a$ is called \textbf{algebraic over $K$} when 
    $\ker (ev_a \circ \bar\iota) \neq (0)$.
    For polynomials $g \in \ker (ev_a \circ \bar\iota)\setminus \{0\}$, 
    we say \textbf{$g$ has $a$ as a root}. 
    Otherwise, $a$ is called \textbf{transcendental over $K$}. 
    
    $\iota : K \to L$ is called an \textbf{algebraic extension}
    when for all $a \in L$, $a$ is algebraic over $K$.
\end{dfn}
%    - Def - \min(a,K) 3 eqv def (min deg, irr, generates ideal) and uniqueness
\begin{dfn}
Let $\iota : K \to L$ be an extension, $a \in L$, $a$ is algebraic over $K$. 
Then there exists a unique non-zero, monic polynomial $\min(a,K)$ 
with $a$ as a root such that one of the three equivalent conditions holds:
\begin{enumerate}
    \item $\min(a,K)$ has minimal degree amongst other 
        non-zero polynomials with $a$ as a root.
    \item $\min(a,K)$ generates $\ker ev_{a} \circ \bar{\iota}$. 
    \item $\min(a,K)$ is irreducible. 
\end{enumerate}
Then $\min(a,K)$ is called the \textbf{minimal polynomial of $a$ over $K$}. 
\end{dfn}
\begin{proof}
    Existence and uniqueness satisfying condition (1) and equivalence to (2) 
    are given by $K[X]$ being a PID. 
    
    $(2 \imp 3)$
    Suppose $\min(a,K)=pq$ for some $p,q \in K[X]$.
    Then evaluating at $a$ gives $p(a)q(a)=0$ in K and hence WLOG $p(a)=0$,
    as K is an integral domain.
    This implies $p \in \ker(ev_a \circ \bar\iota)$, so $\min(a,K) \mid p$.
    Hence $\min(a,K) = pq = rq \min(a,K)$ for some polynomial $r$.
    Since $K[X]$ is an integral domain and $\min(a,K)$ is non-zero, $q$ is a unit.
    
    Here is another proof of ($2 \imp 3$):
    $\min(a,K)$ being irreducible is equivalent to it being prime,
    which is equivalent to $K[X]/(\min(a,K))$ being an integral domain.
    Since $\min(a,K)$ generates $\ker(ev_a \circ \bar\iota)$, 
    we have via the 1st isomorphism theorem for rings, \[
        K[X]/(\min(a,K)) \iso \im (ev_a \bar\iota) \leq L
    \]
    A subring of an integral domain is clearly also integral, hence we are done.
    \footnote{Though the first proof is somewhat more direct, 
    the second proof uses a lot of cool tools we have developed so far. 
    In particular, this isomorphism will show up again later.}
    
    ($3 \imp 2$)
    Let $f$ be some other polynomial in $\ker(ev_a \circ\bar\iota)$.
    Then by irreducibility, 
    \[\min(a, K) \mid f \text{ or } (\min(a,K),f) = 1\]
    Suppose $(\min(a,K),f) = 1$. 
    Then by evaluating everything at $a$, we get $1 = 0$ inside $L$,
    which is a contradiction since $L$ is a field. 
    Hence, $\min(a,K)$ divides all polynomials in $\ker(ev_a \circ \bar\iota)$,
    i.e. generates the kernel.
\end{proof}
\begin{rmk}
    From the motivating example in the first section of the book, 
    we know that for a simple extension $K(a)$ ought to look like
    polynomials in $a$ over $K$. 
    The following theorem proves this.
\end{rmk}
%    - Thm - K(a) = K[a] i.e. Simple Ext iso to Quotient by Min Poly
\begin{thm} The Form of Simple Extensions by Algebraic Elements.

    Let $\iota : K \to L$ be a field extension and 
    $a \in L$ be an algebraic element over $K$. 
    Then $K(a) = (ev_a \circ\bar\iota)K[X]$.
    I.e. $K(a)$ looks like polynomials over $K$ evaluated at $a$.
\end{thm}
\begin{proof}
    Clearly $(ev_a\circ\bar\iota) K[X] \subseteq K(a)$.
    To show the other inclusion, let us recall that
    $K(a)$ is defined as the smallest field containing $\iota K$ and $a$.
    $\iota K$ and $a$ are clearly inside $(ev_a\circ\bar\iota) K[X]$,
    so we only need to prove that it is a field.
    
    Using the 1st isomorphism theorem, we get \[
        (ev_a\circ\bar\iota) K[X] \iso K[X]/(\min(a,K))
    \]
    The irreducibility of $\min(a,K)$ implies that 
    for all polynomials $f$ in $K[X]$, $\min(a,K) \mid f$ or 
    $\al \min(a,K) + \be f = 1$ for some polynomials $\al, \be$.
    Projecting everything to the quotient $K[X]/(\min(a,K))$ shows that
    every element in the quotient is either 0 or unit.
    The quotient is thus a field and we are done. 
\end{proof}
\begin{rmk}
    We can think of $K[X]/(\min(a,K))$ as 
    \[\displaystyle 
        \{\sum_{i=0}^{\deg\min(a,K) - 1} \la_i X^i \mid \la_i \in K\}
    \]
\end{rmk}

%    - Ex - Deg Ext = Deg Min Poly
\begin{ex} The Degree of Simple Extensions over Algebraic Elements.

    Let $\iota : K \to L$ be a field extension and 
    $a \in L$ be an algebraic element over $K$. 
    Show that $[K(a):K] = \deg \min(a, K)$. 
    (Hint: You now know $K(a) \iso K[X]/(\min(a,K))$.)
\end{ex}

%Some equivalent conditions for an element to be algebraic.
%    - Lem - K(a) fin iff a alg
\begin{lem}
    Let $\iota : K \to L$ be a field extension and $a \in L$. 
    Then the following are equivalent:
    \begin{enumerate}
        \item $a$ is algebraic over $K$
        \item $K \to K(a)$ is a finite extension
        \item $K \to K(a)$ is an algebraic extension.
    \end{enumerate}
\end{lem}
\begin{proof}
    ($1 \imp 2$)
        The previous exercise.
        
    ($2 \imp 3$)
        Let $[K(a):K]$ be finite, $b \in K(a)$.
        Then $K(b)$ is a subfield of $K(a)$ containing $\iota K$. 
        So by the Tower Law, we have $[K(b):K]$ finite as well.
        Here is the trick. 
        
        Consider the set of powers of $b$, $\{b^k\}_{k\in\N}$.
        Suppose this is a linearly independent set.
        Then any finite subset of $\{b^k\}_{k\in\N}$ is also linearly independent.
        In particular, $\{b^k\}_{k\in[K(b):K]+1}$ is a linearly independent set
        of $[K(b):K]+1$ vectors.
        By Steinitz exchange, we know that in a finite dimensional vector space, 
        any finite linearly independent set has less than or equal cardinality
        than the dimension. 
        Since $K(b)$ is finite dimensional with dimension $[K(b):K]$,
        we have a contradiction.
        
        Thus $\{b^k\}_{k\in\N}$ is not linearly independent, i.e.
        there is a $K$-linear combination of powers of $b$ equal to zero
        with not-all-zero coefficients.
        This is clearly just a non-zero polynomial over $K$ 
        with $b$ as a root!
        
    ($3 \imp 1$)
        Trivial.
\end{proof}
%    - Lem - L/K fin -> L/K Alg
\begin{thm} Finite Extensions are Algebraic.

    Let $\iota : K \to L$ be a finite field extension. 
    Then it is an algebraic extension.
\end{thm}
\begin{proof}
    Let $a \in L$. Then $K(a)$ is a subfield of $L$ containing $\iota K$.
    Hence by the Tower Law, $[K(a) : K]$ is finite,
    which implies $a$ is algebraic by the previous lemma. 
\end{proof}
\begin{rmk}
    The fact that all finite extensions are algebraic means that
    as far as we are concerned, we do not have to worry about
    elements not being algebraic.
\end{rmk}

% Example of infinite algebraic extension
\begin{ex} An Infinite Algebraic Extension.

    Take $\Q(\{2^{1/n}\}_{n \in \N_{>0}})$ as a $\Q$-extension. 
    We have the following sequence of extensions, 
    adjoining a root of $2$ one by one. 
    \begin{figure}[H]
        \centering
        \begin{tikzcd}
        & 
        & 
        \Q(\{2^{1/n}\}_{n \in \N_{>0}}) & 
        & 
        \\
        \Q = \Q(2^{1/1}) \arrow[r,"\subseteq"{sloped,below}] 
        \arrow[rru,"\subseteq"{sloped,above}] &
        \Q(2^{1/2}) \arrow[r,"\subseteq"{sloped,below}] 
        \arrow[ru,"\subseteq"{sloped,below}] &
        \Q(2^{1/2},2^{1/3}) \arrow[r,"\subseteq"{sloped,below}] 
        \arrow[u,"\subseteq"{sloped,above=1.5pt}] &
        \Q(2^{1/2},2^{1/3},2^{1/4}) \arrow[r,"\subseteq"{sloped,below}] 
        \arrow[lu,"\supseteq"{sloped,below}] &
        \cdots \arrow[llu,"\supseteq"{sloped,above}]
        \end{tikzcd}
    \end{figure}
    Clearly, \[
        \bigcup_{k = 2}^\infty \Q(\{2^{1/n} \mid n < k\}) = 
        \Q(\{2^{1/n}\}_{n \in \N_{>0}})
    \]
    
    We claim that $\Q \to \Q(\{2^{1/n}\}_{n \in \N_{>0}})$ 
    is an infinite algebraic extension. 
    You can see how there exists an infinite basis (don't bother with the details),
    thus the extension is infinite.
    Any element $a$ in $\Q(\{2^{1/n}\}_{n \in \N_{>0}})$ 
    will be a finite linear combination of elements in the basis, 
    and so $a \in K(a) \subseteq \Q(\{2^{1/n}\}_{n \in S})$, where $S$ is finite. 
    Thus $[K(a):K]$ is finite by the tower law, and $a$ is algebraic by our lemma.
    
    %Ugly-ass shit. Someone else write this. 
    %Meh not too bad if you avoid the yucky part.
\end{ex}

We now have enough to solve some problems the greeks had.

%    - Eg - Greek Geometry
%        - Def - Constructible in one step/ Constructible
%        - Lem - Constructible Extensions have deg 1 or 2
%        - Cor - Impossible to Trisect angle/ Double Cube / Square Circle

\end{document}