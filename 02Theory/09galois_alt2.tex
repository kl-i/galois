\documentclass[../book.tex]{subfiles}

\begin{document}

We are now ready to define \emph{Galois extensions}. 

%    -  Def - Galois Ext, Galois Group
\begin{dfn} Fixed Subfield, Galois Extension and Galois Group.

    Let $\iota_L : K \to L$ be a field extension.
    Let $\aut{K}{(L,\iota_L)}$ denote the set of $K$-extension automorphisms of $L$.
    If $\iota_L$ is clear, we simply write $\aut{K}{L}$. 
    This forms a group under function composition and has an obvious action on $L$. 
    For any subgroup $G$ of the group of $\aut{K}{L}$, 
    define \textbf{the subfield of $L$ fixed by $G$} as
    \[ L^G := \{l \in L \mid \forall g \in G, g(l)=l\} \]
    Then following are equivalent.
    \begin{enumerate}
        \item $(L,\iota_L)$ is a finite, normal and separable extension.
        \item There exists a finite subgroup $G$ of $\aut{K}{L}$ such that
        $\iota_L K = L^G$. 
    \end{enumerate} 
    Furthermore, if $(1)$ is the case, $[L : K] = |\aut{K}{L}|$ 
    and if $(2)$ is the case, $G = \aut{K}{L}$.
    If either of the cases are true,
    then $(L,\iota_L)$ is called a \textbf{Galois extension}.
    $\aut{K}{L}$ is then called the \textbf{Galois group of $(L,\iota_L)$}.
    If $\iota_L$ is clear, we say the \textbf{Galois group of $L$ over $K$}. 
\end{dfn}
\begin{proof}
    ($1\imp 2$)
        
        Note that $(L,\iota_L)$ separable implies $[L:K] = [L:K]_S$.
        Since $L$ itself is an extension extending $\iota_L : K \to L$ that is normal,
        we have \[
            [L : K] = [L : K]_S = |\mor{K}{L}{L}| = |\aut{K}{L}|
        \]
        i.e. $\aut{K}{L}$ is finite. 
        
        We claim that $G = \aut{K}{L}$ works. 
        To prove the image of $K$ is the subfield of $L$ fixed by $\aut{K}{L}$,
        note that we already have that $\iota_L K$ is preserved by elements 
        in $\aut{K}{L}$.
        % \[
        %     K \overset{\iota_L}{\to} L^{\aut{K}{L}} \overset{\subseteq}{\to} L
        % \]
        Now let $b$ be an element of $L$ not in $\iota_L K$. 
        $[L : K]$ finite implies $b$ is algebraic over $K$. 
        Since $b$ not in $\iota_L K$, $\min(b,K)$ must have degree higher than one. 
        $L$ normal and separable then implies 
        there exists a Galois conjugate $\be$ of $b$ \emph{not equal to} $b$. 
        So by embedding via conjugates, 
        we have a $K$-extension morphism $\sigma$ 
        from $K(b)$ to $L$ mapping $b$ to $\be$.
        Then we have an $K(b)$-extension automorphism $\bar{\sigma}$ of $L$
        from $K(b) \subseteq L$ to $\sigma : K(b) \to L$, 
        i.e. $\bar{\sigma}$ restricted to $K(b)$ is equal to $\sigma$. 
        Then clearly, $\bar{\sigma}$ is a $K$-extension automorphism of $L$
        that does not fix $b$, 
        so $b$ is not in $L^{\aut{K}{L}}$.
        This proves $\iota_L K = L^{\aut{K}{L}}$. 
        
    ($2\imp 1$)
        Since $\iota_L K = L^G$, it sufficient to consider the extension 
        $L^G \subseteq L$. 
        
        Let $a \in L$. We claim that \[
            \min(a,L^G) = \prod_{b \in Orb(a)} (X - b)
        \]
        where $Orb(a)$ is the orbit of $a$ under the obvious action of $G$ on $L$. 
        This just says the Galois conjugates of $a$ is precisely
        the images of $a$ under $K$-extension automorphisms of $L$
        and they are all of them.
        From this, we have $\min(a,L^G)$ splits and separable. 
    
        We first show that the product is actually in $L^G[X]$. 
        Let $g \in G$. Then \[
            \bar{g} \prod_{b \in Orb(a)} (X - b) 
            = \prod_{b \in Orb(a)} (X - g(b)) = \prod_{b' \in Orb(a)} (X - b')
        \]
        Hence the product is fixed by all $g \in G$. 
        Writing the product as $\sum_{k\in n} \la_k X^k$ where $\la_k \in L$,
        we have $\la_k \in L^G$, i.e. the product is indeed in $L^G[X]$. 
        
        We now prove the product divides 
        all the polynomials over $L^G$ with $a$ as a root.
        Let $f$ be a polynomial over $L^G$ such that $ev_a(f) = 0$.
        Let $b\in Orb(a)$. 
        Then there exists a $\sigma \in G$ such that $\sigma(a) = b$.
        Since $f$ is fixed by $G$, we have
        \[
            ev_b (f) = ev_{\sigma(a)}(f) 
            = ev_{\sigma(a)}(\bar{\sigma} f)
            = \sigma (ev_a (f)) = \sigma(0) = 0
        \]
        i.e. $(X - b)$ divides $f$.
        So the product divides $f$, and hence generates the ideal $\ker(ev_a)$.
        Thus it is equal to $\min(a,L^G)$. 
        
        % We now prove the product is irreducible in $L^G[X]$.
        % The product is clearly non-zero and non-unit. 
        % Let $f, g$ be polynomials over $L^G$ such that \[
        %     fg = \prod_{b \in Orb(a)} (X - b)
        % \]
        % Evaluating both sides at $a$, we have $0 = ev_a f$ or $0 = ev_a g$. 
        % WLOG $ev_a f = 0$. 
        % Then since $f \in L^G[X]$, for all $g \in G$, $\bar{g}f = f$, 
        % we have \[
        %     0 = g(0) = g(ev_a f) = ev_{g(a)}(\bar{g} f) = ev_{g(a)} f
        % \]
        % Hence for all $b \in Orb(a)$, $b$ is a root of $f$, i.e.
        % there exists a polynomial $h$ over $L$ such that \[
        %     h \prod_{b \in Orb(a)} (X - b) = f
        % \]
        % The by $L[X]$ is an integral domain, $1 = gh$,
        % which implies $\deg g = 0$, giving $g$ is a unit in $L^G[X]$. 
        % This proves irreducibility of the product in $L^G[X]$. 
        % Thus it is equal to $\min(a,L^G)$. 
    
        Note we cannot conclude the extension to be normal and separable yet,
        since we defined those as \emph{finite} extensions.
        So we prove $L^G \to L$ is finite next.
    
        Suppose $|G| < [L : L^G]$. 
        Then there exists finite linearly independent set $S$ such that $|G| < |S|$. 
        Since the $L^G$-subspace generated by $S$ is contained in $L^G(S)$, 
        we have \[ 
            |G| < |S| = \dim_{L^G} \<S\>_{L^G} \leq [L^G(S) : L^G] 
        \]
        $S$ finite and algebraic over $L^G$ gives $L^G(S)$ is a finite extension,
        and thus a separable extension as well. 
        Hence there exists a primitive element $a$ where $L^G(S) = L^G(a)$. 
        But then \[
            |G| < [L^G(S) : L^G] = [L^G(a) : L^G] = \deg\min(a,L^G)
            = |Orb(a)| \leq |G|
        \]
        which is a contradiction. 
        Thus $[L : L^G] \leq |G|$ which is finite,
        so $L^G \to L$ is normal and separable as well.
    
        We already have \[
            |\aut{K}{L}| = [L : K] = [L : L^G] \leq |G|
        \]
        Since $G$ is a subset of $\aut{K}{L}$, 
        $|G| \leq |\aut{K}{L}|$ as well and hence $G = \aut{K}{L}$.  
    
\end{proof}

% ex - Galois Closure
\begin{ex} Galois Closure of a Separable Extension.
    
    Let $\iota_L : K \to L$ be an extension.
    Then an $L$-extension $(N,\iota_N)$ is called 
    the \textbf{Galois closure of $(L,\iota_L)$} 
    when it is the smallest $L$-extension that is Galois as a $K$-extension,
    i.e. for all $L$-extensions $(M,\iota_M)$,
    $(M,\iota_M\circ\iota_L)$ Galois implies
    there exists an $L$-extension morphism $\bar{\iota_M} : N \to M$. 
    
    Suppose $\iota_L : K \to L$ be a finite, separable extension. 
    Show that there exists a $(N,\iota_N)$ Galois closure of $(L,\iota_L)$
    and it is unique up to $L$-extension isomorphisms i.e.
    any other Galois closure of $(L,\iota_L)$ is isomorphic to $(N,\iota_N)$
    as an $L$-extension. 
    
\end{ex}

%    - Thm - Galois Correspondance / Fund Thm Gal
%        -  Well Def
%        -  Inv.1 
%        -  Inv.2 
%        -  Order.1
%        -  Order.2
%        -  Degree
%        -  Equivariance
%        -  Normality
%        -  Quotient

\begin{thm} Galois Correspondance. 
    
    Let $\iota_L : K \to L$ be a Galois extension.
    Let $E \subseteq L$ be a $K$-subextension of $L$. 
    Then the group of $E$-extension automorphisms of $L$ is
    a subgroup of the Galois group of $(L,\iota_L)$. \[
        \aut{E}{L} \leq_\Grp \aut{K}{L}
    \]
    Let $G \leq \aut{K}{L}$ be a subgroup of the Galois group of $(L,\iota_L)$. 
    Then the subfield fixed by $G$, $L^G$, is a $K$-subextension of $(L,\iota_L)$.
    \[ K \overset{\iota_L}{\to} L^G \]
    
    The \emph{Galois Correspondance} states the following: 
    \begin{enumerate}
        \item (Inverses)
            For a subgroup $G \leq \aut{K}{L}$ \[
                G = \aut{L^G}{L}
            \]
            For a $K$-subextension $E$, \[
                E = L^{\aut{E}{L}}
            \]
            Hence, we have a bijection between $K$-subextensions of $(L,\iota_L)$
            and the subgroups of the Galois group of $(L,\iota_L)$.
        \item (Order Reversing) 
            Let $G, H \leq \aut{K}{L}$ be subgroups such that $G \subseteq H$.
            Then \[ L^H \subseteq L^G \]
            Similarly, 
            let $E, F \subseteq L$ be $K$-subextensions such that $E \subseteq F$.
            Then \[ \aut{F}{L} \subseteq \aut{E}{L} \]
        \item (Degree) 
            Let $E$ be a $K$-subextension of $L$. 
            Then $[E : K] = [\aut{K}{L} : \aut{E}{L}]$
            where the latter is index of subgroups. 
        \item (Action)
            Let $\sigma$ be an element of the Galois group $\aut{K}{L}$.
            Then $E \mapsto \sigma E$ defines a $\aut{K}{L}$-action
            on the $K$-subextensions of $L$. 
            We also have a $\aut{K}{L}$-action 
            on the set of subgroups of $\aut{K}{L}$ via 
            $H \mapsto \sigma H \sigma^{-1}$.
            
            With these two actions, $\aut{-}{L} : E \mapsto \aut{E}{L}$
            becomes a $\aut{K}{L}$-set isomorphism from the $K$-subextensions of $L$
            to the subgroups of the Galois group of $L$. 
        \item (Normality)
            Let $E$ be a $K$-subextension of $L$. \[
                K \overset{\iota_L}{\to} E \overset{\subseteq}{\to} L
            \]
            Then $\iota_L : K \to E$ is a normal extension if and only if
            $\aut{E}{L}$ is a normal subgroup of $\aut{K}{L}$. 
            When this is the case, $\iota_L : K \to E$ is a Galois extension
            and we have the isomorphism of Galois groups \[
                \aut{K}{E} \iso_\Grp \aut{K}{L} / \aut{E}{L}
            \]
    \end{enumerate}
\end{thm}
\begin{proof}
    (Inverses)
        
        Let $G$ be a subgroup of $\aut{K}{L}$. 
        Then $G$ is finite and clearly a subgroup of $\aut{L^G}{L}$. 
        Hence by definition, $L^G \to L$ is Galois 
        and $G = \aut{L^G}{L}$. 
        
        For the other side, let $E \subseteq L$ be a $K$-subextension of $(L,\iota_L)$.
        Then $(L,\iota_L)$ finite, normal and separable implies
        $(E,\subseteq)$ finite, normal separable, i.e. Galois.
        Hence, there exists a finite subgroup $G$ of $\aut{E}{L}$ such that $E = L^G$.
        Then we have \[
            \aut{E}{L} = \aut{L^G}{L} = G
        \]
        i.e. $E = L^{\aut{E}{L}}$. 
        
    (Order Reversing) Left as an easy exercise. 
        
    (Degree) Left as an easy exercise.
    
    (Action) 
        
        We already know $\aut{-}{L}$ is a bijection,
        so we only need to show it respects the $\aut{K}{L}$-actions.
        
        Let $\sigma \in \aut{K}{L}$. Let $E$ be a $K$-subextension of $L$. 
        We need to show \[
            \aut{\sigma E}{L} = \sigma \aut{E}{L} \sigma^{-1}
        \]
        We invite the reader to stare at this
        until this becomes trivial. 
    
    (Normality)
        
        Let $E$ be a $K$-subextension of $L$ that is normal. 
        Then for all $K$-extension automorphisms $\sigma$ of $L$, $\sigma E = E$. 
        Hence by $\aut{-}{L}$ being a morphism of $\aut{-}{L}$-sets,
        for all $\sigma \in \aut{K}{L}$, \[
            \sigma \aut{E}{L} \sigma^{-1} = \aut{\sigma E}{L} = \aut{E}{L}
        \]
        i.e. $\aut{E}{L}$ is a normal subgroup of $\aut{K}{L}$.
        
        Now the more difficult implication. 
        Suppose $E$ is a $K$-subextension of $L$ such that
        $\aut{E}{L}$ is normal. 
        Then for all $K$-extension automorphisms $\sigma$ of $L$, 
        $\sigma \aut{E}{L} \sigma^{-1} = \aut{E}{L}$.
        Hence for all $\sigma \in \aut{K}{L}$, \[
            \sigma E = L^{\aut{\sigma E}{L}} = L^{\aut{E}{L}} = E
        \]
        This looks almost like the image invariance definition of normality,
        but only for $K$-extension automorphisms of $L$. 
        We will prove $E$ has the image invariance property as a $K$-extension. 
        
        Let $\iota_M : K \to M$ be a $K$-extension and 
        $f, g$ be $K$-extension morphisms from $(E,\iota_L)$ to $(M,\iota_M)$.
        The following diagram presents the proof. 
        \begin{figure} [H]
            \centering
            \begin{tikzcd} [sep = huge]
            K \arrow[r,"\iota_L"] \arrow[rd,"\iota_M"{swap}] &
            E \arrow[r,"\iota_{N(E)}"] \arrow[d,"f"{swap},xshift=-0.5ex] 
            \arrow[d,"g",xshift=0.5ex] &
            N(E) \arrow[r,"\iota"] \arrow[d,"\bar{f}"{swap},xshift=-0.5ex] 
            \arrow[d,"\bar{g}",xshift=0.5ex] 
            \arrow[loop above,"\sigma := \bar{g}^{-1}\circ\bar{f}"] &
            L \arrow[loop above,"\bar{\sigma}"] \\
            & 
            M \arrow[r,"\iota_{N(M)}"] & 
            N(M) & 
            \end{tikzcd}
        \end{figure}
        We do not know whether $E$ has the image invariance property,
        but we do know its normal closure does. 
        Let $(N(E),\iota_{N(E)})$ be the normal closure of $(E,\iota_L)$. 
        Since $L$ is an $E$-extension that is normal as a $K$-extension,
        we have an $E$-extension morphism 
        $\iota : (N(E),\iota_{N(E)}) \to (L,id_E)$,
        i.e. $\iota\circ\iota_{N(E)} = id_E$ 
        where $id_E$ is the inclusion of $E$ into $L$.  
        
        We want to use image invariance on $N(E)$ as a $K$-extension,
        but the maps $f, g$ are from $E$. 
        To "lift" these maps up to $N(E)$, 
        let $(N(M),\iota_{N(M)})$ be the normal closure of $(M,\iota_M)$. 
        Then $(N(M),\iota_{N(M)}\circ f)$ and $(N(M),\iota_{N(M)}\circ g)$
        are both $E$-extensions that are normal as $K$-extensions.
        Hence by minimality of the normal closure of $E$, 
        we have $E$-extension morphisms 
        $\bar{f} : (N(E),\iota_{N(E)}) \to (N(M),\iota_{N(M)}\circ f)$
        and $\bar{g} : (N(E),\iota_{N(E)}) \to (N(M),\iota_{N(M)}\circ g)$.
        Then $\bar{f}$ and $\bar{g}$ are clearly $K$-extension morphisms
        from $N(E)$ to $N(M)$,
        so by image invariance, \[ \bar{f} N(E) = \bar{g} N(E) \]
        Then $\sigma := \bar{g}^{-1} \circ \bar{f}$ is well-defined as a function.
        This gives $\sigma$ as a $K$-extension automorphism of $N(E)$. 
        In particular, $id_E$ and $\iota\circ\sigma\circ\iota_{N(E)}$ 
        are two $K$-extension morphisms from $(E,\iota_L)$ to $(L,\iota_L)$.
        So by embeddings into normal extensions differing by an automorphism, 
        we have an $E$-extension morphism $\bar{\sigma}$
        from $(L,id_E)$ to $(L,\iota\circ\sigma\circ\iota_{N(E)})$.
        We have turned the two $K$-extension morphisms $f, g$ into 
        a $K$-extension automorphism of $L$!
        Of course, $\bar{\sigma} E = E$, which implies \begin{align*}
            (\iota\circ\sigma\circ\iota_{N(E)}) E = (\iota\circ\iota_{N(E)}) E &\imp
            (\bar{g}^{-1}\circ\bar{f}\circ\iota_{N(E)}) E 
            = (\sigma\circ\iota_{N^(E)})E = \iota_{N(E)} E \\
            &\imp (\bar{f}\circ\iota_{N(E)}) E = (\bar{g}\circ\iota_{N(E)}) E \\
            &\imp (\iota_{N(M)}\circ f) E = (\iota_{N(M)}\circ g) E \\
            &\imp f E = g E
        \end{align*}
        Thus, $(E,\iota_L)$ has the image invariance property as a $K$-extension. 
        
        Furthermore, $\iota_L : K \to L$ finite and separable implies 
        $\iota_L : K \to E$ also finite and separable,
        so $(E,\iota_L)$ is a Galois extension. 
        
        We leave the isomorphism of Galois groups as an exercise to the reader. 
        [Hint : Use the first isomorphism theorem, remembering to prove surjectivity.]
\end{proof}

Here is an alternative proof for the reverse implication of Normality
that is more straight forward.

\begin{proof}
    
        Suppose $E$ is a $K$-subextension of $L$ such that 
        $\aut{E}{L}$ is a normal subgroup of $\aut{K}{L}$.
        Then for all $K$-extension automorphisms $\sigma$ of $L$, 
        $\aut{\sigma E}{L} = \sigma \aut{E}{L} \sigma^{-1} = \aut{E}{L}$.
        Hence for all $\sigma \in \aut{K}{L}$, \[
            \sigma E = L^{\aut{\sigma E}{L}} = L^{\aut{E}{L}} = E
        \]
        
        Let $a$ be in $E$. 
        Recall that we proved for Galois extensions $\iota_L : K \to L$,
        $\bar{\iota_L} \min(a,K) = \prod_{b \in Orb(a)} (X - b)$. 
        This implies each $b = \sigma(a) \in \sigma E$
        for some $K$-extension automorphism $\sigma$.
        Hence all $b \in Orb(a)$ lies in $E$,
        that is to say, $\min(a,K)$ splits in $E$ and $K \to E$ is normal. 
        
    
\end{proof}

\end{document}