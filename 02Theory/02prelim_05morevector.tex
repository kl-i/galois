\documentclass[../book.tex]{subfiles}
\begin{document}
% Direct Sum of Vector Spaces
\begin{dfn} Direct Sum of Vector Spaces.
    
    The idea is to be able to formally add vectors from different vector spaces
    together in some larger vector space.
    Let $(V_i)_{i\in I}$ be a collection of $K$-vector spaces.
    The \textbf{direct sum}, $\bigoplus_{i\in I}V_i$, 
    will contain linear combinations of vectors from the $(V_i)_{i\in I}$.
    It is formalised as follows. 
    Let $\bigsqcup_{i\in I} V_i$ be the disjoint union 
    of the collection of $K$-vector spaces.
    Define the direct sum to be \[
        \bigoplus_{i\in I} V_i := \{v : I \to \bigsqcup_{i\in I} V_i \mid 
        v(i) \in V_i, \supp v \text{ finite}\}
    \]
    Then this is naturally a $K$-vector space under pointwise addition and scaling.
    For all $i \in I$, we have a natural injective morphism of $K$-vector spaces 
    $\iota_i : V_i \to \bigoplus_{i\in I} V_i$ by interpreting
    $v \in V_i$ as the function \[
        v : I \to \bigsqcup_{i\in I} V_i, j \mapsto \begin{cases}
            v   &, j = i\\
            
            0   &, j \neq i
        \end{cases}
    \]
    Then, as desired, all vectors $v$ in the direct sum can be written as
    a linear combination \[
        v = \sum_{i \in I} v_i , v_i = v(i) \in V_i
    \]
\end{dfn}

\begin{thm} The ``Characterising" Property of the Direct Sum.
    
    Let $(V_i)_{i\in I}$ be a collection of $K$-vector spaces.
    Then the direct sum is the \emph{smallest} vector space 
    containing $(V_i)_{i\in I}$ in the sense that
    for any $K$-vector space $W$ and 
    $(f_i : V_i \to W)_{i\in I}$ $K$-vector space morphisms,
    there exists a unique morphism of $K$-vector spaces 
    $f : \bigoplus_{i\in I} V_i \to W$ such that for all $i\in I$, 
    $f_i = f \circ \iota_i$. 
    I.e. the diagram below commutes for all $i \in I$.  
    \begin{figure} [ht]
        \centering
        \begin{tikzcd}
        & \bigoplus_{i\in I} V_i \arrow[dd,"\exists ! f ",dashed]\\
        V_i \arrow{ru}{\iota_i} \arrow{rd}{f_i} & \\
        & W \\
        \end{tikzcd}
    \end{figure}

    
    Hence if $W$ with $(f_i : V_i \to W)_{i\in I}$ also has the above property,
    $f$ becomes an \emph{isomorphism} between the direct sum and $W$.
    So the above property completely characterises the direct sum. 
\end{thm}
\begin{proof}
    Let $W$ be a $K$-vector space and 
    $(f_i : V_i \to W)_{i\in I}$ $K$-vector space morphisms.
    Let $v$ be a vector in the direct sum, 
    i.e. $v = \sum_{i\in I} v_i$ is a finite sum where $v_i \in V_i$. 
    $f$ being a $K$-vector space morphism and $f \circ \iota_i = f_i$ implies \[
        f(v) = \sum_{i\in I} f(v_i) = \sum_{i\in I} f_i(v_i)
    \]
    This serves as a definition of $f$ and proves uniqueness of $f$. 
    
    Note that since the direct sum itself trivially has the above property, 
    there is a unique morphism from it to itself commuting with all $\iota_i$,
    namely the identity morphism of the direct sum. 
    
    Suppose $W$ also has the above property. 
    Then there exists a unique morphism $g : W \to \bigoplus_{i\in I} V_i$
    such that all $g \circ f_i = \iota_i$. 
    Then it is easily verified that $g \circ f$ is a morphism 
    from the direct sum to itself that commutes with all $\iota_i$.
    But we know this must be the identity, hence $g \circ f = id$. 
    Similarly, $f \circ g = id$. 
    So $\bigoplus_{i\in I} V_i \iso W$ as $K$-vector spaces via $f$. 
\end{proof}
% Free Vector Space as direct sum indexed by set
\begin{ex} Free Vector Space is a Direct Sum.
    
    Let $S$ be a set. 
    Show that $\<S\> \iso_{\KVec} \bigoplus_{s\in S} K$ where the latter is
    the direct sum of copies of $K$ indexed by $S$. 
    This gives another construction for the free vector space over a set. 
    
    In particular, $K^n \iso K \oplus \cdots \oplus K$ $n$-times. 
\end{ex}

% Linear Indep, Spanning, Basis as Inject, Surject, Biject
\begin{dfn} Linear Independence, Spanning, Basis.
    
    Let $V$ be a $K$-vector space and $\{v_i\}_{i\in I}$ a subset of vectors.
    Then via the inclusion $\iota : \{v_i\}_{i\in I} \to V$ and
    the characterising property of free vector spaces, 
    we have a $K$-vector space morphism \[
        \bar\iota : \bigoplus_{\{v_i \mid i\in I\}} K \to V, 
        \sum_{i\in I}\la_i v_i \mapsto \sum_{i\in I}\la_i v_i
    \]
    Note that this makes sense because we interpret elements of 
    $\bigoplus_{\{v_i \mid i\in I\}} K$ as elements of $\<S\>$. 
    Keep in mind that linear combinations are finite. 
    Then \begin{enumerate}
        \item If $\bar\iota$ injects,
        the subset of vectors $\{v_i\}_{i\in I}$ 
        is called \textbf{$K$-linearly independent}. 
        Equivalently, 
        $\sum_{i\in I} \la_i v_i = 0 \in V$ implies for all $i \in I$, $\la_i = 0$.
        \item If $\bar\iota$ surjects, 
        the subset of vectors $\{v_i\}_{i\in I}$ is called \textbf{spanning}.
        Equivalently, every vector is a linear combination of $\{v_i\}_{i\in I}$. 
        \item If $\bar\iota$ bijects,
        the subset of vectors $\{v_i\}_{i\in I}$ is called a \textbf{$K$-basis of $V$}.
    \end{enumerate}
    If the underlying field is clear, we omit the prefix ``$K$-". 
    
    These three definitions are a sort of way of measuring the 
    ``vector space data" of a subset of vectors.
    Linearly independent is no-redundant data, spanning is sufficient data, and
    a basis is just the right amount. 
\end{dfn}

\begin{dfn} Dimension of a Vector Space.

    Let $\{v_i\}_{i\in I}$ be a basis of $V$. 
    Note that this is equivalent to $V$ being a free vector space, namely, 
    \[
        V \iso \bigoplus_{\{v_i \mid i\in I\}} K \iso \<\{v_i\}_{i\in I}\> 
    \]
    Then the \textbf{dimension of $V$} is defined as the cardinality of the basis
    and is denoted, $\dim_K V$. 
    
    We will only prove the well-definedness of dimension for finite-bases
    \footnote{Bases is plural for basis.}, 
    since we will only need finite-dimensional vector spaces for Galois theory
    and the proof for general cardinality bases is somewhat tricky. 
\end{dfn}
% Steinitz Exchange / Dimension is Well-defined
\begin{lem} Steinitz Exchange / Finite Dimenion is Well-Defined. 

    Let $S = \{v_i\}_{i=1}^n, T = \{w_j\}_{j=1}^m$ be finite subsets 
    of a $K$-vector space $V$ such that 
    $S$ is linearly independent and $T$ is spanning. 
    Then $|S| \leq |T|$.
    
    Hence, given two finite bases of $V$, they have the same cardinality. 
    Also, should $V$ have a finite bases,
    finite linearly independent sets have less than or equal cardinality
    than the dimension.
\end{lem}
\begin{proof}
    The idea is to swap out vectors in $T$ by vectors in $S$, 
    creating an injection from $S$ to $T$. 
    ...
\end{proof}
\begin{cor} Morphisms between Finite Vector Spaces 
are like Functions between Finite Sets. 

    Let $f : V \to W$ be a $K$-vector space morphism and $\dim V = \dim W$ finite.
    Then $f$ injects if and only if $f$ surjects, i.e.
    either is automatically a bijection. 
\end{cor}
\begin{proof}
    WLOG $V = K^{\dim V}$ and $W = K^{\dim W}$. 

    Assume $f$ is injective. 
    Suppose $f$ is not surjective, i.e.
    there exists a vector $w \in W$ that is not in the image of $f$. 
    We will show that $\dim V < \dim W$ and derive a contradiction. 
    Since $f$ is a $K$-vector space morphism and 
    $\{i\}_{i\in\dim V}$ is a basis of $V$, 
    the image of $f$ is equal to the subspace generated by $\{f(i)\}_{i\in\dim V}$.
    So $w \notin \im f$ implies 
    $w$ cannot be a linear combination of $\{f(i)\}_{i\in\dim V}$.
    We now claim that $\{f(i)\}_{i\in\dim V} \cup \{w\}$ is linearly independent. 
    Let $\la w + \sum_{i\in I} \la_i f(i) = 0$. 
    If $\la = 0$, then by linear independence of $\{f(i)\}_{i\in I}$, all $\la_i = 0$. If $\la \neq 0$, then $\la \in K^\times$,
    so $w = \sum_{i\in I} (-\la_i/\la)f(i)$, 
    which says $w$ is in $\im f$, a contradiction. 
    This proves $\{f(i)\}_{i\in\dim V} \cup \{w\}$ is a linearly independent set.
    Hence by Steinitz exchange, $\dim V + 1 \leq W$,
    which gives $\dim V < \dim W$, the desired contradiction. 
    
    We leave the reverse implication as an exercise for the reader. 
    (Hint : Already, $\{f(i)\}_{i\in\dim V}$ is spanning. 
    Show that it is linearly independent by assuming not and 
    deriving $\dim W \leq \dim V - 1$ via Steinitz Exchange, 
    contradicting with $\dim W = \dim V$.)
\end{proof}
\begin{cor} Arbitrary Linearly Independent Sets are smaller than Finite Dimension.
    
    Let $V$ be a finite dimensional $K$-vector space and
    $S$ a linearly independent subset of $V$, potentially infinite.
    
    Then $S$ is finite and $|S| \leq \dim V$. 
    In particular, if $f : W \to V$ is an injective $K$-vector space morphism
    then $\dim W$ must also be finite and less than or equal to $\dim V$.
\end{cor}
\begin{proof}
    Suppose $S$ is infinite. Then $S$ contains a countably infinite subset $T$.
    $S$ linearly independent implies any finite subset linearly independent.
    In particular, we can pick a subset of $\dim V + 1$ from $T$,
    which will be linearly independent. 
    This is contradiction since in finite dimensional vector spaces,
    finite linearly independent sets 
    have less than or equal cardinality than the dimension.
    
    Thus, $S$ must be finite and $|S| \leq \dim V$ follows from Steinitz exchange.
    
    Let $f : W \to V$ be an injective $K$-vector space morphism. 
    Let $\{w_i\}_{i\in\dim W}$ be a basis of $W$. 
    Then injectivity of $f$ implies $\{w_i\}_{i\in\dim W}$ bijects with
    $\{f(w_i)\}_{i\in\dim W}$.
    But the latter is clearly a linearly independent set of vectors in $V$.
    So $\{f(w_i)\}_{i\in\dim W}$ must be finite and 
    have less than or equal cardinality than $\dim V$.
    Hence $\dim W \leq \dim V$. 
\end{proof}

\end{document}