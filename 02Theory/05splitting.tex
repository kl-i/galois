
\documentclass[../book.tex]{subfiles}

\begin{document}
%    - Def - Splits.
\begin{dfn} Extension Splitting a Polynomial. 
    
    Let $\iota : K \to L$ be a field extension, $f$ a polynomial over $K$. 
    Then \textbf{$(L,\iota)$ splits $f$} when there exist degree 1 polynomials
    $(r_i)_{i\in\deg f} \in L[X]$ such that \[
        \bar\iota f = \prod_{i\in\deg f} r_i
    \]
    Equivalently, there exists elements $(a_i)_{i\in\deg f} \in L, \la \in K$ 
    such that \[
        \bar\iota f = \iota(\la)\prod_{i\in\deg f} (X - a_i)
    \]
    If the extension is clear, we just say $L$ splits $f$ or
    $L$ contains all the roots of $f$. 
\end{dfn}

\begin{ex} Degree Bounds Number of Roots. 

    Let $f$ be a polynomial over a field $K$.
    Show that an element $a \in K$ is a root if and only if $(X-a) \mid f$.
    Let $S_f := \{a \in K \mid f(a) = 0\}$ be the set of roots of $f$. 
    Hence show that $|S_f| \leq \deg f$.
    
    This shows that for the definition of a polynomial splitting, 
    we did not have to assume it factorises into exactly $\deg f$ many 
    degree 1 polynomials. 
\end{ex}

\begin{lem} Splitting Minimal Polynomials.
    
    Let $K \overset{\iota_0}{\to} L \overset{\iota_1}{\to} M$ be field extensions,
    $a \in M$ algebraic over $K$. 
    Then $M$ splits $\min(a,K)$ implies $M$ splits $\min(a,L)$.
\end{lem}
\begin{proof}
    We have the following situation. 
    \begin{figure} [H]
        \centering
        \begin{tikzcd} 
            K[X] \arrow{r}{\bar\iota_0} & 
            L[X] \arrow{r}{\bar\iota_1} & 
            M[X] \arrow{d}{ev_a} \\
            K \arrow{r}{\iota_0} & 
            L \arrow{r}{\iota_1} & 
            M
        \end{tikzcd}
    \end{figure}
    By definition, $\bar\iota_1 (\bar\iota_0 \min(a,K))(a)=0$ i.e. 
    $\bar\iota_0 \min(a,K)$ is a polynomial over $L$ that has $a$ as a root. 
    Thus $min(a,L)$ is well-defined,
    and $\min(a,L) \mid \bar\iota_0 \min(a, K)$,
    which implies $\bar\iota_1 \min(a,L) \mid \bar\iota_1 (\bar\iota_0 \min(a,K))$.
    So there exists a polynomial $f$ over $M$ such that \[
        f \bar\iota_1 \min(a,L) = \bar\iota_1(\bar\iota_0 \min(a,K))
    \]
    Since $M$ splits $\min(a,K)$, \[
        f \bar\iota_1 \min(a,L) =
        \bar\iota_1(\bar\iota_0 \min(a,K)) = \prod_{\tiny{i \in \deg\min(a,K)}} r_i
    \]
    where $r_i \in M[X]$ are polynomials of degree 1. 
    
    Clearly, the degree 1 polynomials $r_i$ are irreducible. 
    So by unique factorisation, there is a subset $I \subseteq \deg\min(a,K)$
    such that \[
        \bar\iota_1 \min(a,L) = \prod_{i \in I} r_i
    \]
    where the $r_i$ may have been reordered and scaled by units.
    Clearly, the $r_i$ are still degree 1. 
    To see that $|I| = \deg min(a,L)$, note that $I$ is finite and apply
    degree of product is sum of degrees. 
\end{proof}
\begin{rmk}
    Previously, we showed that for an extension $\iota : K \to L$ with 
    $a \in L$ algebraic over $K$, 
    $K(a) \iso K[X]/(\min(a,K))$ where $\min(a,K)$ is irreducible.
    This assumed the existence of a larger field $L$ inside which $a$ sits. 
    We now do the reverse: 
    given a (monic) irreducible polynomial $f$ over $K$, 
    we will create a larger field in which 
    we have an element $a$ which has $f$ as its minimal polynomial. 
\end{rmk}
%    - Lem - Quotient Ideal gen by Irr is Field via Irr Div or Coprime
\begin{lem} Quotienting by an Irreducible Element gives a Field. 

    Let $f$ be an irreducible polynomial over $K$. 
    Then the quotient $K[X]/(f)$ is a field and hence a finite extension of $K$. 
    
    For $g \in K[X]$, let $\bar{g}$ denote the image of $g$ in the quotient.
    Then $f = \la \min(\bar{X},K)$ where $\la \in K^\times$ 
    ($X$ being the polynomial in $K[X]$). 
    In particular, if $f$ was monic, it would be the minimal polynomial of $\bar{X}$.
\end{lem}
\begin{proof}
    Let $\bar{g} \in K[X]/(f)$. 
    Then by irreducibility of $f$, $f \mid g$ or $(f,g) = 1$. 
    If $f \mid g$, then $g \in (f)$ and hence $\bar{g} = \bar{0}$ in the quotient.
    If $(f, g) = 1$, then there exists polynomials $\al, \be \in K[X]$ 
    such that $\al f + \be g = 1$. 
    Thus by projecting to the quotient, $\bar{\be}\bar{g} = \bar{1}$, 
    i.e. $\bar{g}$ is a unit in the quotient. 
    We just showed that an element in the quotient is either zero or a unit.
    So $K[X]/(f)$ is a field. 
    By the division algorithm, $\{\bar{1}, \bar{X}, \dots, \bar{X}^{\deg f - 1}\}$
    is a basis of $K[X]/(f)$ as a $K$-vector space
    so this is a finite extension of $K$. 
    
    It is easy to show that $f$ is a constant multiple of
    the minimal polynomial of $\bar{X} \in K[X]/(f)$. 
\end{proof}
\begin{ex} Alternative Proof. 
    
    Let $f$ be a irreducible polynomial over $K$, $(f)$ the ideal generated by $f$. 
    Show that for any ideal $I$ that contains $(f)$, $I = (f)$ or $I = K[X]$.
    Hence deduce from the 3rd isomorphism theorem that $K[X]/(f)$ is a field.
    
    In general, ideals such that the only ideals containing it are
    itself and the entire ring are called \textbf{maximal ideals}.
    These are precisely the ideals whose quotient ring are fields. 
\end{ex}
\begin{eg}
    The polynomial $X^2 + 1$ is irreducible over $\R$.
    Taking the quotient ring, let $i$ denote $\bar{X}$ in $\R[X]/(X^2 + 1)$. 
    Then the quotient is what we call $\C$. 
\end{eg}
\begin{rmk}
    The above lemma is an example of how quotients are used to ``set things to zero",
    forcing desired properties.
    Specifically, we took a polynomial $f$ which previously is not equal to zero,
    and we set it to zero, thus making $X$ a root of $f$. 
\end{rmk}
\begin{rmk}
    We now show that for any polynomial, there is an extension where it splits. 
\end{rmk}
%    - Thm - Existence of Fields that split any polynomial 
\begin{thm} Existence of Finite Extensions Splitting any Polynomial. 
    
    Let $f$ be a polynomial over a field $K$. 
    Then there exists a finite extension $\iota : K \to L$ that splits $f$.
\end{thm}
\begin{proof}
    The idea is to repeatedly apply our previous lemma.
    We proceed by induction on the degree of $f$. 
    Assume the theorem is true for polynomials with degree less than $n$. 
    Let $\deg f = n$.
    By unique factorisation, there exists 
    irreducible polynomials $r_1, \dots, r_m$ over $K$ such that $f = r_1\cdots r_m$.
    By applying the previous lemma on $r_1$, 
    there exists a finite extension $\iota_0 : K \to L$ such that
    there is a root of $r_1$, $a_1$. 
    In other words, $(X - a_1) \mid \bar\iota r_1$, 
    which gives $(X - a_1) \mid \bar\iota_0 f$, i.e. \[
        \bar\iota_0 f = (X - a_1) g
    \]
    Clearly, $\deg g < n$. So by assumption, there exists 
    a finite extension $\iota_1 : L \to M$ such that $M$ splits $g$.
    Then $M$ splits $f$ and is a finite extension by the Tower Law.
    This completes the induction.
\end{proof}
%    - Def - Splitting Field of Poly as Field that splits Poly and gen by Roots
\begin{dfn} Splitting Field of a Polynomial. 
    
    Let $f$ be a polynomial over a field $K$. 
    A $K$-extension $\iota : K \to L$ (or simply $L$) is called 
    a \textbf{splitting field of $f$} when
    $L$ splits $f$ and $L$ is generated by the roots of $f$. 
    
    Note that splitting fields are automatically finite extensions. 
\end{dfn}
\begin{eg}
    Let $f = X^2 - 2$ the polynomial over $\Q$. 
    Then the extension $\Q \to \R$ is \emph{not} a splitting field of $f$
    whilst the extension $\Q \to \Q(\sqrt{2})$ is. 
\end{eg}
\begin{rmk}
    We will now proceed to show that a splitting field of a polynomial 
    is in a sense the \emph{smallest} field that splits that polynomial.
\end{rmk}
%    - Def - Galois Conjugates
\begin{dfn} Galois Conjugates.

    Let $\iota_L : K \to L, \iota_M : K \to M$ be $K$-extensions and
    $a \in L, b \in M$ be elements algebraic over $K$. 
    Then $a, b$ are called \textbf{galois conjugates} when
    either of the following equivalent statements are true: 
    \begin{enumerate}
        \item $b$ is a root of $\min(a,K)$.
        \item $\min(a,K) = \min(b,K)$.
    \end{enumerate}
    This forms an equivalence relation on algebraic elements in $K$-extensions.
\end{dfn}
\begin{eg} Conjugates
    \begin{enumerate}
        \item Let $\R \to \C$ be the usual extension.
        Then $i$ and $-i$ are galois conjugates.
        \item Let $\Q \to \Q(\sqrt{2})$ be the usual extension.
        Then $\sqrt{2}$ and $-\sqrt{2}$ are conjugates.
        \item Let $\Q \to \Q(\omega)$ where $\omega = \exp(2\pi i / 3)$.
        Then $1$ and $\omega$ are \emph{not} galois conjugates even though
        both are roots of $X^3 - 1$. 
        The mistake is that $X^3 - 1$ is not the minimal polynomial.
        The minimal polynomial of $\omega$ is $X^2 + X + 1$,
        and the minimal polynomial of $1$ is $X-1$.
    \end{enumerate}
\end{eg}
%    - Lem - Embedding of Simple Ext to Field with Conjugate
\begin{lem} Embedding Simple Extensions via Conjugates.\footnote{
    The word "embedding" is synonymous to injecting/injection.}
    
    Let $\iota : K \to K(a)$ be a simple $K$-extension
    where $a$ is algebraic over $K$. 
    Let $\iota_M : K \to M$ be another $K$-extension.
    Then for any $K$-extension morphism $f : K(a) \to M$, 
    $f(a)$ is a galois conjugate of $a$.
    Hence the map $f \mapsto f(a)$ gives a bijection between
    the $K$-extension morphisms $K(a) \to M$ and the galois conjugates of $a$ in $M$.
    \[
        \{f : K(a) \to M \mid f \text{ $K$-extension morphism} \} \leftrightarrow 
        \{b \in M \mid b \text{ galois conjugate to } a \}
    \]
    In particular, the number of such morphisms equals $\deg \min(a,K) = [K(a) : K]$ 
    if and only if $M$ contains all the roots of $\min(a,K)$ and they are distinct. 
\end{lem}
\begin{proof}
    Let $f : K(a) \to M$ be a $K$-extension morphism. 
    Then the fact that $f(a)$ is a galois conjugate of $a$ just comes from
    the property of the evaluation morphism, 
    \[
        \bar\iota_M \min(a,K) (f(a)) = ev_{f(a)}(\bar\iota_M \min(a,K))
        = ev_{f(a)} (\bar{f}(\bar\iota \min(a,K))) 
        = f(ev_a (\bar\iota \min(a,K))) = 0
    \]
    So the function $f \mapsto f(a)$ is well-defined. 
    
    To prove injectivity, note that $K(a)$ being a simple extension with
    $a$ algebraic implies $\{a^i\}_{i\in\deg\min(a,K)}$ is a basis of $K(a)$. 
    $f : K(a) \to M$ is $K$-extension morphism implies 
    it is a $K$-vector space morphism. 
    Hence $f$ is determined by its image on the basis elements. 
    What's more, since $f$ is a ring morphism, 
    $f(a^i) = f(a)^i$ implies $f$ is determined entirely by $f(a)$. 
    This proves injectivity. 
    
    Surjectivity is a consequence of the form of $K(a)$.
    Let $b \in M$ be a galois conjugate of $a$. 
    Then $\min(b,K) = \min(a,K)$ implies \[
        K(a) \iso K[X]/(\min(a,K)) = K[X]/(\min(b,K)) \iso K(b)
    \]
    This composition of ring isomorphisms is a $K$-extension morphism 
    $f : K(a) \to K(b) \subseteq M$ that maps $a$ to $b$. 
\end{proof}
\begin{rmk}
    We extend the previous result to finite extensions. 
\end{rmk}
%    - Thm - Embedding of Fin Ext into Field with Conjugates
\begin{thm} Embedding Finite Extensions via Conjugates.
    
    Let $\iota_L : K \to L$ be a finite extension, i.e. (by picking a basis)
    there exists a finite set of elements $\{a_1,\dots,a_n\} \subseteq L$ such that 
    $L = K(a_1, \dots, a_n)$. 
    Let $\iota_M : K \to M$ be another extension, such that 
    for all $a_i$ generators of $L$, $M$ splits $\min(a_i,K)$. 
    Then there exists a morphism of $K$-extensions $\iota : L \to M$.
    Furthermore, the number of such $K$-extension morphisms is less than equal
    to $[L : K]$ and is equal when for all generators $a_i$, 
    $\min(a_i,K)$ has distinct roots in $M$. 
\end{thm}
\begin{proof}
    For $1 \leq i \leq n$, let $K_i$ denote $K(a_1,\dots,a_i)$.
    The following diagram is the idea of the proof. 
    \begin{figure}[H]
        \centering
        \begin{tikzcd}[sep = huge]
        K \arrow{d}{\iota_M} \arrow{r}{\iota_1 = \iota_L} & 
        K(a_1) = K_1 \arrow{dl} \arrow{r}{\iota_2 = \subseteq} & 
        \cdots \arrow{r}{\iota_{n-1} = \subseteq} & 
        K_{n-1} \arrow{dlll}[sloped,above]{\iota_{M,n-1}} 
        \arrow{r}{\iota_n = \subseteq} & 
        K_{n-1}(a_n) = K_n = L \arrow{dllll}{\iota} \\
        M & & & &
        \end{tikzcd}
    \end{figure}
    Since $K_{i-1}(a_i) = K_i$, 
    we can break the extension $\iota_L : K \to L$ into a chain of simple extensions
    and proceed by induction on the number of generators. 
    The base case of one generator, i.e. a simple extension, 
    is covered by the previous lemma. 
    
    Assume the theorem is true for $n-1$ generators. 
    Let $\iota_{M,n-1}$ be one of the $K$-extension morphisms from
    $K_{n-1}$ to $M$. 
    Since $M$ splits $\min(a_n,K)$, it also splits $\min(a_n,K_{n-1})$.
    So by applying the previous lemma to the simple extension
    $\iota_n : K_{n-1} \to K_n$,
    we obtain a $K_{n-1}$-extension morphism $\iota : K_n \to M$.
    $\iota$ is clearly a $K$-extension morphism and
    there are less than or equal $[K_n : K_{n-1}]$ of $\iota$'s,
    equal if $\min(a_n,K_{n-1})$ has distinct roots in $M$. 
    Since we had at most $[K_{n-1} : K]$ many $\iota_{M,n-1}$
    to choose from, by the Tower Law, we have at most \[
        [K_n : K_{n-1}][K_{n-1} : K] = [K_n : K]
    \]
    many morphisms $\iota$ arising in this way.
    
    We show that all $K$-extension morphisms from $K_n$ to $M$
    arise from applying the previous lemma to $K_{n-1} \to K_n$. 
    Let $\iota : K_n \to M$ be a $K$-extension morphism. 
    This gives a $K_{n-1}$-extension $\res{\iota}{K_{n-1}} : K_{n-1} \to M$ 
    by restricting to the subfield $K_{n-1}$.
    So the restriction $\res{\iota}{K_{n-1}}$ must be 
    one of the at most $[K_{n-1} : K]$ many morphisms of $K$-extensions
    and $\iota$ becomes one of the at most $[K_n : K_{n-1}]$ many 
    morphisms of $K_{n-1}$-extensions from 
    $\iota_n : K_{n-1} \to K_n$ to $\res{\iota}{K_{n-1}} : K_{n-1} \to M$.
    Thus it must be one of the at most $[K_n : K]$ many morphisms we already have. 
    
    Suppose for all generators $a_m$, $\min(a_m,K)$ has distinct roots in $M$.
    Clearly, $\min(a_n,K)$ having distinct roots in $M$ implies 
    $\min(a_n,K_{n-1})$ has distinct roots in $M$,
    and hence we have exactly $[L : K]$ many $\iota$'s.
    This concludes the induction.
\end{proof}
%    - Thm - Min Prop of Splitting Field
\begin{thm} Minimal Property of Splitting Field. 
    
    Let $f$ be a polynomial over $K$ and 
    $\iota_L : K \to L$ a splitting field of $f$.
    Then $L$ is the \emph{smallest} extension splitting $f$, in the sense that 
    for any $K$-extension $\iota_M : K \to M$ that splits $f$, 
    there exists a $K$-extension morphism $\bar\iota_M : L \to M$. 
    Diagrammatically, 
    \begin{figure}[H]
        \centering
        \begin{tikzcd}
        &
        L \arrow[dd,"\bar\iota_M"]
        \\
        K \arrow[ru,"\iota_L"{sloped,above}] \arrow[rd,"\iota_M"{sloped,below}] &
        \\
        &
        M
        \end{tikzcd}
    \end{figure}
    In particular, we also have uniqueness up to isomorphism, that is
    if $M$ is also a splitting field of $f$, 
    then $M,L$ are isomorphic as $K$-extensions.
    
\end{thm}
\begin{proof}
    Let $\iota_M : K \to M$ be an extension that splits $f$. 
    For any $a$ root of $f$, $\min(a,K) \mid f$ implies $M$ splits $\min(a,K)$.
    So by embedding finite extensions via conjugates,
    we get a morphism of $K$-extensions $\bar\iota_M : L \to M$. 
    In particular, $\bar\iota_M$ is an injective morphism of $K$-vector spaces.
    If $M$ is a splitting field of $f$, then it is a finite extension.
    So $\dim_K L \leq \dim_K M$. 
    By the same argument on $M$, $\dim_K M \leq \dim_K L$.
    Hence $\bar\iota_M$ must be bijective,
    implying it is an isomorphism of $K$-extensions. 
\end{proof}
\begin{rmk}
    We conclude with a very special property of splitting fields,
    which we will further explore in the next section. 
\end{rmk}
%    - Thm - Invariant Image of Splitting Field (Secretly Normality)
\begin{thm} Image Invariance of Splitting Fields. 
    
    Let $f$ be a polynomial over $K$ and 
    $\iota_L : K \to L$ the splitting field of $f$.
    Then for all $K$-extensions $\iota_M : K \to M$ and
    $K$-extension morphisms $\phi, \psi : L \to M$, $\phi L = \psi L$. 
\end{thm}
\begin{proof}
    Let $\{a_i\}_{i\in\deg f}$ be the roots of $f$ in $L$. 
    Since $L$ is the splitting field of $f$, $\{a_i\}_{i\in\deg f}$ generates $L$,
    and hence $\{\phi(a_i)\}_{i\in\deg f}$ generates $\phi L$. 
    By the same argument, $\{\psi(a_i)\}_{i\in\deg f}$ generates $\psi L$. 
    We will show that the two sets of generators are the same.
    
    Since the following argument is completely symmetrical, it suffices to show
    that for all $i \in \deg f$, there exists $j \in \deg f$ such that
    $\phi(a_i) = \psi(a_j)$.
    This is surprisingly just a consequence
    of the basic properties of the evaluation morphism and 
    induced ring morphisms on polynomial rings.
    The situation is this, 
    \begin{figure}[H]
        \centering
        \begin{tikzcd}
        & 
        L \arrow[dd,xshift = 1ex,"\phi" description] 
        \arrow[dd,xshift = -1ex,"\psi" description] & 
        L[X] \arrow[dd,xshift = -1ex,"\bar\phi" description] 
        \arrow[dd,xshift = 1ex,"\bar\psi" description] 
        \arrow[l,"ev_{a_i}"] & 
        \\
        K \arrow{ru}[sloped,above]{\iota_L} \arrow{rd}[sloped,below]{\iota_M} &
        &
        & 
        K[X] \arrow{lu}[sloped,above]{\bar\iota_L} 
        \arrow{ld}[sloped,below]{\bar\iota_M} \\
        & 
        M & 
        M[X] \arrow[l,"ev_{\phi(a_i)}"] &
        \end{tikzcd}
    \end{figure}
    By definition, $L$ splits $f$, that is, there exists a $\la \in K$ such that 
    \[\bar\iota_L f = \iota_L(\la) \prod_{i\in\deg f}(X - a_i)\]
    This implies \[
        (\overline{\psi \circ \iota})(f) 
        = \bar\psi(\bar\iota(f))
        = \psi(\iota_L \la) \prod_{i\in\deg f} (X-\psi(a_i))
    \]
    Now since $\phi \circ ev_{a_i} = ev_{\phi(a_i)} \circ \bar\phi$
    and $a_i$ is a root of $f$, we have \[
        0 = \phi(0) = \phi (ev_{a_i} (\bar\iota_L f)) 
        = ev_{\phi(a_i)} (\bar\phi(\bar\iota_L (f)))
        = ev_{\phi(a_i)}((\overline{\phi \circ \iota_L}) (f))
    \]
    i.e. $\phi(a_i)$ is a root of $f$. 
    But of course, since $\phi$ and $\psi$ are $K$-extension morphisms,
    $\phi \circ \iota_L = \iota_M = \psi \circ \iota_L$.
    And hence, \[
        0 = ev_{\phi(a_i)}( (\overline{\psi \circ \iota_L})(f) )
        = ev_{\phi(a_i)} (\psi(\iota_L \la) \prod_{i\in\deg f} (X-\psi(a_i)))
        = \psi(\iota_L \la) \prod_{j\in\deg f} (\phi(a_i) -\psi(a_j))
    \]
    which by $K$ being an integral domain gives $\phi(a_i) = \psi(a_j)$
    for some $j$, finishing the proof. 
\end{proof}
\end{document}