\documentclass[../book.tex]{subfiles}

\begin{document}
%    -  Def - Galois Ext, Galois Group
\begin{dfn} Galois Extension and Galois Group.
    Let $\iota_L : K \to L$ be a finite field extension. 
    $(L,\iota_L)$ is called a \textbf{Galois extension} when 
    it is any of the following equivalent things: 
    \begin{enumerate}
        \item a normal and separable extension.
        \item the splitting field of some separable polynomial over $K$. 
    \end{enumerate} 
    These are easily verified to be equivalent. 
    Then the \textbf{Galois group of $(L,\iota_L)$} is defined as 
    the group of $K$-extension automorphisms of $(L,\iota_L)$, 
    denoted $\aut{K}{(L,\iota_L)}$.
    If $\iota_L$ is clear, we simply write $\aut{K}{L}$. 
\end{dfn}

\begin{ex} Galois Closure of a Separable Extension.
    
    Let $\iota_L : K \to L$ be an extension.
    Then an $L$-extension $(N,\iota_N)$ is called 
    the \textbf{Galois closure of $(L,\iota_L)$} 
    when it is the smallest $L$-extension that is Galois as a $K$-extension,
    i.e. for all $L$-extensions $(M,\iota_M)$,
    $(M,\iota_M\circ\iota_L)$ Galois implies
    there exists an $L$-extension morphism $\bar{\iota_M} : N \to M$. 
    
    Suppose $\iota_L : K \to L$ be a finite, separable extension. 
    Show that there exists a $(N,\iota_N)$ Galois closure of $(L,\iota_L)$
    and it is unique up to $L$-extension isomorphisms i.e.
    any other Galois closure of $(L,\iota_L)$ is isomorphic to $(N,\iota_N)$
    as an $L$-extension. 
    
\end{ex}

%    -  Thm - Galois Correspondance Part -1
\begin{thm} Galois Extension Degree equals Order of Galois Group.
    
    Let $\iota_L : K \to L$ be a Galois extension. 
    Then $[L : K] = |\aut{K}{L}|$. In particular the Galois group is finite. 
\end{thm}
\begin{proof}
    Since $L$ itself is an extension extending $\iota_L : K \to L$ that is normal,
    we have \[
        [L : K] = [L : K]_S = |\mor{K}{L}{L}| = |\aut{K}{L}|
    \]
\end{proof}
%    -  Lem - Important Lemma / Reverse Galois / Fixed Subfields
\begin{lem} The All Important Lemma.
    
    Let $L$ be a field, and $G$ be a finite subgroup of $\aut{\Ring}{L}$. 
    Since $G \leq_\Grp \aut{\Ring}{L} \subseteq \aut{\Set}{L}$, 
    we have a natural $G$ action on $L$ via $l \mapsto g(l)$ for $l \in L$. 
    Define \textbf{the subfield of $L$ fixed by $G$} as
    \[ L^G := \{l \in L \mid \forall g \in G, g(l)=l\} \]
    Then $\subseteq : L^G \to L$ is normal and separable with $[L:L^G] = |G|$,
    i.e. it is a Galois extension. 
    Furthermore, $G = \aut{L^G}{L}$.
\end{lem}
\begin{proof}
    We leave the reader to check that $L^G$ indeed forms a subfield. 

    Let $a \in L$. We claim that \[
        \min(a,L^G) = \prod_{b \in Orb(a)} (X - b)
    \]
    Hence $\min(a,L^G)$ splits and has no repeated roots. 
    
    We first show that the product is actually in $L^G[X]$. 
    Let $g \in G$. Then \[
        \bar{g} \prod_{b \in Orb(a)} (X - b) 
        = \prod_{b \in Orb(a)} (X - g(b)) = \prod_{b' \in Orb(a)} (X - b')
    \]
    Hence the product is fixed by all $g \in G$. 
    Writing the product as $\sum_{k\in n} \la_k X^k$ where $\la_k \in L$,
    we have $\la_k \in L^G$, i.e. the product is indeed in $L^G[X]$. 
    
    We now prove the product is irreducible in $L^G[X]$.
    The product is clearly non-zero and non-unit. 
    Let $f, g$ be polynomials over $L^G$ such that \[
        fg = \prod_{b \in Orb(a)} (X - b)
    \]
    Evaluating both sides at $a$, we have $0 = ev_a f$ or $0 = ev_a g$. 
    WLOG $ev_a f = 0$. 
    Then since $f \in L^G[X]$ implies for all $g \in G$, $\bar{g}f = f$, 
    we have \[
        0 = g(0) = g(ev_a f) = ev_{g(a)}(\bar{g} f) = ev_{g(a)} f
    \]
    Hence for all $b \in Orb(a)$, $b$ is a root of $f$, i.e.
    there exists a polynomial $h$ over $L$ such that \[
        h \prod_{b \in Orb(a)} (X - b) = f
    \]
    The by $L[X]$ is an integral domain, $1 = gh$,
    which implies $\deg g = 0$, giving $g$ is a unit in $L^G[X]$. 
    This proves irreducibility of the product in $L^G[X]$. 
    Hence it is equal to $\min(a,L^G)$. 
    
    Note that since we defined normal and separable extensions as finite extensions,
    we must prove the extensions degree is finite to conclude the extension is Galois.
    
    Suppose $|G| < [L : L^G]$. 
    Then there exists finite linearly independent set $S$ such that $|G| < |S|$. 
    Then since the $L^G$-subspace generated by $S$ is contained in $L^G(S)$, we have 
    \[ |G| < |S| = \dim_{L^G} \<S\>_{L^G} \leq [L^G(S) : L^G] \]
    Since all of $S$ is algebraic over $L^G$, 
    $L^G(S)$ is obtained by adjoining finitely many algebraic elements.
    So $L^G(S)$ is a finite extension, and thus a separable extension as well. 
    Hence there exists a primitive element $a$ where $L^G(S) = L^G(a)$. 
    But then \[
        |G| \leq |Orb(a)| = |\min(a,L^G)| = [L^G(a) : L^G] = [L^G(S) : L^G]
    \]
    which is a contradiction. 
    
    Thus, $[L : L^G] \leq |G|$. In particular, $L$ is a Galois $L^G$-extension.
    Furthermore, since $G \subseteq \gal{L^G}{L}$, \[
        |G| \leq |\gal{L^G}{L}| = [L : L^G] 
    \]
    So $[L : L^G] = |G|$. 
    We also have $|\gal{L^G}{L}| = |G|$. 
    Since both are finite sets and $G \subseteq \gal{L^G}{L}$,
    we have $G = \gal{L^G}{L}$. 
\end{proof}

% % Alternative formulation
% \begin{dfn} 4 Equivalent Definitions of Galois Extension. 
%     % Adapted from Milne's Galois Theory notes
    
%     Let $\iota_L : K \to L$ be an extension. 
%     Let $\aut{K}{L}$ denote the $K$-extension automorphisms of $(L,\iota_L)$. 
%     $\aut{K}{L}$ is a subgroup of 
%     the group of ring automorphisms of $L$, $\aut{\Ring}{L}$. 
%     For a subgroup $G$ of $\aut{\Ring}{L}$, 
%     since $G \leq_\Grp \aut{\Set}{L}$, 
%     we have a natural $G$ action on $L$ via $l \mapsto g(l)$ for $l \in L$. 
%     Define \textbf{the subfield of $L$ fixed by $G$} as
%     \[ L^G := \{l \in L \mid \forall g \in G, g(l)=l\} \]
%     Then the following are equivalent: 
%     \begin{enumerate}
%         \item $(L,\iota_L)$ is a finite, normal and separable extension.
% %        \item $(L,\iota_L)$ is the splitting field 
% %            of some separable polynomial $f$ over $K$.
%         \item $[L : K] = |\aut{K}{L}|$ is finite and 
%         $\bar{\iota_L}K = L^{\aut{K}{L}}$.
%         \item There exists a finite subgroup $G$ of $\aut{\Ring}{L}$ such that $\bar{\iota_L} K = L^G$.
%     \end{enumerate}
%     If any of the above is true,
%     we call $(L,\iota_L)$ a \textbf{Galois $K$-extension}
%     and $\aut{K}{L}$ is called the \textbf{Galois group of $(L,\iota_L)$},
%     which is sometimes denoted with $\gal{K}{L}$.
% \end{dfn}
% \begin{proof}
%     ($1\imp 2$)

% %        There exists generators $a_0,\dots,a_{n-1}$ of $L$ as a $K$-extension.
% %        Let $f = \prod_{i\in n} \min(a_i,K)$. 
% %        Then $(L,\iota_L)$ algebraic, normal and separable implies 
% %        $L$ splits $f$ and $f$ is separable.
% %        Since the roots of $f$ contain the generators $a_0,\dots,a_{n-1}$,
% %        the roots generate $L$. Hence $L$ is the splitting field of $f$.
% %        
% %    ($2\imp 3$)
% %        
% %        $L$ generated by roots of $f$ implies 
% %        $L$ is generated by finitely many algebraic elements.
% %        So $L$ is a finite extension. 
% %        
% %        We leave the reader to check that $L^{\aut{K}{L}}$ indeed forms a subfield. %
% %        We wish to show $K = L^{\aut{K}{L}}$. 
% %        Clearly, $K \subseteq L^{\aut{K}{L}}$.
% %        We will show $[L : K] \leq [L : L^{\aut{K}{L}}]$ and hence
% %        by the tower law, $[L^{\aut{K}{L}} : K] = 1$, giving the desired result.
% %        
% %        We begin by noting that $\aut{K}{L} \subseteq \aut{L^{\aut{K}{L}}}{L}$.
% %        For all generators $a$ of $L$, 
% %        $\min(a,K)$ divides $f$ which is split and separable, 
% %        so $\min(a,K)$ is split and separable. 
% %        So by embedding via conjugates and all $\min(a,K)$ having distinct roots,
% %        we have $[L : K] = |\aut{K}{L}|$.
% %        Note that $L$ is also the splitting field of 
% %        $\bar{\iota_L} f$ considered as a polynomial over $L^{\aut{K}{L}}$
% %        and $\bar{\iota_L} f$ is still separable,
% %        so by the same argument we have 
% %        $[L : L^{\aut{K}{L}}] = |\aut{L^{\aut{K}{L}}}{L}|$. 
% %        Combining everything, we have \[
% %            [L : K] = |\aut{K}{L}| 
% %            \leq |\aut{L^{\aut{K}{L}}}{L}| = [L : L^{\aut{K}{L}}]
% %        \]
% %        as desired. 
%         Since $(L,id)$ itself is an $L$-extension extending $\iota_L : K \to L$
%         that is normal as a $K$-extension, by definition of separable degree,
%         we have \[
%             |\aut{K}{L}| = [L : K]_S = [L : K]
%         \]
%         which is finite. 
        
%         To see that $K = L^{\aut{K}{L}}$, note that we already have \[
%             K \overset{\iota_L}{\to} L^{\aut{K}{L}} \overset{\subseteq}{\to} L
%         \]
%         Then $K \to L$ finite, normal and separable implies
%         $L^{\aut{K}{L}}$ finite normal and separable, and hence 
%         $|\aut{L^{\aut{K}{L}}}{L}| = [L : L^{\aut{K}{L}}]$ as well. 
%         Combining this with $\aut{K}{L} \subseteq \aut{L^{\aut{K}{L}}}{L}$,
%         we have \[
%             [L : K] = |\aut{K}{L}| 
%             \leq |\aut{L^{\aut{K}{L}}}{L}| = [L : L^{\aut{K}{L}}]
%         \]
%         implying $[L^{\aut{K}{L}} : K] = 1$ by the tower law, 
%         i.e. $\bar{\iota_L}K = L^{\aut{K}{L}}$. 
        
%     ($2\imp 3$) Trivial. 
% %        We claim that $\aut{K}{L}$ is finite and thus our desired subgroup $G$. 
% %        The key is to use embedding via conjugates to bound 
% %        $|\aut{K}{L}|$ by $[L : K]$. 
% %        Since $[L : K]$ is finite, we have a finite set of generators of $L$.
% %        But we do not know whether $L$ splits 
% %        the minimal polynomials of these generators as required in the condition
% %        of embedding via conjugates. 
% %        So we take $(N,\iota_N)$, the normal closure of $(L,\iota_L)$ instead
% %        and apply embedding via conjugates to $(N,\iota_N\circ\iota_L)$,
% %        giving $|\mor{K}{L}{N}| \leq [L : K]$.
% %        But of course, $K$-extension automorphisms of $L$ give rise to 
% %        $K$-extension morphisms from $L$ to $N$ by composition with $\iota_N$.
% %        So \[
% %            |\aut{K}{L}| \leq |\mor{K}{L}{N}| \leq [L : K]
% %        \] which is finite. 
        
%     ($3\imp 1$)        
%         This is the more difficult implication.
%         Let $a \in L$. We claim that \[
%         \bar\iota_L \min(a,K) = \prod_{b \in Orb(a)} (X - b)
%         \]
%         This shows all elements of $L$ have minimal polynomials 
%         that $L$ splits and are separable. 
    
%         We first show that the product is actually in $\bar{\iota_L}K[X] = L^G[X]$. 
%         Let $g \in G$. Then \[
%             \bar{g} \prod_{b \in Orb(a)} (X - b) 
%             = \prod_{b \in Orb(a)} (X - g(b)) = \prod_{b' \in Orb(a)} (X - b')
%         \]
%         Hence the product is fixed by all $g \in G$. 
%         Writing the product as $\sum_{k\in n} \la_k X^k$ where $\la_k \in L$,
%         we have $\la_k \in L^G$, i.e. the product is indeed in $\bar{\iota_L}K[X]$. 
%         This shows $a$ is algebraic over $K$, and hence $\min(a,K)$ exists.
        
%         Since $\bar{\iota_L} K = L^G$, $\bar\iota_L \min(a,K) = \min(a,L^G)$.
%         So we only need to prove the product is irreducible in $L^G[X]$.
%         The product is clearly non-zero and non-unit. 
%         Let $f, f_0$ be polynomials over $L^G$ such that \[
%             f f_0 = \prod_{b \in Orb(a)} (X - b)
%         \]
%         Evaluating both sides at $a$, 
%         we have $0 = ev_a (f)$ or $0 = ev_a(f_0)$. 
%         WLOG $0 = ev_a (f)$. 
%         Then since $f \in L^G[X]$ implies 
%         for all $g \in G$, $\bar{g}(f) = f$, 
%         we have \[
%             0 = g(0) = g(ev_a f) = ev_{g(a)}(\bar{g} f) = ev_{g(a)} f
%         \]
%         Hence for all $b \in Orb(a)$, $b$ is a root of $f$, i.e.
%         there exists a polynomial $h$ over $L$ such that \[
%             h \prod_{b \in Orb(a)} (X - b) = f
%         \]
%         Then by $L[X]$ is an integral domain, $1 = f_0 h$,
%         which implies $\deg f_0 = 0$, giving $f_0$ is a unit in $L^G[X]$. 
%         This proves irreducibility of the product in $L^G[X]$. 
%         Hence it is equal to $\min(a,L^G) = \bar{\iota_L}\min(a,K)$. 
    
%         Thus, once we prove $[L : K]$ is finite, it is both normal and separable. 
%         We claim that $[L : K]$ is bounded by $|G|$.
    
%         Suppose $|G| < [L : K]$. 
%         Then there exists finite linearly independent set $S$ such that $|G| < |S|$. 
%         Then since the $K$-subspace generated by $S$ is contained in $K(S)$, 
%         we have \[ |G| < |S| = \dim_{K} \<S\>_{K} \leq [K(S) : K] \]
%         Since all of $S$ is algebraic over $K$, 
%         $K(S)$ is obtained by adjoining finitely many algebraic elements.
%         So $K(S)$ is a finite extension, and thus a separable extension as well. 
%         Hence there exists a primitive element $a$ where $K(S) = K(a)$. 
%         But then \[
%             |G| \leq |Orb(a)| = |\min(a,K)| = [K(a) : K] = [K(S) : K]
%         \]
%         which is a contradiction. 
%         Thus, $[L : K] \leq |G|$ which is finite. 
% \end{proof}
% Fuck me, this turns out not to be an alternative formulation, but
% still uses the original All-Important-Lemma. 
% I wonder whether Kenny thought about all of this. 

%    - Thm - Galois Correspondance / Fund Thm Gal
%        -  Well Def
%        -  Inv.1 
%        -  Inv.2 
%        -  Order.1
%        -  Order.2
%        -  Degree
%        -  Equivariance
%        -  Normality
%        -  Quotient

\begin{thm} Galois Correspondance. 
    
    Let $\iota_L : K \to L$ be a Galois extension.
    Let $E \subseteq L$ be a $K$-subextension of $L$. 
    Then the group of $E$-extension automorphisms of $L$ is
    a subgroup of the Galois group of $(L,\iota_L)$. \[
        \aut{E}{L} \leq_\Grp \aut{K}{L}
    \]
    Let $G \leq \aut{K}{L}$ be a subgroup of the Galois group of $(L,\iota_L)$. 
    Then the subfield fixed by $G$, $L^G$, is a $K$-subextension of $(L,\iota_L)$.
    \[ K \overset{\iota_L}{\to} L^G \]
    
    The \emph{Galois Correspondance} states the following: 
    \begin{enumerate}
        \item (Inverses)
            For a $K$-subextension $E$, \[
                E = L^{\aut{E}{L}}
            \]
            For a subgroup $G \leq \aut{K}{L}$ \[
                G = \aut{L^G}{L}
            \]
            Hence, we have a bijection between $K$-subextensions of $(L,\iota_L)$
            and the subgroups of the Galois group of $(L,\iota_L)$.
        \item (Order Reversing) 
            Let $E, F \subseteq L$ be $K$-subextensions such that $E \subseteq F$.
            Then \[ \aut{F}{L} \subseteq \aut{E}{L} \]
            Similarly, 
            let $G, H \leq \aut{K}{L}$ be subgroups such that $G \subseteq H$.
            Then \[ L^H \subseteq L^G \]
        \item (Degree) 
            Let $E$ be a $K$-subextension of $L$. 
            Then $[E : K] = [\aut{K}{L} : \aut{E}{L}]$
            where the latter is index of subgroups. 
        \item ($\aut{K}{L}$-Sets)
            Let $\sigma$ be an element of the Galois group $\aut{K}{L}$.
            Then $E \mapsto \sigma E$ defines a $\aut{K}{L}$-action
            on the $K$-subextensions of $L$. 
            We also have a $\aut{K}{L}$-action 
            on the set of subgroups of $\aut{K}{L}$ via 
            $H \mapsto \sigma H \sigma^{-1}$.
            
            With these two actions, $\aut{-}{L} : E \mapsto \aut{E}{L}$
            becomes a $\aut{K}{L}$-set isomorphism from the $K$-subextensions of $L$
            to the subgroups of the Galois group of $L$. 
        \item (Normality)
            Let $E$ be a $K$-subextension of $L$. \[
                K \overset{\iota_L}{\to} E \overset{\subseteq}{\to} L
            \]
            Then $\iota_L : K \to E$ is a normal extension if and only if
            $\aut{E}{L}$ is a normal subgroup of $\aut{K}{L}$. 
            When this is the case, $\iota_L : K \to E$ is a Galois extension
            and we have the isomorphism of Galois groups \[
                \aut{K}{E} \iso_\Grp \aut{K}{L} / \aut{E}{L}
            \]
    \end{enumerate}
\end{thm}
\begin{proof}
    (Inverses)
        
        Let $E$ be a $K$-subextension of $L$. 
        We already have \[
            E \overset{\subseteq}{\to} L^{\aut{E}{L}} \overset{\subseteq}{\to} L
        \]
        with $E \to L$ Galois and $[L : E] = |\aut{E}{L}|$.
        Note that $[L : K]$ finite implies $\aut{E}{L}$ is finite 
        and hence $L^{\aut{E}{L}} \to L$ Galois
        with $[L : L^{\aut{E}{L}}] = |\aut{E}{L}|$ by the previous lemma. 
        Then by the tower law, $[L^{\aut{E}{L}} : E] = 1$, i.e. $E = L^{\aut{E}{L}}$.
        
        Let $G$ be a subgroup of $\aut{K}{L}$. 
        Then by the previous lemma $G = \aut{L^G}{L}$. 
        
    (Order Reversing) Left as an easy exercise. 
        
    (Degree) Left as an easy exercise.
    
    ($\aut{K}{L}$-Sets) 
        
        We already know $\aut{-}{L}$ is a bijection,
        so we only need to show it respects the $\aut{K}{L}$-actions.
        
        Let $\sigma \in \aut{K}{L}$. Let $E$ be a $K$-subextension of $L$. 
        We need to show \[
            \aut{\sigma E}{L} = \sigma \aut{E}{L} \sigma^{-1}
        \]
        We invite the reader to stare at this
        until this becomes trivial. 
    
    (Normality)
        
        Let $E$ be a $K$-subextension of $L$ that is normal. 
        Then for all $K$-extension automorphisms $\sigma$ of $L$, $\sigma E = E$. 
        Hence by $\aut{-}{L}$ being a morphism of $\aut{-}{L}$-sets,
        for all $\sigma \in \aut{K}{L}$, \[
            \sigma \aut{E}{L} \sigma^{-1} = \aut{\sigma E}{L} = \aut{E}{L}
        \]
        i.e. $\aut{E}{L}$ is a normal subgroup of $\aut{K}{L}$.
        
        Now the more difficult implication. 
        Suppose $E$ is a $K$-subextension of $L$ such that
        $\aut{E}{L}$ is normal. 
        Then for all $K$-extension automorphisms $\sigma$ of $L$, 
        $\sigma \aut{E}{L} \sigma^{-1} = \aut{E}{L}$.
        Hence for all $\sigma \in \aut{K}{L}$, \[
            \sigma E = L^{\aut{\sigma E}{L}} = L^{\aut{E}{L}} = E
        \]
        This looks almost like the image invariance definition of normality,
        but only for $K$-extension automorphisms of $L$. 
        We will prove $E$ has the image invariance property as a $K$-extension. 
        
        Let $\iota_M : K \to M$ be a $K$-extension and 
        $f, g$ be $K$-extension morphisms from $(E,\iota_L)$ to $(M,\iota_M)$.
        The following diagram presents the proof. 
        \begin{figure} [H]
            \centering
            \begin{tikzcd} [sep = huge]
            K \arrow[r,"\iota_L"] \arrow[rd,"\iota_M"{swap}] &
            E \arrow[r,"\iota_{N(E)}"] \arrow[d,"f"{swap},xshift=-0.5ex] 
            \arrow[d,"g",xshift=0.5ex] &
            N(E) \arrow[r,"\iota"] \arrow[d,"\bar{f}"{swap},xshift=-0.5ex] 
            \arrow[d,"\bar{g}",xshift=0.5ex] 
            \arrow[loop above,"\sigma := \bar{g}^{-1}\circ\bar{f}"] &
            L \arrow[loop above,"\bar{\sigma}"] \\
            & 
            M \arrow[r,"\iota_{N(M)}"] & 
            N(M) & 
            \end{tikzcd}
        \end{figure}
        We do not know whether $E$ has the image invariance property,
        but we do know its normal closure does. 
        Let $(N(E),\iota_{N(E)})$ be the normal closure of $(E,\iota_L)$. 
        Since $L$ is an $E$-extension that is normal as a $K$-extension,
        we have an $E$-extension morphism 
        $\iota : (N(E),\iota_{N(E)}) \to (L,id_E)$,
        i.e. $\iota\circ\iota_{N(E)} = id_E$ 
        where $id_E$ is the inclusion of $E$ into $L$.  
        
        We want to use image invariance on $N(E)$ as a $K$-extension,
        but the maps $f, g$ are from $E$. 
        To "lift" these maps up to $N(E)$, 
        let $(N(M),\iota_{N(M)}$ be the normal closure of $(M,\iota_M)$. 
        Then $(N(M),\iota_{N(M)}\circ f)$ and $(N(M),\iota_{N(M)}\circ g)$
        are both $E$-extensions that are normal as $K$-extensions.
        Hence by minimality of the normal closure of $E$, 
        we have $E$-extension morphisms 
        $\bar{f} : (N(E),\iota_{N(E)}) \to (N(M),\iota_{N(M)}\circ f)$
        and $\bar{g} : (N(E),\iota_{N(E)}) \to (N(M),\iota_{N(M)}\circ g)$.
        Then $\bar{f}$ and $\bar{g}$ are clearly $K$-extension morphisms
        from $N(E)$ to $N(M)$,
        so by image invariance, \[ \bar{f} N(E) = \bar{g} N(E) \]
        Then $\sigma := \bar{g}^{-1} \circ \bar{f}$ is well-defined as a function.
        This gives $\sigma$ as a $K$-extension automorphism of $N(E)$. 
        In particular, $id_E$ and $\iota\circ\sigma\circ\iota_{N(E)}$ 
        are two $K$-extension morphisms from $(E,\iota_L)$ to $(L,\iota_L)$.
        So by embeddings into normal extensions differing by an automorphism, 
        we have an $E$-extension morphism $\bar{\sigma}$
        from $(L,id_E)$ to $(L,\iota\circ\sigma\circ\iota_{N(E)})$.
        We have turned the two $K$-extension morphisms $f, g$ into 
        a $K$-extension automorphism of $L$!
        Of course, $\bar{\sigma} E = E$, which implies \begin{align*}
            (\iota\circ\sigma\iota_{N(E)}) E = (\iota\circ\iota_{N(E)}) E &\imp
            (\bar{g}^{-1}\circ\bar{f}\circ\iota_{N(E)}) E = (\sigma\circ\iota_{N^(E)})E
                = \iota_{N(E)} E \\
            &\imp (\bar{f}\circ\iota_{N(E)}) E = (\bar{g}\circ\iota_{N(E)}) E \\
            &\imp (\iota_{N(M)}\circ f) E = (\iota_{N(M)}\circ g) E \\
            &\imp f E = g E
        \end{align*}
        Thus, $(E,\iota_L)$ has the image invariance property as a $K$-extension. 
        
        Furthermore, $\iota_L : K \to L$ finite and separable implies 
        $\iota_L : K \to E$ also finite and separable,
        so $(E,\iota_L)$ is a Galois extension. 
        
        We leave the isomorphism of Galois groups as an exercise to the reader. 
        [Hint : Use the first isomorphism theorem, remembering to prove surjectivity.]
\end{proof}
%    -  Cor - Fund Thm Alg
\begin{cor}
For all polynomial $f \in \mathbb{C}[x]$, $\mathbb{C}$ splits $f$.
\end{cor}

To prove this corollary, we need to notice two lemmas first:

%        -  No non-trivial odd deg ext of R
\begin{lem}
There does not exists non-trivial field $F$ such that $[F: \mathbb{R}]$ is odd.
\end{lem}
\begin{proof}
Assume such $F$ exist, pick $ \alpha \in F\setminus \mathbb{R}$, it will have a minimal polynomial with odd degree. By immediate value theorem, then there exist real root in $\mathbb{R}$. Hence $[F: \mathbb{R}]=1$, i.e. this is a trivial extension.
\end{proof}
%        -  No non-trivial even deg ext of C 
\begin{lem}
There does not exists field $F$ such that $[F: \mathbb{C}]=2$
\end{lem}
\begin{proof}
Assume such $F$ exist, pick $ \alpha \in F\setminus \mathbb{C}$, it will have a minimal polynomial with degree 2. However $\mathbb{C}$ must split any quadratic polynomial. Contradiction.
\end{proof}

\textit{Proof of Corollary.}
\end{document}