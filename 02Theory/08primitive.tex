\documentclass[../book.tex]{subfiles}
\begin{document}

We now present a short chapter on the Primitive Element theorem,
which states that finite separable extensions are in fact simple extensions.
We split into cases of the base field $K$ being a finite set and an infinite set. 
We deal with the infinite case first. 

\begin{thm} Primitive Element for Infinite Base Field.
    %From Milne's Galois theory notes
    
    Let $\iota_L : K \to L$ be a finite separable extension where $K$ is infinite. 
    Then there exists an element $a \in L$ such that $L = K(a)$. 
\end{thm}
\begin{proof}
    
    There exists a finite set of generators $S$ of $L$ as a $K$-extension.
    We proceed by induction on the number of generators. 
    Assume finite separable $K$-extensions with $|S|-1$ many generators 
    can all be written as simple extensions. 
    Then WLOG $L = K(a, b)$. 
    Let $c = a + \la b$ for some $\la \in K$. 
    We have the following situation. 
    \begin{figure}[H]
        \centering
        \begin{tikzcd}
        K \arrow[r,"\iota_L"] &
        K(c) \arrow[r,"\subseteq"] &
        K(a,b) = L
        \end{tikzcd}
    \end{figure}
    We claim there exists a $\la$ such that $K(c) = L$. 
    Let $(N,\iota_N)$ be the normal closure of $(L,\iota_L)$
    with $a_0,\dots,a_n$ and $b_0,\dots,b_m$ respectively Galois conjugates of $a, b$ in $N$.
    WLOG $a_0 = \iota_N(a)$ and $b_0 = \iota_N(b)$.
    For $i \in n, 0 \neq j \in m$, the equation \[
        a_i + \la b_j = a_0 + \la b_0
    \]
    has the unique solution \[
        \la = \la_{i,j} := \frac{a_0 - a_i}{b_j - b_0}
    \]
    $\la_{i,j}$ is well-defined since $L$ separable implies $b_j \neq b_0$ for $j\neq0$.
    There are only finitely many $\la_{i,j} \in K$, so since $K$ is infinite, 
    we can pick $\la$ that is not equal to any $\la_{i,j}$. 
    We show that such a $\la$ is sufficient.
    
    Suppose $\deg\min(b,K(c)) = 1$. 
    Then $b \in K(c)$, which implies $a \in K(c)$, i.e. $L = K(c)$ and we are done. 
    On the other hand, assume $\deg\min(b,K(c)) > 1$. 
    Then $\min(b,K(c))$ divides $\min(b,K) = \min(b_0,K)$ implies
    there exists a $b_j$ Galois conjugate of $b_0$ such that 
    $b_j \neq b_0$ and $b_j$ is a root of $\min(b,K(c))$. 
    Note that $\min(a,K(c)) (c - \la b) = 0$,
    so $\min(b,K(c))$ divides $\min(a,K(c)) \circ (c - \la X)$.
    Hence $b_j$ being a root of $\min(b,K(c))$ implies
    \[ 0 = \bar{\iota_N} \min(a,K(c)) (\iota_N(c) - \la b_j) \]
    i.e. $\iota_N(c) - \la b_j = a_i$ a Galois conjugate of $a_0$.
    Thus, $a_i + \la b_j = \iota_N(c) = a_0 + \la b_0$ for $j \neq 0$, a contradiction. 
    This completes the proof. 
    
\end{proof}

\begin{rmk}
    The above proof shows that primitive elements of a finite separable extension
    $K(a_0,\dots,a_{n})$ are of the form \[
        c = a_0 + \la_1 a_1 + \cdots + \la_n a_n
    \]
    where $\la_i$ are appropriate elements in $K$. 
    We will come back to computations of primitive elements 
    after developing the Galois correspondance. 
\end{rmk}

Now the machinery for the finite case. 
We shall argue that for a finite field $L$, 
the group of its units $L^\times$ forms a \emph{cyclic} group. 

\begin{dfn} Order of an Element, Cyclic Group.
    
    Let $G$ be a group. 
    For an element $g \in G$, the \textbf{order of $g$} is defined as 
    the smallest positive integer $o(g)$ such that $g^{o(g)} = e$. 
    If no such $o(g)$ exists, we say $g$ has \emph{infinite order}. 
    Note that in the case of $G$ finite, 
    $o(g)$ exists for all $g \in G$ and $o(g)$ is the order of $\<g\>$,
    hence divides the cardinality of $G$ by Lagrange's theorem. 
    Using the division algorithm, one can also show that $o(g)$
    divides any other positive natural $n$ where $g^n = e$. 
    
    $G$ is called a \textbf{cyclic group} when 
    there exists an element $g \in G$ such that $\<g\> = G$. 
    We say \textbf{$g$ generates $G$} or \textbf{$g$ is a generator of $G$}.
\end{dfn}

\begin{lem} Basic Results about Cyclic Groups and a Number-Theoretic Result.
    
    Let $G$ be a finite cyclic group with a generator $g$. 
    Then \begin{enumerate}
        \item For any positive integer $n$, the \textbf{totient of $n$}
        is defined as the number of positive integers coprime to $n$, 
        denoted $\phi(n)$. 
        
        Let $g^k$ be an element. Then $o(g) = |G| / (|G|,k)$.
        Hence, the number of generators of $|G|$ is $\phi(|G|)$
        
        \item Let $H$ be a subgroup of $G$. Then $H$ is cyclic, too. 
        \item Let $n = |G|$. Then \[
            n = \sum_{d \mid n} \phi(d)
        \]
    \end{enumerate}
\end{lem}
\begin{proof}
    (1)
    
        Clearly, $(g^k)^{|G|/(|G|,k)} = e$. 
        We now show it is minimal. 
        Let $(g^k)^n = e$. 
        Then $g^{kn} = e$ implies $|G|$ divides $kn$ since $g$'s order is $|G|$. 
        That is to say, there exists an integer $m$ such that $|G| m = k n$. 
        Dividing by $(|G|,k)$ on both sides gives \[
            \frac{|G|}{(|G|,k)} m = \frac{k}{(|G|,k)} n
        \]
        We leave it to the reader to show that 
        $|G| / (|G|,k)$ and $k / (|G|, k)$ are coprime, 
        and hence $|G| / (|G|,k)$ divides $n$. 
        In particular, it is smaller than or equal to $n$.
        
        We say that $g^k$ is a generators if and only if $(|G|,k) = 1$.
        Hence number of generators is the totient of $|G|$. 
    (2)
        
        If $H = \<e\>$, then $H$ is trivially cyclic.
        So suppose there exists a non-identity element $g^k$ in $H$.
        Since non-empty subsets of $\N$ have a minimal element, 
        WLOG $k$ is the minimal positive integer such that $g^k$ is in $H$.
        We leave it to the reader to check that for any other element $g^m$ in $H$,
        $k$ divides $m$, and hence $g^k$ generates $H$. 
    
    (3)
        
        By assumption, there exists a positive integer $n$ such that $d n_d = |G|$.
        We leave it to the reader to verify that $g^{n_d}$ has order $d$,
        hence the subgroup generated by it has cardinality $d$. 
        This shows existence. 
        
        Let $H$ be another subgroup of order $d$. 
        Then there exists a $g^{m_d}$ that generates $H$. 
        Then $g^{dm_d} = e$ implies $|G|$ divides $dm_d$. 
        It is easy to check that $n_d$ divides $m_d$.
        Hence $g^{m_d}$ is in $H_d$, i.e. all of $H$ is in $H_d$. 
        Both are sets of order $d$, so $H = H_d$.
        
    (4)
        
        The orders of elements in $G$ give a partition of $G$,
        and since orders of elements must divide $|G| = n$, we have \[
            n = \sum_{d \mid n} \{x \in G \mid o(x) = d\}
        \]
        Let $d$ be one of such orders. 
        Then all elements of order $d$ generate subgroups of order $d$,
        so by (3), they are generate the same cyclic subgroup $H_d$. 
        Generators of $H_d$ are clearly order $d$, 
        so elements are order $d$ if and only if they generate $H_d$.
        Let $h$ be a generator of $H_d$ so that 
        all elements of $H_d$ are $h^k$ for some natural number $k$. 
        Since $H_d$ is finite cyclic, 
        the number of generators of $H_d$ is equal to $\phi(d)$. 
        This gives \[
            n = \sum_{d \mid n} \{x \in G \mid o(x) = d\} = \sum_{d \mid n} \phi(d)
        \]
    
\end{proof}

\begin{thm} A Condition for Cyclic, Primitive Element for Finite Fields.
    
    Let $G$ be a finite group such that for all $d$ dividing $|G|$, 
    the number of $x \in G$ satisfying $x^d = e$ is less than or equal to $d$. 
    Then $G$ is cyclic.
    
    In particular, take $G = L^\times$ where $L$ is a finite field.
    Then $L^\times$ is cyclic. 
    Hence if $\iota_L : K \to L$ is a finite extension where $K$ is finite,
    then there exists an element $c \in L$ such that $L = K(c)$.
\end{thm}
\begin{proof}
    
    We again partition $G$ by the orders of elements. 
    Let $G_d$ be the set of elements in $G$ with order $d$. 
    Then we have \[
        |G| = \sum_{d \mid |G|} |G_d|
    \]
    We show that $|G_d| \leq \phi(d)$. 
    Note that $G_d$ is a subset of the elements that satisfy $x^d = e$. 
    Suppose $G_d$ is empty. Then we are done.
    Suppose $G_d$ is non-empty. 
    Then there exists a non-identity element $g$ with order $d$.
    Consider the subgroup generated by $g$, $\<g\>$. 
    All elements in $\<g\>$ satisfy $x^d = e$.
    So $\<g\>$ is a $d$-element subset of the solutions to $x^d = e$,
    which has at most $d$-elements.
    Hence $\<g\>$ \emph{is} the set of solutions to $x^d = e$.
    In particular, all other elements of order $d$ must be in $\<g\>$.
    Generators of $\<g\>$ are also order $d$, so we have $|G_d| = \phi(d)$. 
    
    We then have \[
        |G| = \sum_{d \mid |G|} |G_d| \leq \sum_{d \mid |G|} \phi(d) = |G|
    \]
    Thus $|G_d| = \phi(d)$ for all $d$ dividing $|G|$. 
    In particular, $|G_{|G|}| = \phi(|G|) \neq 0$, 
    so we have an element of order the cardinality of $G$,
    i.e. $G$ is cyclic.
    
    Let $L$ be a finite field. 
    Then elements in $(L^\times,\cdot)$ such that $x^d = e$ are precisely 
    the roots to the polynomial $X^d - 1$.
    The number of such roots is bounded by $\deg (X^d - 1) = d$. 
    $L^\times$ satisfies the condition of the above result, hence must be cyclic.
    
    If $\iota_L : K \to L$ is a finite extension where $K$ is finite,
    then $L$ is also finite, so $L^\times$ is cyclic.
    Let $c$ be a generator of $L^\times$ as a group. 
    Then $K(c)$ clearly contains $L^\times$, and hence is equal to $L$. 
\end{proof}

\end{document}