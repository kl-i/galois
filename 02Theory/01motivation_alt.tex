\documentclass[../book.tex]{subfiles}

\begin{document}
% This example is the original one Kenny showed us.

What follows is a simple example of the \emph{Galois correspondence},
the central result upon which Galois theory is built. 
We assume the reader is familiar with the notion of groups, subgroups,
dimension of vector spaces, linear maps and subspaces.
If not, feel free to study the preliminaries in the next section first.

Consider $\R$, the set of real numbers.
We can do basic arithmetic 
with addition and multiplication in $\R$,
as well as divide by non-zero elements. 
Objects of this form are called \emph{fields}.
With the basic arithmetic operations, 
we can form equations of unknowns and ask for their solutions.
The simplest of such equations are \emph{polynomial equations},
i.e. asking for roots of a polynomial. 
One famous example is $X^2 + 1$.
It has no roots in $\R$.
However, we can artificially add in a root, called $i$,
to \emph{extend} $\R$ to some larger field that 
does contain a root of $X^2 + 1$.
(Think about why we only need to add one root.)
This is what the field of complex numbers $\C$ is. 
$\R$ then injects into $\C$.
This is an example of what we call a \emph{field extension}. 

The way in which $\R$ injects into $\C$ is not arbitrary. 
If one takes two real numbers $a, b$, 
then adding them inside the reals followed by injecting yields
the same result as injecting them separately then adding inside the complex numbers.
That is to say, if we let $\iota : \R \to \C$ denote the injection,
then we have \[
    \iota(a + b) = \iota(a) + \iota(b)
\]
The same holds for multiplication. \[
    \iota(a b) = \iota(a)\iota(b)
\]
In particular, $\iota(1) = \iota(1\cdot 1) = \iota(1)^2$,
so \[
    \iota(1) = 1
\]
where the first ``1" is in $\R$ and the second is in $\C$. 
Similarly $\iota(0) = 0$.
In a sense, the field structure of $\R$ is ``preserved'' when passing through $\iota$.
Thus when considering fields, 
functions satisfying the above three conditions are of interest. 
They are called \emph{field morphisms}. 

Note that $\iota$ being a field morphism gives $\C$ a real vector space structure.
The fact that every element of $\C$ is of the form $a + b i$ with real $a, b$
says $\{1,i\}$ is a basis of $\C$ as a real vector space. 

Consider the field morphisms $\sigma$ from $\C$ to itself that ``preserve'' $\R$,
which we shall call $\R$-automorphisms of $\C$.
For now think of these as morphisms that keep $\R$ where it is. 
One way to characterise this is saying $\sigma$ is a $\R$-linear field morphism.\[
    \sigma(\la z + w) = \la \sigma(z) + \sigma(w) \,\,\,\,,\, \la \in \R
\]
What could such a $\sigma$ be? 
Since $\{1,i\}$ is a basis for $\C$, 
$\sigma$ is uniquely determined by $\sigma(1), \sigma(i)$.
But $\sigma$ is a field morphism, so $\sigma(1) = 1$. 
So we only need to figure out what $\sigma(i)$ could be. 
Intuitively, we only need $\sigma(i)$ because 
the data of $\C$ only consisted of : \begin{enumerate}
    \item It is a field.
    \item It has $\R$ sitting in it.
    \item It is obtained by creating a root of $X^2 + 1$, $i$.
\end{enumerate}
Since $\sigma$ preserves the field structure of $\C$ and leaves $\R$ unchanged,
the only data left to play with is regarding $i$. 

To see what $\sigma(i)$ can be, recall that 
$i$ is a root of the real equation $X^2 + 1 = 0$.
Here is the trick : 
since $\sigma$ is a field morphism that preserves $\R$, 
this equation is \emph{also preserved},
i.e. \[
    0 = i^2 + 1 \imp 
    0 = \sigma(0) = \sigma(i^2 + 1) = \sigma(i)^2 + 1
\]
So $\sigma(i)$ is still a root of $X^2 + 1$
and hence can only be $\pm i$. 
This shows that $\sigma$ can only be either the identity function of $\C$
or complex conjugation. 

Both the identity function and complex conjugation are invertible functions,
with inverses that are also $\R$-automorphisms of $\C$. 
This means the set of all $\R$-automorphisms of $\C$ forms a group
under function composition. 
This is called the \emph{Galois group of the extension $\R \to \C$},
denoted $\aut{\R}{\C}$.
Observe that $\aut{\R}{\C}$ is has cardinality two
and the dimension of $\C$ as a real vector space is also two.
This is one part of the Galois correspondence.

Next, consider the fields inside $\C$ that contain $\R$ called 
\emph{subextensions of $\C$}.
Let $L$ be a subextension. What could $L$ be?
Since $L$ contains $\R$, it is also a real vector space,
in fact a real subspace of $\C$. 
The dimension of $L$ is less than or equal to the dimension of $\C$. 
But the dimension of $\C$ is 2, so the dimension of $L$ can only be 1 or 2,
i.e. $L$ is either $\R$ or $\C$.
Observe that since $\aut{\R}{\C}$ is a two element group, 
the number of its subgroups is also 2. 
So the set of subextensions of $\C$ has the same cardinality as 
the set of subgroups of its Galois group.

In fact, we can give an explicit bijection between the subextensions and subgroups.
For a subextension $L$, 
let $\aut{L}{\C}$ denote the group of $L$-automorphisms of $\C$.
This is a subgroup of the Galois group $\aut{\R}{\C}$.
For a subgroup $G$ of $\aut{\R}{\C}$,
let $\C^G$ be the set of all complex numbers fixed by all elements of $G$.
This is a field inside $\C$ containing $\R$, i.e. a subextension of $\C$.
One can verify that these are inverse functions of each other
and thus they give a bijection between 
the subextensions of $\C$ and the subgroups of its Galois group.
Furthermore, the bijection reverses inclusions. 
\begin{figure} [H]
    \centering
    \begin{tikzcd}
    \C \ar[r, "\C^-", rightharpoonup, xshift = 2,yshift = -15] & 
    \<\id{\C}\> \ar[d,"\subseteq"] \\
    \R \ar[u,"\subseteq"]
    \ar[r, "\aut{-}{\C}" below=2, leftharpoondown, swap, xshift = 2, yshift = 15] &
    \aut{\R}{\C}
    \end{tikzcd}
\end{figure}
This is the Galois correspondence for the case of the extension $\R \to \C$. 

The general Galois correspondence is as follows. 
\begin{thm} More or Less Galois Correspondance. 
    
    Let $\iota_L : K \to L$ be a ``nice" field extension.
    \footnote{
        i.e. there are conditions for the Galois correspondence to hold,
        which we do not have the language to describe yet.
    }
    
    Let $E \subseteq L$ be a subextension of $L$. 
    Then the group of $E$-automorphisms of $L$ is
    a subgroup of the Galois group of $L$. \[
        \aut{E}{L} \leq_\Grp \aut{K}{L}
    \]
    Let $G$ be a subgroup of the Galois group of $L$. 
    Then the subfield fixed by $G$, $L^G$, is a subextension of $L$.
    \[ K \overset{\iota_L}{\to} L^G \]
    
    The following are true :  
    \begin{enumerate}
        \item (Inverses)
            For a subgroup $G$ of $\aut{K}{L}$ \[
                G = \aut{L^G}{L}
            \]
            For a subextension $E$ of $L$, \[
                E = L^{\aut{E}{L}}
            \]
            Hence, we have a bijection between subextensions of $L$
            and the subgroups of the Galois group of $L$.
            
            \begin{figure} [H]
                \centering
                \begin{tikzcd}
                \{G \mid G \le \aut{K}{L}\} 
                \ar[r, "L^-", rightharpoonup, 
                start anchor=north east, end anchor=north west, yshift = -5] 
                \ar[r, "\aut{-}{L}" below=2, leftharpoondown, 
                start anchor=south east, end anchor=south west, swap, yshift = 7]
                &  \{E \mid K \to E \subseteq L\} \end{tikzcd}
            \end{figure}
            
            \item (Order Reversing) 
            Let $G, H \leq \aut{K}{L}$ be subgroups such that $G \subseteq H$.
            Then \[ L^H \subseteq L^G \]
            Similarly, 
            let $E, F \subseteq L$ be $K$-subextensions such that $E \subseteq F$.
            Then \[ \aut{F}{L} \subseteq \aut{E}{L} \]
        \item (Degree) 
            Let $E$ be a subextension of $L$. 
            Let $[E : K]$ denote the dimension of $E$ as a vector space over $K$
            and $[\aut{K}{L} : \aut{E}{L}]$ denote the subgroup index of $\aut{E}{L}$.
            Then $[E : K] = [\aut{K}{L} : \aut{E}{L}]$. 
    \end{enumerate}
\end{thm}
We call the above the ``more or less" Galois correspondence because
as the reader will see, the correspondence is quite an elaborate one,
relating many other properties of field extensions to properties of groups. 

\end{document}