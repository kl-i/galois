\documentclass[../book.tex]{subfiles}

\begin{document}
%    -  Def - char
%    -  Lem - char prime
%    -  Lem - char K = p iff K/Fp
%    -  Lem - K/L -> char K = L
%    -  Thm - Classification 1.  char K = p -> o(K) = p^n
%    -  Lem - Freshmans’ Dream
%    -  Lem - K/{x^p^n - x = 0} 
%    -  Lem - char K = p -> x^p^n - x sep in K[x]
%    -  Thm - Classification 2.  o(K) = p^n -> K splitting x^p^n - x
%    -  Thm - Classification 3.  o(K) = o(L) fin -> K ≅ L
%    -  Thm - Classification 4. Exists o(K) = p^n
%    -  Def - Fp^n
%    -  Thm - Classification 5.  Fp^n = Fp(ω), ω prim-(p^n-1)-root
%    -  Thm - Classification 6. Fp^m/Fp^n iff n|m
%    -  Def - Frob(p^n,K/Fp^n)
%    -  Thm - Classification 7. Fp^nm/Fp^n Galois AND Galois Grp Generated by Frob(p^n)
%    -  Thm - Classification 8. (Fp^n,+) ≅ Cp^n AND (Fp^n*, *) ≅ Cp^n-1

\begin{thm} Classification of Finite Fields.
    
    Let $p$ be a prime. Then the following are true. 
    \begin{enumerate}
        \item (Existence)
        
        For every power of $p$, $p^n$, 
        the splitting field of $X^{p^n} - X \in \F_p[X]$
        is a field with $p^n$ elements.
        \item (Uniqueness)
        
        All finite fields have non-zero characteristic and 
        those with characteristic $p$ are in fact 
        the splitting field of $X^{p^n} - X \in \F_p[X]$ for some $p$-power $p^n$. 
        We thus use $\F_{p^n}$ to denote any of these isomorphic fields
        and refer to it as \textbf{the field with $p^n$ elements}. 
        \item (Primitive)
        
        The extension $\iota : \F_p \to \F_{p^n}$ is a simple extension,
        In particular, the Primitive Element theorem holds for finite fields
        and so the Galois correspondence, too. 
        \item (Galois from $\F_p$)
        
        The extension $\iota : \F_p \to \F_{p^n}$ is Galois and 
        its Galois group is a cyclic group generated by 
        the \textbf{Frobenius map}, $Frob(p) : x \mapsto x^p$. 
        \item (Dunno what to call this)
        
        Let $\F_{p^n}$ be a finite field with characteristic $p$. 
        Then for every $d$ that divides $n$, 
        there exists a unique subfield with $p^d$ elements,
        so we can inject $\F_{p^d}$ into $\F_{p^n}$.
        Conversely, if we have a $\F_p$-extension $\F_{p^d} \to \F_{p^n}$,
        then $d$ divides $n$. 
        \item (Galois)
        
        Let $\F_{p^n} \to \F_{p^m}$ be an extension of finite fields. 
        Then this is a Galois extension and 
        its Galois group is a cyclic group generated by 
        the $m/n$-th power of the Frobenius map. 
    \end{enumerate}
    
\end{thm}
\begin{proof}
    
    (1)
        
        The derivative of $X^{p^n} - X$ is $-1$, 
        which is coprime to $X^{p^n} - X$ itself.
        So $X^{p^n} - X$ is a separable polynomial. 
        In particular, it has $p^n$ number of roots. 
        Suppose $a, b$ are both roots.
        Clearly, $ab$ is a root and if $b \neq 0$, then $a / b$ is a root as well.
        Furthermore, by Freshmen's Dream, we have \[
            (a \pm b)^{p^n} = (a^p \pm b^p)^{p^{n-1}} = \cdots
            = a^{p^n} \pm b^{p^n} = a \pm b
        \]
        i.e. the roots of $X^{p^n} - X$ form a field.
        The splitting field must be equal to this, and hence has $p^n$ elements.
        
    (2)
        
        Let $K$ be a finite field. 
        It is easy to show that it has non-zero and hence prime characteristic. 
        So let the characteristic of $K$ be $p$. 
        Then we have a natural extension $\F_p \to K$. 
        Since $K$ is finite as a set, 
        it must have finite dimension as a $\F_p$-vector space.
        So $K$ is isomorphic as a $\F_p$-vector space to $\bigoplus_{i < n} \F_p$ 
        for some positive natural $n$ and hence \[
            |K| = |\bigoplus_{i < n} \F_p| = p^n
        \]
        
        Let $a$ be an element of $K$.
        If it is zero, it is clearly a root.
        If it is non-zero, it is an element of the finite group $(K^\times,\cdot)$,
        so $a^{p^n - 1} = a^{|K^\times|} = 1$.
        Hence $a$ is again a root. 
        We just showed that $K$ consists of $p^n$ roots of $X^{p^n} - X$, 
        i.e. all the roots.
        Thus $K$ is in fact the splitting field of $X^{p^n} - X$.
        
    (3)
        
        To show that it is a simple extension,
        note that the group of units $\F_{p^n}$ is a finite group
        satisfying the condition that the number of solutions to $a^n = 1$ 
        is bounded by $n$. 
        Hence, it is a cyclic group with generator $a$ say. 
        Then clearly $\F_{p^n} = \F_p(a)$. 
        
        Now let $\iota_L : K \to L$ be a finite extension where $K$ is finite.
        Then both $K$ and $L$ have the same non-zero characteristic.
        WLOG it's $p$, which gives $L \iso \F_{p^n}$ as a $\F_p$-extension
        for some positive integer $n$. 
        Then since $\F_{p^n}$ is a simple extension of $\F_p$,
        $L$ is also a simple extension of $\F_p$,
        and thus of $K$ as well. 
        
    (4)
        
        Since $\iota : \F_p \to \F_{p^n}$ is the splitting field of $X^{p^n} - X$, 
        which is separable, it is clear that the extension is Galois.
        We now show that $\iota \F_p = \F_{p^n}^{\<Frob(p)\>}$
        and hence $\aut{\F_p}{\F_{p^n}} = \<Frob(p)\>$ by the Galois correspondence.
        It suffices to check the elements fixed by only the Frobenius map. 
        
        Note that for any element $a \in \F_{p^n}$,
        $a$ is fixed by $Frob(p)$ if and only if it is the root of $X^p - X$. 
        But the set of all roots of $X^p - X$ is $\F_p$!
        So we are done. 
        
    (5)
    
\end{proof}


\begin{dfn} Order of an Element, Cyclic Group.
    
    Let $G$ be a group. 
    For an element $g \in G$, the \textbf{order of $g$} is defined as 
    the smallest positive integer $o(g)$ such that $g^{o(g)} = e$. 
    If no such $o(g)$ exists, we say $g$ has \emph{infinite order}. 
    Note that in the case of $G$ finite, 
    $o(g)$ exists for all $g \in G$ and $o(g)$ is the order of $\<g\>$,
    hence divides the cardinality of $G$ by Lagrange's theorem. 
    Using the division algorithm, one can also show that $o(g)$
    divides any other positive natural $n$ where $g^n = e$. 
    
    $G$ is called a \textbf{cyclic group} when 
    there exists an element $g \in G$ such that $\<g\> = G$. 
    We say \textbf{$g$ generates $G$} or \textbf{$g$ is a generator of $G$}.
    For any natural $n$, $C_n$ denotes the \textbf{cyclic group of order $n$}
    which is anything isomorphic to cyclic group of cardinality $n$. 
\end{dfn}

\begin{lem} Basic Results about Cyclic Groups and a Number-Theoretic Result.
    
    Let $G$ be a finite cyclic group with a generator $g$. 
    Then \begin{enumerate}
        \item For any positive integer $n$, the \textbf{totient of $n$}
        is defined as the number of positive integers coprime to $n$, 
        denoted $\phi(n)$. 
        
        Let $g^k$ be an element. Then $o(g) = |G| / (|G|,k)$.
        Hence, the number of generators of $|G|$ is $\phi(|G|)$
        
        \item Let $H$ be a subgroup of $G$. Then $H$ is cyclic, too. 
        \item Let $n = |G|$. Then \[
            n = \sum_{d \mid n} \phi(d)
        \]
    \end{enumerate}
\end{lem}
\begin{proof}
    (1)
    
        Clearly, $(g^k)^{|G|/(|G|,k)} = e$. 
        We now show it is minimal. 
        Let $(g^k)^n = e$. 
        Then $g^{kn} = e$ implies $|G|$ divides $kn$ since $g$'s order is $|G|$. 
        That is to say, there exists an integer $m$ such that $|G| m = k n$. 
        Dividing by $(|G|,k)$ on both sides gives \[
            \frac{|G|}{(|G|,k)} m = \frac{k}{(|G|,k)} n
        \]
        We leave it to the reader to show that 
        $|G| / (|G|,k)$ and $k / (|G|, k)$ are coprime, 
        and hence $|G| / (|G|,k)$ divides $n$. 
        In particular, it is smaller than or equal to $n$.
        
        We say that $g^k$ is a generators if and only if $(|G|,k) = 1$.
        Hence number of generators is the totient of $|G|$. 
    (2)
        
        If $H = \<e\>$, then $H$ is trivially cyclic.
        So suppose there exists a non-identity element $g^k$ in $H$.
        Since non-empty subsets of $\N$ have a minimal element, 
        WLOG $k$ is the minimal positive integer such that $g^k$ is in $H$.
        We leave it to the reader to check that for any other element $g^m$ in $H$,
        $k$ divides $m$, and hence $g^k$ generates $H$. 
    
    (3)
        
        By assumption, there exists a positive integer $n$ such that $d n_d = |G|$.
        We leave it to the reader to verify that $g^{n_d}$ has order $d$,
        hence the subgroup generated by it has cardinality $d$. 
        This shows existence. 
        
        Let $H$ be another subgroup of order $d$. 
        Then there exists a $g^{m_d}$ that generates $H$. 
        Then $g^{dm_d} = e$ implies $|G|$ divides $dm_d$. 
        It is easy to check that $n_d$ divides $m_d$.
        Hence $g^{m_d}$ is in $H_d$, i.e. all of $H$ is in $H_d$. 
        Both are sets of order $d$, so $H = H_d$.
        
    (4)
        
        The orders of elements in $G$ give a partition of $G$,
        and since orders of elements must divide $|G| = n$, we have \[
            n = \sum_{d \mid n} \{x \in G \mid o(x) = d\}
        \]
        Let $d$ be one of such orders. 
        Then all elements of order $d$ generate subgroups of order $d$,
        so by (3), they are generate the same cyclic subgroup $H_d$. 
        Generators of $H_d$ are clearly order $d$, 
        so elements are order $d$ if and only if they generate $H_d$.
        Let $h$ be a generator of $H_d$ so that 
        all elements of $H_d$ are $h^k$ for some natural number $k$. 
        Since $H_d$ is finite cyclic, 
        the number of generators of $H_d$ is equal to $\phi(d)$. 
        This gives \[
            n = \sum_{d \mid n} \{x \in G \mid o(x) = d\} = \sum_{d \mid n} \phi(d)
        \]
    
\end{proof}

\begin{thm} A Condition for Cyclic, Primitive Element for Finite Fields.
    
    Let $G$ be a finite group such that for all $d$ dividing $|G|$, 
    the number of $x \in G$ satisfying $x^d = e$ is less than or equal to $d$. 
    Then $G$ is cyclic.
    
    In particular, take $G = L^\times$ where $L$ is a finite field.
    Then $L^\times$ is cyclic. 
    Hence if $\iota_L : K \to L$ is a finite extension where $K$ is finite,
    then there exists an element $c \in L$ such that $L = K(c)$.
\end{thm}
\begin{proof}
    
    We again partition $G$ by the orders of elements. 
    Let $G_d$ be the set of elements in $G$ with order $d$. 
    Then we have \[
        |G| = \sum_{d \mid |G|} |G_d|
    \]
    We show that $|G_d| \leq \phi(d)$. 
    Note that $G_d$ is a subset of the elements that satisfy $x^d = e$. 
    Suppose $G_d$ is empty. Then we are done.
    Suppose $G_d$ is non-empty. 
    Then there exists a non-identity element $g$ with order $d$.
    Consider the subgroup generated by $g$, $\<g\>$. 
    All elements in $\<g\>$ satisfy $x^d = e$.
    So $\<g\>$ is a $d$-element subset of the solutions to $x^d = e$,
    which has at most $d$-elements.
    Hence $\<g\>$ \emph{is} the set of solutions to $x^d = e$.
    In particular, all other elements of order $d$ must be in $\<g\>$.
    Generators of $\<g\>$ are also order $d$, so we have $|G_d| = \phi(d)$. 
    
    We then have \[
        |G| = \sum_{d \mid |G|} |G_d| \leq \sum_{d \mid |G|} \phi(d) = |G|
    \]
    Thus $|G_d| = \phi(d)$ for all $d$ dividing $|G|$. 
    In particular, $|G_{|G|}| = \phi(|G|) \neq 0$, 
    so we have an element of order the cardinality of $G$,
    i.e. $G$ is cyclic.
    
    Let $L$ be a finite field. 
    Then elements in $(L^\times,\cdot)$ such that $x^d = e$ are precisely 
    the roots to the polynomial $X^d - 1$.
    The number of such roots is bounded by $\deg (X^d - 1) = d$. 
    $L^\times$ satisfies the condition of the above result, hence must be cyclic.
    
    If $\iota_L : K \to L$ is a finite extension where $K$ is finite,
    then $L$ is also finite, so $L^\times$ is cyclic.
    Let $c$ be a generator of $L^\times$ as a group. 
    Then $K(c)$ clearly contains $L^\times$, and hence is equal to $L$. 
\end{proof}

%
%-  Lem - Dedekind’s 
%-  Thm - Deg Sep leq Deg Ext
%    -  Dedekind


\end{document}