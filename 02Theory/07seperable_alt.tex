\documentclass[../book.tex]{subfiles}

\begin{document}
In this section, we deal with when minimal polynomials having multiple roots.

\begin{rmk}
    The following is intimately related to when
    polynomials can have multiple roots or not.
\end{rmk}

\begin{dfn} Characteristic of a Field.
    
    Let $K$ be a field. 
    Since $\Z$ is generated by $1$ as a ring
    and all ring morphisms $\Z \to K$ must map $1_\Z$ to $1_K$,
    we have only one ring morphism from $\iota_\Z : \Z \to K$. 
    $\ker \iota_\Z$ is an ideal of $\Z$ so by $\Z$ being a PID,
    there exists an element $p \in \Z$
    that that generates $\ker \iota_\Z$ as an ideal.
    $p$ can be made non-negative. 
    Then the \textbf{characteristic of $K$} is defined as this $p$. 
    We will denote it with $\chi_K$. 
\end{dfn}
\begin{thm} Characteristics are Zero or Prime.
    
    Let $K$ be a field and $\chi_K$ its characteristic.
    Then $\chi_K$ is zero or prime. 
\end{thm}
\begin{proof}
    By the 1st isomorphism theorem, 
    \[ \Z/ (\chi_K) = \Z / \ker \iota_\Z \iso_\Ring \iota_\Z \Z \subseteq K\]
    where $\iota_\Z$ is the canonical ring morphism from $\Z$ into $K$.
    Since fields are integral rings
    and subrings of integral domains are integral,
    $\Z / (\chi_K)$ is an integral domain.
    If $\chi_K$ is zero, we are done.
    If $\chi_K$ is non-zero, note that $\iota_\Z$ cannot be the zero map
    since $1 \neq 0$ in the field $K$. 
    Hence $\chi_K$ is cannot be a unit. 
    So $\chi_K$ is a non-zero non-unit 
    which generates an ideal with an integral quotient ring
    i.e. $\chi_K$ is prime.
\end{proof}
\begin{dfn} Multiplicity of Roots.
    
    Let $f$ be a polynomial over $K$ and 
    $\iota_L : K \to L$ an extension splitting $f$.
    Let $S_f$ be the set of roots of $f$ inside $L$. 
    Then we have, \[
            \bar\iota_L f = \iota(\la) \prod_{a\in S_f} (X - a)^{m_a}
    \]
    where $\la \in K$. 
    For a root $a \in S_f$, $m_a$ is called its \textbf{multiplicity}.
    The multiplicities of roots are actually independent of the extension $L$,
    since any $L$ that splits $f$ contains a splitting field of $f$
    and inside the splitting field, $\bar{\iota_L} f$ is the same as in $L$
    as well as inside any other splitting field since 
    any two splitting fields of $f$ are isomorphic. 
    A root $a$ with multiplicity larger than one is called a \textbf{repeated root}.
\end{dfn}
Before we introduce the three equivalent definitions of separability, 
we will need to following two lemmas. 
\begin{lem} GCD stays the same moving up Extensions. 
    
    Let $f, g$ be polynomials over $K$ and $\iota_L : K \to L$ an extension.
    Let $h \in K[X]$ be a gcd of $f, g$. 
    Then $\bar\iota_L h$ is a gcd of $\bar\iota_L f, \bar\iota_L g$. 
    In particular, distinct irreducible polynomials remain coprime 
    when moving up an extension. 
\end{lem}
\begin{proof}
    Let $h_1$ be a gcd of $\bar\iota_L f$ and $\bar\iota_L g$. 
    Clearly, $\bar\iota_L h$ divides $\bar\iota_L f$ and $\bar\iota_L g$. 
    So $h_1$ divides $\bar\iota_L h$. 
    Then any polynomial in $L[X]$ dividing both $\bar\iota_L f$ and $\bar\iota_L g$
    divides $h_1$ and hence divides $\bar\iota_L h$. 
    Thus, $\bar\iota_L h$ is a gcd of $\bar\iota_L f$ and $\bar\iota_L g$. 
\end{proof}
%\begin{rmk}
%    The above lemma is significant because it says
%    if $f$ has a repeated root, then that repeated root belongs to 
%    \emph{only one} of its irreducible factors, 
%    because the gcd of any two distinct irreducible factors is $1$. 
%    Hence, $f$ has a repeated root if and only if it has some
%    irreducible factor with repeated roots or two non-distinct %irreducible factors.
%    This means from now on when considering the issue of repeated roots, 
%    we only need to consider irreducible polynomials. 
%\end{rmk}
\begin{lem} Freshmen's Dream.
    
    Let $K$ be a field where $\chi_K \neq 0$ in $\Z$. 
    Then for all $a, b \in K$, \[
        (a+b)^{\chi_K} = a^{\chi_K} + b^{\chi_K}
    \]
\end{lem}
\begin{proof}
    Follows from the binomial theorem and $\chi_K = 0$ in $K$. 
\end{proof}
%    - Def - Sep Ext, 2 Eqv defs
%          - Contrapositive
%    - Def - Sep Element over a Field, Sep Ext
\begin{dfn} Seperable Polynomials, Separable Extensions.
% Multiple definitions taken from John Milne's Galois notes. 

    Let $f$ be a polynomial over $K$. 
    The following are equivalent: 
    \begin{enumerate}
        \item There exists a $K$-extension $\iota_L : K \to L$ splitting $f$ where
        $f$ has a root with multiplicity higher than 1, i.e. a repeated root.
        \item Suppose $f = \sum_{k \leq \deg f} f_k X^k \in K[X]$.
        Define its formal derivative as 
        \[ f' := \sum_{k < \deg f} (k+1)f_{k+1}X^k \]
        Then $(f,f') \neq 1$.
    \end{enumerate}
    If $f$ is irreducible, then the following is also equivalent: 
    \begin{enumerate} [resume]
        \item $\chi_K$ is non-zero and there exists a polynomial $g$ over $K$
        such that $f = g(X^{\chi_K})$.
    \end{enumerate}
    If any of the above is true, then $f$ is called \textbf{inseparable}.
    Otherwise, it is called \textbf{separable}.
    Note that (3) implies irreducible polynomials over fields with characteristic zero
    are automatically separable. 
    
    Let $\iota_L : K \to L$ be a finite $K$-extension.
    Then $(L,\iota_L)$ is called a \textbf{separable extension} when
    all minimal polynomials of elements are separable. 
\end{dfn}
\begin{proof}
    ($1\imp 2$)
    
        Let $\iota_L : K \to L$ be a $K$-extension 
        and let $a$ be a root of $f$ in $L$ with multiplicity $m_a > 1$. 
        WLOG, let $m_a = 2$, so there exists a polynomial $g$ over $L$ such that \[
            \bar{\iota_L} f = (X - a)^2 g
        \]
        Then by the product rule\footnote{
        Prove it yourself if you can be arsed.
        Strong induction is your friend.}, \[
            \bar{\iota_L} f' = (\bar{\iota_L} f)' = 2(X - a) g + (X - a)^2 g
        \]
        In particular, $(\bar\iota_L f, \bar\iota_L f') \neq 1$
        Suppose $(f,f') = 1$ in $K[X]$. 
        Then $(\bar\iota_L f, \bar\iota_L f') = 1$ since GCD's stay the same. 
        We have a contradiction. Hence $(f,f') \neq 1$. 
        
    ($2 \imp 1$)
        
        Suppose $(f,f') \neq 1$. 
        Let $\iota_L : K \to L$ be the splitting field of $L$. 
        Then $(\bar{\iota_L f},(\bar{\iota_L f})') \neq 1$. 
        Suppose all roots of $f$ have multiplicity 1. 
        Then by the product rule, 
        one can see that no irreducible factor $(X - a)$ of $\bar{\iota_L f}$
        can be a factor of $(\bar{\iota_L} f)'$. 
        Hence $(\bar{\iota_L f},(\bar{\iota_L f})') = 1$, which is a contradiction. 
        So there exists a root of $f$ with multiplicity higher than one. 
        
    We now assume $f$ is irreducible. 
    
    ($2\imp 3$)
    
        By definition, there exists a polynomial $g \in K[X]$ such that 
        \[f = (f,f') g\]
        By irreducibility of $f$, $(f,f')$ is a unit or $g$ is a unit. 
        Since $(f,f') \neq 1$, we have $g$ is a unit, 
        which implies $f$ divides $f'$.
        We know $f$ is non-zero. 
        Suppose $f'$ is non-zero as well.
        Then we have $\deg f \leq \deg f'$ whilst $\deg f' < \deg f$,
        a contradiction. 
        Thus $f'$ must be zero. 
        Let $f = \sum_{k\leq\deg f} f_k X^k$. 
        Then we have \[
            0 = f' = \sum_{k < \deg f} (k+1) f_{k+1} X^k
        \]
        which implies for all $k < \deg f$, $(k+1) f_{k+1} = 0$.
        Let $S$ be the set of $k$ such that $f_{k+1} \ne 0$.
        Then for all $k \in S$, we have $k+1 = 0$ in $K$.
        This implies the kernel of the canonical ring morphism 
        $\Z \to K$ is non-trivial. 
        Hence, $\chi_K$ is non-zero. 
        Furthermore, for all $k \in S$,
        $k+1 = 0$ in $K$ implies there exist $n_k \in Z$ such that 
        $k+1 = n_k\chi_K$ in $\Z$
        since $\chi_K$ generates the kernel of $\Z \to K$. 
        Letting $g := f_0 + \sum_{k \in S} f_{k+1}X^{n_k}$,
        we have $f = g(X^{\chi_K})$.\footnote{
            Composition of polynomials is such that $(X^a)^b=X^{ab}$.
            This comes from the interpretation of polynomials over a field $K$
            as a function from $K$ to itself. 
        }
        
    ($3\imp 1$)
    
        Let $\chi_K \neq 0$ and $f = g(X^{\chi_K})$
        for some polynomial $g \in K[X]$.
        There exists an extension $\iota_M : K \to M$ that splits $g$. 
        Then inside $M[X]$ we have \[
            \bar\iota_M f = \bar\iota_M g(X^{\chi_K})
            = \iota_M(\la) \prod_{i < \deg g} (X^{\chi_K} - b_i)
        \]
        where $\la \in K$ and $b_i \in M$. 
        Furthermore, there exists an extension $\iota : M \to L$ 
        that splits $\bar\iota_M f$.
        Let $(L,\iota_L) := (L,\iota\circ\iota_M)$. 
        Then for all $b_i$, we have an $a_i \in L$ such that $a_i^{\chi_K} = \iota b_i$. 
        Hence we have \begin{align*}
            \bar\iota_L f &= \bar\iota_L g(X^p) 
            = \iota_L(\la) \prod_{i < \deg g}  (X^{\chi_K} - a_i^{\chi_K}) \\
            &= \iota_L(\la) \prod_{i < \deg g}  (X - a_i)^{\chi_K}  
        \end{align*}
        where the second line comes from Freshmen's dream. 
        Thus in fact, \emph{all} the roots of $f$ have multiplicity larger than 1. 
\end{proof}
%    - Def - Deg Sep as Cardinality of Maps into any Normal Extension 
%            (Well def via Maps into Normal Closure)
\begin{dfn} Seperable Degree. 
    
    Let $\iota_L : K \to L$ be an extension. 
    Let $\iota_M : L \to M$ be an extension such that 
    $(M,\iota_M\circ\iota_L)$ is normal. 
    Then the \textbf{separable degree of $(L,\iota_L)$} is
    the number\footnote{... the cardinality of the set of...}
    of $K$-extension morphisms 
    from $(L,\iota_L)$ to $(M,\iota_M\circ\iota_L)$.
    This is denoted $[L : K]_S$. 
    
    We prove below that for any choice of $M$,
    the separable degree is equal to the seperable degree 
    when $M$ is chosen to be the normal closure of $(L,\iota_L)$.
    Hence it is well-defined.
\end{dfn}
\begin{proof}
    Let $(N,\iota_N)$ be the normal closure of $(L,\iota_L)$.
    Let $(M,\iota_M)$ be another $L$-extension such that 
    $(M,\iota_M\circ\iota_L)$ is normal. 
    By the minimal property of the normal closure, 
    we have an $L$-extension $\bar\iota_M : N \to M$. 
    The situation is this. 
    \begin{figure} [H]
        \centering
        \begin{tikzcd} [sep = huge]
        & 
        & 
        N \arrow[d,"\bar{\iota_M}"] \\
        K \arrow[r,"\iota_L"] &
        L \arrow[ru,"\iota_N"{sloped,below}]
        \arrow[ru,"\phi"{sloped,above},xshift=-0.7ex, yshift=0.7ex] 
        \arrow[rd,"\iota_M"{sloped,above}] 
        \arrow[rd,"f"{sloped,below},xshift=-0.7ex, yshift=-0.7ex] &
        \bar{\iota_M} N \arrow[d,"\subseteq"] 
        \arrow[loop right,"\res{\bar{f}}{\bar{\iota_M} N}"] \\
        & 
        &
        M \arrow[loop right,"\bar{f}"]
        \end{tikzcd}
    \end{figure}
    Let $\mor{K}{L}{N}$ denote the set of $K$-extension morphisms
    from $(L,\iota_L)$ to $(N,\iota_N\circ\iota_L)$
    and similarly let $\mor{K}{L}{M}$ denote the same for $(M,\iota_M\circ\iota_L)$. 
    Let $\phi \in \mor{K}{L}{N}$. Then clearly, 
    $\bar\iota_M \circ \phi$ is a $K$-extension morphism 
    from $(L,\iota_L)$ to $(M,\iota_M\circ\iota_L)$. 
    So we have the following function, \[
        (\bar\iota_M \circ - ) : \mor{K}{L}{N} \to \mor{K}{L}{M}, 
        \phi \mapsto \bar\iota_M \circ \phi
    \]
    We claim this is a bijection. 
    Since $\bar\iota_M$ is injective, $(\bar\iota_M \circ -)$ is injective as well.
    
    We now show surjectivity. 
    Let $f \in \mor{K}{L}{M}$. 
    Then we have an $L$-extension $\bar{f} : (M,\iota_M) \to (M,f)$
    since two embeddings into a normal extension differ by an automorphism. 
    Then $\bar\iota_M$ and $\bar{f}\circ\bar\iota_M$ are
    both $K$-extension morphisms from
    $(N,\iota_N\circ\iota_L)$ to $(M,\iota_M\circ\iota_L)$.
    Hence by image invariance definition of normality, 
    the restriction of $\bar{f}$ to $\bar{\iota_M} N$ is 
    a well-defined bijective $L$-extension morphism, \[
        \res{\bar{f}}{\bar{\iota_M} N} : \bar{\iota_M} N \to \bar{\iota_M} N,
        x \mapsto \bar{f}(x)
    \]
    Then $\bar{\iota_M}^{-1} \circ \res{\bar{f}}{\bar{\iota_M}N} 
    \circ \bar{\iota_M} \circ \iota_N $ is a $K$-extension morphism
    from $(L,\iota_L)$ to $(N,\iota_N)$. 
    This clearly maps to $f$ via $(\bar{\iota_M} \circ -)$. 
\end{proof}
We now have the a second tower law. 
%    - Thm - Tower Law of Deg Sep
\begin{thm} Tower Law of Seperable Degree. 
    
    Let $K \overset{\iota_L}{\to} L \overset{\iota_M}{\to} M$ be extensions.
    Then $[M : K]_S = [M : L]_S [L : K]_S$.
\end{thm}
\begin{proof}
    Let $\iota_N : M \to N$ be an extension such that 
    $(N,\iota_N\circ\iota_M\circ\iota_L)$ is normal. 
    Let $\mor{K}{M}{N}, \mor{K}{L}{N}$ be the sets of $K$-extensions
    from respectively $(M,\iota_M\circ\iota_L)$ and $(L,\iota_L)$ 
    to $(N,\iota_N\circ\iota_M\circ\iota_L)$. 
    Then $[M:K]_S = |\mor{K}{M}{N}|$ and $[L:K]_S = |\mor{K}{L}{N}|$.
    Similarly, let $\mor{L}{M}{N}$ be the set of $L$-extensions
    from $(M,\iota_M)$ to $(N,\iota_N\circ\iota_M)$ so that 
    $[M:L]_S = |\mor{L}{M}{N}|$.
    We want to show \[
        |\mor{K}{M}{N}| = |\mor{L}{M}{N}| |\mor{K}{L}{N}|
    \]
    We start by noting that by restricting to the image of $L$, 
    the $K$-extension morphisms from $M$ to $N$,
    we obtain $K$-extension morphisms from $L$ to $N$, i.e.
    \[
        (- \circ \iota_M) : \mor{K}{M}{N} \to \mor{K}{L}{N}, 
        \phi \mapsto \phi\circ\iota_M
    \]
    is a well-defined function. 
    We will show this is surjective and 
    the preimages of elements in $\mor{K}{L}{N}$ 
    all have cardinality $|\mor{L}{M}{N}|$.
    Then since the preimages of elements in $\mor{K}{L}{N}$ form
    a partition of $\mor{K}{M}{N}$, we have our desired result. 
    
    We now show surjectivity. 
    Let $f \in \mor{K}{L}{N}$. 
    The situation is this: \begin{figure}[H]
        \centering
        \begin{tikzcd}
        K \arrow[r,"\iota_L"] &
        L \arrow[r,"\iota_M"] \arrow[rd,"f"{sloped,below}] &
        M \arrow[d,"\iota_N"] \\
        & 
        &
        N \arrow[loop right,"\bar{f}"]
        \end{tikzcd}
    \end{figure}
    Then since two embeddings into a normal extension differ by an automorphism,
    we have an $L$-extension morphism $\bar{f}$ 
    from $(N,\iota_N\circ\iota_M)$ to $(N,f)$. 
    Then $\bar{f}\circ\iota_N : M \to N$ is clearly
    a $K$-extension morphism from $(M,\iota_M\circ\iota_L)$ 
    to $(N,\iota_N\circ\iota_M\circ\iota_L)$ 
    that gets sent to $f$ under $(- \circ \iota_M)$. 
    
    Let $f \in \mor{K}{L}{N}$. 
    We now show that $(- \circ \iota_M)^{-1}(f)$ bijects with $\mor{L}{M}{N}$.
    Let $\phi \in (- \circ \iota_M)^{-1}(f)$.
    Then \[
        \phi\circ\iota_M = f = \bar{f}\circ\iota_N\circ\iota_M 
        \imp \bar{f}^{-1}\circ\phi\circ\iota_M = \iota_N\circ\iota_M 
    \]
    which says $\bar{f}^{-1}\circ\phi$ is a $L$-extension morphism
    from $(M,\iota_M)$ to $(N,\iota_N\circ\iota_M)$.
    That is to say we have the following function, \[
        (\bar{f}^{-1} \circ -) : (- \circ \iota_M)^{-1}(f) \to \mor{L}{M}{N},
        \phi \mapsto \bar{f}^{-1}\circ\phi
    \]
    By injectivity of $\bar{f}^{-1}$, $(\bar{f}^{-1} \circ -)$ is injective.
    Let $\psi \in \mor{L}{M}{N}$. Then \[
        \bar{f}\circ\psi\circ\iota_M = \bar{f}\circ\iota_N\circ\iota_M = f
    \]
    i.e. $\bar{f}\circ\psi \in (- \circ \iota_M)^{-1}(f)$.
    Clearly, $\bar{f}\circ\psi$ maps to $\psi$.
    This proves surjectivity and completes the proof. 
\end{proof}
We now move towards three equivalent charactersations of separable extensions. 
First, a lemma. 
%    - Lem - Separability Inheritance
\begin{lem} 
    
    Let $K \overset{\iota_L}{\to} L \overset{\iota_M}{\to} M$ be extensions
    and $a \in M$, $b \in L$. 
    Then $\min(a,K)$ separable implies $\min(a,L)$ separable
    and $\min(\iota_M(b),K)$ seperable implies $\min(b,K)$ separable.
    Consequently, $(M,\iota_M\circ\iota_L)$ separable implies
    both $(L,\iota_L)$ and $(M,\iota_M)$ separable. 
\end{lem}
\begin{proof}
    Let $\min(a,K)$ be separable.
    $\min(a,L)$ divides the image of $\min(a,K)$. 
    Since $\min(a,K)$ is separable, it has no repeated roots,
    which clearly implies $\min(a,L)$ also has no repeated roots. 
    Hence $\min(a,L)$ is separable. 
    
    Let $\min(\iota_M(b),K)$ be separable. 
    Since $\min(\iota_M(b),K) = \min(b,K)$, $\min(b,K)$ is separable, too. 
\end{proof}
%    - Thm - For K(a), Deg Sep = Deg Ext iff a sep over K 
%          - Key : Roots <-> Maps from K(a)
%    - Lem - Simp Ext Sep iff Sep Deg = Ext Deg
%          - Key : Ineq Trick
\begin{thm} Equivalent Conditions for Simple Algebraic 
Extensions to be Separable.

    Let $\iota : K \to K(a)$ be a simple extension where 
    $a$ is algebraic over $K$.
    Then the following are equivalent: 
    \begin{enumerate}
        \item $\min(a,K)$ is separable.
        \item $[K(a) : K]_S = [K(a) : K]$
        \item $\iota : K \to K(a)$ is a separable extension. 
    \end{enumerate}
\end{thm}
\begin{proof}
    ($1\iff 2$)
    
        Let $(N,\iota_N)$ be a $K(a)$-extension where 
        $(N,\iota_N\circ\iota)$ is normal. 
        Then $N$ splits $\min(a,K)$. 
        Hence by embedding via conjugates for simple extensions,
        the number of $K$-extension morphisms 
        from $(K(a),\iota)$ to $(N,\iota_N\circ\iota)$
        is equal to the number of distinct Galois conjugates of $a$
        inside $N$. 
        This is equal to $\deg\min(a,K)$, which equals $[K(a):K]$,
        if and only if $\min(a,K)$ is separable. 
        
    ($3\imp 1$) 
        Trivial.
    
    ($2 \imp 3$)
        Let $b \in K(a)$. We use $1,2$ on $b$ to prove this implication. 
        Let $(N,\iota_N)$ be a $K(a)$-extension such that 
        $N$ is normal as a $K(b)$-extension. 
        We have the following trick of inequalities. 
        By embedding via conjugates from $K(b)$ to $N$, 
        we have $[K(a):K(b)]_S \leq [K(a):K(b)]$, and 
        $[K(b):K]_S \leq [K(b):K]$.
        Then by the tower laws of separable degree and extension degree,
        \begin{align*}
            [K(a) : K]_S &= [K(a):K(b)]_S [K(b):K]_S \\
            &\leq [K(a):K(b)] [K(b):K]_S \\
            &\leq [K(a):K(b)] [K(b):K] = [K(a):K] = [K(a):K]_S
        \end{align*}
        Hence we have equality everywhere, giving 
        $[K(b):K]_S = [K(b):K]$ which implies $\min(b,K)$ is separable.
        Thus, $K(a)$ is a separable extension. 
\end{proof}
The generalisation of the above to finite extensions is what we are after. 
%    - Thm - General Ext Sep iff Sep Deg = Ext Deg
%          - Key : Decompose into Simp Exts AND Ineq Trick
%    - Cor - Ext Sep iff Generators Sep
%          - Key : Decompose into Simp Ext, Tower Law
\begin{thm} 3 Equivalent Definitions for Separable Extensions.
    
    Let $\iota_L : K \to L$ be a finite extension.
    Then the following are equivalent:
    \begin{enumerate}
        \item There exists a finite set of generators $a_0,\dots,a_{n-1}$ of $L$ 
            such that for all $a_i$, $\min(a_i,K)$ is separable. 
        \item $[L:K]_S = [L:K]$.
        \item $(L,\iota_L)$ is separable.
    \end{enumerate}
\end{thm}
\begin{proof}
    ($1\imp 2$)
        
        We break $L$ into finite steps of simple extensions. 
        For $i\in n$, let $K_i := K(a_0,\dots,a_i)$.
        Then $K_{i+1} = K_i(a_{i+1})$. 
        Since $\min(a_{i+1},K)$ is separable, $\min(a_{i+1},K_i)$ is separable as well. 
        Hence by the previous lemma, $[K_{i+1}:K_i]_S = [K_{i+1}:K_i]$. 
        So by the tower laws of separable degree and extension degree applied $n$-times, 
        $[L : K]_S = [L : K]$. 
        
    ($2\imp 3$)
        
        Let $b \in L$. 
        This is the same inequality trick.
        By embedding via conjugates, we have $[L : K(b)]_S \leq [L : K(b)]$. 
        Then by the tower laws of separable degree and extension degree,
        \begin{align*}
            [L : K]_S &= [L : K(b)]_S [K(b) : K]_S \\
            &\leq [L : K(b)] [K(b) : K]_S \\
            &\leq [L : K(b)] [K(b) : K] = [L : K] = [L : K]_S 
        \end{align*}
        So by everywhere equality, we have $[K(b) : K]_S = [K(b) : K]$
        which implies $\min(b,K)$ is separable. 
        
    ($3\imp 1$)
        Trivial. 
\end{proof}
%    - Cor - Splitting Field of Sep Poly is Sep
%    - Cor - Ext Sep iff Normal Closure Sep

    
\end{document}