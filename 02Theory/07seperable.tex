\documentclass[../book.tex]{subfiles}

\begin{document}
In this section, we deal with when minimal polynomials having multiple roots.

\begin{rmk}
    The following turns out to be intimately related to when
    polynomials can have multiple roots or not.
\end{rmk}

\begin{dfn} Characteristic of a Field.
    
    Let $K$ be a field. 
    Since $\Z$ is generated by $1$ as a ring
    and all ring morphisms $\Z \to K$ must map $1_\Z$ to $1_K$,
    we have only one ring morphism from $\iota_\Z : \Z \to K$. 
    $\ker \iota_\Z$ is an ideal of $\Z$ so by $\Z$ being a PID,
    there exists an element $p \in \Z$
    that that generates $\ker \iota_\Z$ as an ideal.
    $p$ can be made non-negative. 
    Then the \textbf{characteristic of $K$} is defined as this $p$. 
    We will denote it with $\chi_K$. 
\end{dfn}
\begin{thm} Characteristics are Zero or Prime.
    
    Let $K$ be a field and $\chi_K$ its characteristic.
    Then $\chi_K$ is zero or prime. 
\end{thm}
\begin{proof}
    By the 1st isomorphism theorem, 
    \[ \Z/ (\chi_K) = \Z / \ker \iota_\Z \iso_\Ring \iota_\Z \Z \subseteq K\]
    where $\iota_\Z$ is the canonical ring morphism from $\Z$ into $K$.
    Since fields are integral and subrings of integral domains are integral,
    $\Z / (\chi_K)$ is an integral domain.
    If $\chi_K$ is zero, we are done.
    If $\chi_K$ is non-zero, note that $\iota_\Z$ cannot be the zero map
    since $1 \neq 0$ in the field $K$. 
    Hence $\chi_K$ is cannot be a unit. 
    So $\chi_K$ is a non-zero non-unit 
    which generates an ideal with an integral quotient ring
    i.e. $\chi_K$ is prime.
\end{proof}
\begin{dfn} Multiplicity of Roots.
    
    Let $f$ be a polynomial over $K$ and 
    $\iota_L : K \to L$ an extension splitting $f$.
    Suppose we have, \[
            \bar\iota_L f = \bar\iota(\la) \prod_{i\in n} (X - a_i)^{m_i}
    \]
    where $\la \in K$ and $a_i \in L$. 
    For a root $a_i$ of $f$, $m_i$ is called its \textbf{multiplicity}.
    This is actually independent of the extension $L$,
    since $\bar\iota_L f$ is the same over $K(\{a_i\}_{i\in n})$,
    the splitting field of $f$ and any two splitting fields are isomorphic. 
    A root $a_i$ with multiplicity larger than one is called \textbf{repeated}.
\end{dfn}
Before we introduce the three equivalent definitions of separability, 
we will need to following two lemma. 
\begin{lem} GCD stays the same moving up Extensions.
    
    Let $f, g$ be polynomials over $K$ and $\iota_L : K \to L$ an extension.
    Let $h \in K[X]$ be a gcd of $f, g$. 
    Then $\bar\iota_L h$ is a gcd of $\bar\iota_L f, \bar\iota_L g$. 
    In particular, distinct irreducible polynomials remain coprime 
    when moving up an extension. 
\end{lem}
\begin{proof}
    Let $h_1$ be a gcd of $\bar\iota_L f$ and $\bar\iota_L g$. 
    Clearly, $\bar\iota_L h$ divides $\bar\iota_L f$ and $\bar\iota_L g$. 
    So $h_1$ divides $\bar\iota_L h$. 
    Then any polynomial in $L[X]$ dividing both $\bar\iota_L f$ and $\bar\iota_L g$
    divides $h_1$ and hence divides $\bar\iota_L h$. 
    Thus, $\bar\iota_L h$ is a gcd of $\bar\iota_L f$ and $\bar\iota_L g$. 
\end{proof}
\begin{rmk}
    The above lemma is significant because it says
    if $f$ has a repeated root, then that repeated roots belongs to 
    \emph{only one} of its irreducible factors. 
    Hence, $f$ has a repeated root if and only if it has some irreducible factor
    with repeated roots. 
    This means from now on when consider the issue of repeated roots, 
    we only need to consider irreducible polynomials. 
\end{rmk}
\begin{lem} Freshmen's Dream.
    
    Let $K$ be a field where $\chi_K \neq 0$. 
    Then for all $a, b \in K$, \[
        (a+b)^{\chi_K} = a^{\chi_K} + b^{\chi_K}
    \]
\end{lem}
\begin{proof}
    Follows from the binomial theorem and $\chi_K = 0$. 
\end{proof}
%    - Def - Sep Ext, 2 Eqv defs
%          - Contrapositive
\begin{dfn} Seperable, Separable Extensions
% Multiple definitions taken from John Milne's Galois notes. 
    Let $f$ be an irreducible polynomial over $K$. 
    The following are equivalent: 
    \begin{enumerate}
        \item There exists a $K$-extension $\iota_L : K \to L$ splitting $f$ where
        $f$ has a root with multiplicity higher than 1, i.e. a repeated root.
        \item Suppose $f = \sum_{k \in \deg f} f_k X^k \in K[X]$.
        Define its formal derivative as 
        \[ f' := \sum_{k \in \deg f} (k+1)f_{k+1}X^k \]
        Then $(f,f') \neq 1$.
        \item $\chi_K$ is non-zero and there exists a polynomial $g$ over $K$
        such that $f = g(X^{\chi_K})$.
    \end{enumerate}
    If any of the above is true, then $f$ is called \textbf{inseparable}.
    Otherwise, it is called \textbf{separable}.
    
    Let $\iota_L : K \to L$ be a finite $K$-extension.
    Then $(L,\iota_L)$ is called a \textbf{separable extension} when
    all minimal polynomials of elements are separable. 
\end{dfn}
\begin{proof}
    ($1\imp 2$)
    
        Let $\iota_L : K \to L$ be a $K$-extension where \[
            \bar\iota_L f = \iota_L(\la) \prod_{i\in n} (X - a_i)^{m_i}
        \]
        WLOG $m_0 > 1$. WLOG again, let $m_0 = 2$.
        Suppose $(f,f') = 1$ in $K[X]$. 
        Then by the previous lemma, $(\bar\iota_L f, \bar\iota_L f') = 1$. 
        However, by the product rule, $\bar\iota_L f'$ has a factor of $(X - a_0)$
        which $\bar\iota_L f$ also has as a factor, i.e.
        $(\bar\iota_L f, \bar\iota_L f') \neq 1$.
        We have a contradiction. 
    
    ($2\imp 3$)
    
        By definition, there exists a polynomial $g \in K[X]$ such that 
        \[f = (f,f') g\]
        By irreducibility of $f$, $(f,f')$ is a unit or $g$ is a unit. 
        Since $(f,f') \neq 1$, we have $g$ is a unit, 
        which implies $f$ divides $f'$.
        We know $f$ is non-zero. 
        Suppose $f'$ is non-zero as well.
        Then we have $\deg f \leq \deg f'$ whilst $\deg f' < \deg f$,
        a contradiction. 
        Thus $f'$ must be zero. 
        Let $f = \sum_{k\in\deg f + 1} f_k X^k$. 
        Then we have \[
            0 = \sum_{k\in \deg f} (k+1) f_{k+1} X^k
        \]
        which implies for all $k\in\deg f$, $(k+1) f_{k+1} = 0$.
        Then on the non-zero coefficients of $f$, we have $k+1 = 0$.
        This implies the kernel of the canonical ring morphism $\Z \to K$
        is non-trivial. 
        Hence, $\chi_K$ is non-zero. 
        Furthermore, for $k\in\deg f$ where $f_{k+1} \neq 0$, 
        $k+1 = 0$ implies $k+1 = m_k\chi_K$ 
        since $\chi_K$ generates the kernel of $\Z \to K$. 
        Letting $g := f_0 + \sum_{k\in\deg f} f_{k+1}X^{m_k}$,
        we have $f = g(X^{\chi_K})$.
        
    ($3\imp 1$)
    
        Let $\chi_K \neq 0$ and $f = g(X^{\chi_K})$
        for some polynomial $g \in K[X]$.
        There exists an extension $\iota_M : K \to M$ that splits $g$. 
        Then inside $M$ we have \[
            \bar\iota_M f = \bar\iota_M g(X^{\chi_K})
            = \iota_M(\la) \prod_{i\in\deg g+1} (X^{\chi_K} - b_i)
        \]
        where $\la \in K$ and $b_i \in M$. 
        Furthermore, there exists an extension $\iota : M \to L$ 
        that splits $\bar\iota_M f$.
        Let $(L,\iota_L) := (L,\iota\circ\iota_M)$. 
        Then for all $b_i$, we have an $a_i \in L$ such that $a_i^{\chi_K} = b_i$. 
        Hence we have \begin{align*}
            \bar\iota_L f &= \bar\iota_L g(X^p) 
            = \iota_L(\la) \prod_{i\in\deg g + 1}  (X^{\chi_K} - a_i^{\chi_J}) \\
            &= \iota_L(\la) \prod_{i\in\deg g + 1}  (X - a_i)^{\chi_K}  
        \end{align*}
        where the second line comes from Freshmen's dream. 
        Thus in fact, \emph{all} the roots of $f$ have multiplicity larger than 1. 
\end{proof}
%    - Def - Deg Sep as Cardinality of Maps into any Normal Extension 
%            (Well def via Maps into Normal Closure)
%    - Def - Sep Element over a Field, Sep Ext
%    - Lem - Separability Inheritance
%    - Thm - Tower Law of Deg Sep
%    - Thm - For K(a), Deg Sep = Deg Ext iff a sep over K 
%          - Key : Roots <-> Maps from K(a)
%    - Lem - Simp Ext Sep iff Sep Deg = Ext Deg
%          - Key : Ineq Trick
%    - Thm - General Ext Sep iff Sep Deg = Ext Deg
%          - Key : Decompose into Simp Exts AND Ineq Trick
%    - Cor - Ext Sep iff Generators Sep
%          - Key : Decompose into Simp Ext, Tower Law
%    - Cor - Splitting Field of Sep Poly is Sep
%    - Cor - Ext Sep iff Normal Closure Sep


\end{document}