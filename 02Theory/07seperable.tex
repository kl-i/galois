\documentclass[../book.tex]{subfiles}

\begin{document}
In this section, we deal with when minimal polynomials having multiple roots.

\begin{rmk}
    The following turns out to be intimately related to when
    polynomials can have multiple roots or not.
\end{rmk}

\begin{dfn} Characteristic of a Field.
    
    Let $K$ be a field. 
    Since $\Z$ is generated by $1$ as a ring
    and all ring morphisms $\Z \to K$ must map $1_\Z$ to $1_K$,
    we have only one ring morphism from $\iota_\Z : \Z \to K$. 
    $\ker \iota_\Z$ is an ideal of $\Z$ so by $\Z$ being a PID,
    there exists an element $p \in \Z$
    that that generates $\ker \iota_\Z$ as an ideal.
    $p$ can be made non-negative. 
    Then the \textbf{characteristic of $K$} is defined as this $p$. 
    We will denote it with $\chi_K$. 
\end{dfn}
\begin{clm} Characteristics are Zero or Prime.
    
    Let $K$ be a field and $\chi_K$ its characteristic.
    Then $\chi_K$ is zero or prime. 
\end{clm}
\begin{proof}
    By the 1st isomorphism theorem, 
    \[ \Z/ (\chi_K) = \Z / \ker \iota_\Z \iso_\Ring \iota_\Z \Z \subseteq K\]
    where $\iota_\Z$ is the canonical ring morphism from $\Z$ into $K$.
    Since fields are integral and subrings of integral domains are integral,
    $\Z / (\chi_K)$ is an integral domain.
    If $\chi_K$ is zero, we are done.
    If $\chi_K$ is non-zero, note that $\iota_\Z$ cannot be the zero map
    since $1 \neq 0$ in the field $K$. 
    Hence $\chi_K$ is cannot be a unit. 
    So $\chi_K$ is a non-zero non-unit 
    which generates an ideal with an integral quotient ring
    i.e. $\chi_K$ is prime.
\end{proof}
%    - Def - Sep Ext, 2 Eqv defs
%          - Contrapositive
\begin{dfn} Seperable, Separable Extensions
% Multiple definitions taken from John Milne's Galois notes. 
    
    Let $f$ be an irreducible polynomial over $K$. 
    The following are equivalent: 
    \begin{enumerate}
        \item There exists a $K$-extension $\iota_L : K \to L$ splitting $f$ where
        $f$ has non-distinct roots, i.e.
        \[
            \bar\iota_L f = \la \prod_{i\in n} (X - a_i)^{m_i}
        \]
        where $\la \in K, n < \deg f$. 
        For a root $a_i$ of $f$, $m_i$ is called its \textbf{multiplicity}.
        \item Suppose $f = \sum_{k \in \deg f} f_k X^k$.
        Define its formal derivative as 
        \[ f' := \sum_{k \in \deg f - 1} (k+1)f_{k+1}X^k \]
        Then $(f,f') = 1$.
        \item $\chi_K$ is non-zero and there exists a polynomial $g$ over $K$
        such that $f = g(X^{\chi_K})$.
        \item 
    \end{enumerate}
\end{dfn}
%    - Def - Deg Sep as Cardinality of Maps into any Normal Extension 
%            (Well def via Maps into Normal Closure)
%    - Def - Sep Element over a Field, Sep Ext
%    - Lem - Separability Inheritance
%    - Thm - Tower Law of Deg Sep
%    - Thm - For K(a), Deg Sep = Deg Ext iff a sep over K 
%          - Key : Roots <-> Maps from K(a)
%    - Lem - Simp Ext Sep iff Sep Deg = Ext Deg
%          - Key : Ineq Trick
%    - Thm - General Ext Sep iff Sep Deg = Ext Deg
%          - Key : Decompose into Simp Exts AND Ineq Trick
%    - Cor - Ext Sep iff Generators Sep
%          - Key : Decompose into Simp Ext, Tower Law
%    - Cor - Splitting Field of Sep Poly is Sep
%    - Cor - Ext Sep iff Normal Closure Sep


\end{document}